\clearpage{\thispagestyle{empty}}
\section{Representation Theory of Finite Groups}

\marginnote{Include note on source!} 

%\paragraph{First definitions.}

The main purpose of this text is to study linear representations of groups, or in other words, studying groups and identifying them with vector spaces and their elements with maps of those spaces. Only finite groups and finite-dimensional vector spaces will be discussed.

Let $G$ be a finite group written multiplicatively. Denote its size by $|G|$ and the action of its elements with $\cdot$, for example by $g \cdot x$, sometimes shorted to $gx$.

Let $V$ be a finite-dimensional vector space over the field of complex numbers $\CC$. The group of invertible linear transformations of $V$ is denoted by $\GL(V)$. If $V$ is provided with a basis, usually denoted $(\bas_i)_{i=1}^n$ where $n = \dim V$, then $\GL(V)$ is identified with the set of linear transformations of $V$, denoted $\GL_n(\CC)$\cite[18.1]{DummitFoote}.

\begin{definition}[Representation]\cite{Serre}
	A representation of a group $G$ in the vector space $V$ is a homomorphism
	\begin{align}\label{eq:DefRepr}
		\rho: G \rightarrow \GL(V).
	\end{align}
\end{definition}
For every $g$ in $G$, there is a linear map $\rho_g$ in $\GL(V)$ with an action on the elements of $V$ defined by 
\begin{align*}
	g \cdot \vvec = \rho_g (\vvec).
\end{align*}
The linearity of the map means that for any $\vvec, \wvec \in V$,
\begin{align}\label{eq:linearityOfg}
	g \cdot (a\vvec + b\wvec) = ag\cdot\vvec + bg\cdot\wvec,
\end{align} 
with $a,b \in \CC$ and also since it is a homomorphism, for any $g,h \in G$, we have
\begin{align}\label{eq:homomorphism}
	\rho_{gh} = \rho_g \rho_h.
\end{align}
From the homomorphism of $\rho$, two consequences follow.
\begin{proposition}\label{prop:homoidinv}
	The homomorphism preserves identity and inverses. For the identity element $e$ of $G$ and an arbitrary element $g$ with inverse $g^{-1}$ in $G$ we have,
	\begin{align}\label{eq:rhoIdAndInv}
		\rho_e = \id \quad \text{and} \quad \rho_g^{-1} = \rho_{g^{-1}},
	\end{align}
	where $\id$ is the identity transformation and $\rho_{g^{-1}}$ is the map associated with the inverse of $g$.
\end{proposition}
\begin{proof}
	Take $g$ as an arbitrary element of $G$. The first identity follows from taking $h=e$ in Equation~\ref{eq:homomorphism}:
	\begin{align*}
		\rho_{eg} = \rho_e \rho_g.
	\end{align*}
	Since $eg=g$ for any $g\in G$ we must have that $\rho_g = \rho_e\rho_g$ which is true if and only if $\rho_e = \id$. Now instead taking $h=g^{-1}$ we have:
	\begin{align*}
		\rho_{gg^{-1}} = \rho_g \rho_{g^{-1}},
	\end{align*}
	but $gg^{-1} = e$ and $\rho_e = \id$, implying $\rho_{g^{-1}} = \rho_g^{-1}$.
\end{proof}

\begin{notation}
	A vector space $V$ provided with such a homomorphism discussed above is said to be a \textit{representation space} of $G$\footnote{In fact, this gives $V$ the structure of a $G$-module \cite[1.3]{Sagan}.}.
\end{notation}

\begin{note}
	The homomorphism $\rho$, the set of maps $\{\rho_g\}_{g \in G}$ and the representation space $V$ are interchangeably and abusively called the \textit{representation of $G$}.
\end{note}

%\paragraph{Choice of basis.} 

If a basis $(\bas_i)_{i=1}^n$ is provided for a representation space, then a \textit{matrix representation} can be provided. In this case, $\rho_g$ is the matrix in $\GL_n(\CC)$ associated with the action of $g$ expressed in the basis provided. A matrix representation is not canonical and is dependent on the basis chosen.

\begin{definition}[Degree of a representation]
	Let $V$ be a representation space of $G$. The dimension of $V$ is referred to as the degree of the representation.
\end{definition}