\subsection{Character table of $S_3$}

\begin{theorem}[Character of the permutation representation]
	The character of a permutation representation is the number of fixed points of the associated element of the symmetric group.
\end{theorem}
\begin{proof}
	The character is the sum of the diagonal elements of the matrix representation. The matrices of the permutation representations are permutation matrices, which have a 1 on the diagonal if that row represents a fixed point and 0 otherwise. Therefore the number of fixed point is the sum along the diagonal.
\end{proof}

The representations of $S_3$ found so far in sections \ref{sect:trivrepr} and \ref{sect:regS3} and examples \ref{ex:permS3} and \ref{ex:standS3} are presented in table \ref{table:charS3}. The group is presented in their conjugacy classes $[\sigma] \subset S_3$ where $\sigma$ is any representative of that class. Beneath every class is the number of elements in that class. Rows below the dashed horizontal line are not irreducible.

\begin{table}[hbt!]
	\centering
	\begin{tabular}{c | c c c}
		     $S_3$      & $[1]$   & $[12]$  & $[123]$ \\
		$\Ss|[\sigma]|$ & $\Ss 1$ & $\Ss 3$ & $\Ss 2$ \\ \hline
		   $\chi_T$     & 1       & 1       & 1       \\
		   $\chi_A$     & 1       & -1      & 1       \\
		   $\chi_S$     & 2       & 0       & -1      \\ \hline\hline
		   $\chi_P$     & 3       & 1       & 0       \\
		   $\chi_R$     & 6       & 0       & 0
	\end{tabular}
	\caption{Character table of $S_3$.}
	\label{table:charS3}
\end{table}

\paragraph{Decomposition of $P$.} In section $\ref{sect:standSn}$, the permutation representation was decomposed through geometric arguments to the direct sum of the trivial and the standard representation, this is seen in the table as $\chi_T + \chi_S = (1,1,1) + (2,0,-1) = (3,1,0) = \chi_P$, confirming that $P = T \oplus S$. 

\paragraph{Decomposition of $R$.} In section \ref{sect:regS3}, the matrices of the regular representation was exhaustively calculated. The character of $R$ thus is $\chi_R = (6,0,0)$, since the identity element is the only element in $S_3$ that leaves every other element fixed. In fact this is true for any $S_n$:
\begin{theorem}[Character of the regular representation of $S_n$]\label{thm:charregSn}
	 The character of the regular representation for any $S_n$ is $(n!, 0, \dots, 0)$, where $n!$ is the size of $S_n$.
\end{theorem}
\begin{proof}
	The center of $S_n$ is trivial.
\end{proof}

To decompose $R$, we take the inner product of $\chi_R$ with every other irreducible character.
\begin{align}
	\bra{\chi_T}\ket{\chi_R} = 1, \quad
	\bra{\chi_A}\ket{\chi_R} = 1, \quad
	\bra{\chi_S}\ket{\chi_R} = 2, 
\end{align}
so $\chi_R = \chi_T + \chi_A + 2\chi_S$ and $R = T \oplus A \oplus 2S$.

Also:
\begin{theorem}
	If $V_i$ are the irreducible representations of a group, then the regular representation $R$ is the direct sum of $\dim V_i$ copies of $V_i$, ie. 
	\begin{align}
		R = \bigoplus_{i=1}^{k} (\dim V_i) \cdot V_i = \bigoplus_{i=1}^{k} \chi_{V_i}(e) \cdot V_i .
	\end{align}
\end{theorem}
\begin{proof}
	It came to me in a dream. %Needs a proof and/or source, noticed while staring at character tables.
\end{proof}

\paragraph{Decomposition of $S \otimes S$.} The character of $S \otimes S$ is $\chi_{S \otimes S} = \chi_S \cdot \chi_S = (4,0,1)$, which by a quick glance on the character table is seen to be the sum of all irreducibles, ie. $\chi_{S \otimes S} = \chi_T + \chi_A + \chi_S$ and $S \otimes S = T \oplus A \oplus S$.

\paragraph{Decomposition of $S^{\otimes n}$.}\cite[Exercise 2.7.]{FultonHarris} To find the decomposition of larger tensor powers than $n=2$, we study the character $\chi_{S^{\otimes n}} = (2^n, 0, (-1)^n)$ and take the inner product of it with the other irreducibles and find:
\begin{align}
	\bra{\chi_T}\ket{\chi_{S^{\otimes n}}} = \bra{\chi_A}\ket{\chi_{S^{\otimes n}}} = \frac{1}{6}\left(2^n + (-1)^n\right), \\
	\bra{\chi_S}\ket{\chi_{S^{\otimes n}}} = \frac{1}{6}\left(2^{n+1} + (-1)^{n+1}\right), 
\end{align}
so $\chi_{S^{\otimes n}} = a_n \chi_T \oplus a_n \chi_A + a_{n+1} \chi_S$ and thus $S^{\otimes n} = a_n T \oplus a_n A \oplus a_{n+1} S$ where $a_n = \left(2^n + (-1)^n\right)/6$.

\paragraph{Decomposition of $\Sym^2S$ and $\bigwedge^2S$.} The character of $\Sym^2S$ is $\chi_{\Sym^2S} = \frac{1}{2}\left[ (4,0,1) - (2,2,-1) \right] = (1,-1,1) = \chi_A$, so $\Sym^2S \cong A$. The character of $\bigwedge^2S$ is $\chi_{\bigwedge^2S} = (3,1,0) = \chi_P = \chi_T + \chi_S$, so $\bigwedge^2S \cong P \cong T \oplus S$, confirming $S \otimes S = \Sym^2S \oplus \bigwedge^2S$ since $\chi_{\Sym^2S} + \chi_{\bigwedge^2S} = A \oplus T \oplus S = \chi_{S \otimes S}$. Also, $R = S \otimes S \oplus S$.

\paragraph{} The contributions from the last few paragraphs are to the now complete character table of $S_3$ is presented in table \ref{table:completecharS3}.

\begin{table}[hbt!]
	\centering
	
	\begin{tabular}{c | c c c | l}
		       $S_3$         & $[1]$   & $[12]$  & $[123]$ &                                          \\
		  $\Ss|[\sigma]|$    & $\Ss 1$ & $\Ss 3$ & $\Ss 2$ &  \textit{Alternate compositions}         \\ \hline
		      $\chi_T$       & 1       & 1       & 1       &                                          \\
		      $\chi_A$       & 1       & -1      & 1       & $ A \cong \Sym^2S$                       \\
		      $\chi_S$       & 2       & 0       & -1      &                                          \\ \hline\hline
		      $\chi_P$       & 3       & 1       & 0       & $ P \cong T \oplus S \cong \bigwedge^2S$ \\
		$\chi_{S \otimes S}$ & 4       & 0       & 1       & $ S \otimes S \cong T \oplus A \oplus S$ \\
		      $\chi_R$       & 6       & 0       & 0       & $ R \cong T \oplus A \oplus 2S$
	\end{tabular}
	\label{table:completecharS3}
	\caption{Complete character table of $S_3$. The representations above the dashed line are irreducibles, and those below are composed. Some compositions are presented in the right-most column.}
\end{table}

