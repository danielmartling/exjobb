\subsection{Character table of the Klein 4-group $V$}

The Klein 4-group, denoted by\footnote{$V$ is for vierer, four in german.} $V$ is, along with $Z_4$, the only group of order 4. It can be described as the set
\begin{align}
	V = \{ e, x, y, xy \},
\end{align}
where every element is its own inverse and the product of any two distinct non-identity elements is the third one.
It is an abelian group, which means that every element has its own conjugacy class (SOURCE).

Since every group has a trivial representation $T$ in any vector space, so do $V$. The center of $V$ is trivial, so the regular representation $R$ has the character $(4,0,0,0)$ and is clearly not irreducible.

If $V$ instead is considered as a subgroup of $S_4$, that is
\begin{align}
	V = \{ (1), (12)(34), (13)(24), (14)(23) \} < S_4,
\end{align}
then each entry in the character of the permutation representation $P$ is the number of fixed points of the associated element. Then one sees that the character of the permutation representation is also $(4,0,0,0)$, it isomorphic to the regular representation.

From theorem \ref{thm:numberirrep} we expect three additional irreducible representations. Since the square sum over all representations of the character of $e$ must be equal to the order of the group (SOURCE), they must be of degree 1. We know that they must be parallel to themselves and orthogonal to $T$ and have a 1 as the character of $e$, so the three additional are alterations of $T$ with two of the ones exchanged with negative ones. Then $R$ is the direct sum of these four irreducibles. These findings are presented in table \ref{table:charV} as the representations $A, B$ and $C$.


\begin{table}[hbt!]
	\centering
	\begin{tabular}{c | c c c c l}
		$V$ & $e$ & $x$ & $y$ & $xy$ &             \\ \hline
		$T$ & 1   & 1   & 1   & 1   &             \\
		$A$ & 1   & -1  & -1  & 1   &             \\
		$B$ & 1   & -1  & 1   & -1  &             \\
		$C$ & 1   & 1   & -1  & -1  &             \\ \hline\hline
		$R$ & 4   & 0   & 0   & 0   & %$R \cong P \cong T \oplus A \oplus B \oplus C$
	\end{tabular}
	\caption{Character table for $V$.}
	\label{table:charV}
\end{table}