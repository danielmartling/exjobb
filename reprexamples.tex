\subsection{Examples}\label{sect:basicreprs}

This section follows \cite[1.?]{Serre}.

\subsubsection{Unital representations}

	A representation $\rho$ where every $g$ is such that $|\rho_g| = 1$, is called a unital representation.

\begin{example}[Trivial representation]\marginnote{Is it a repr. in ANY v.s.? For example if dimV=3, is triv.repr. mapping to 3 by 3 id matrix? Later decomposable to direct sum 3T.}
	For any group $G$ there is a trivial degree 1 representation defined by mapping each element of $G$ to 1, ie.
	\begin{align*}
		\rho_g = 1,
	\end{align*}
	for every $g \in G$. It is clearly a homomorphism since for any $g,h \in G$ we have that
	\begin{align*}
		\rho_{gh} = 1 = 1 \cdot 1 = \rho_g \rho_h.
	\end{align*}
	This could be extended to any vector space by mapping $g$ to the identity transformation of that vector space.
\end{example}

\begin{note}
	The trivial representation (the mapping is trivial) is not to be confused with the trivial space (vector space containing only the zero vector).
\end{note}

\begin{example}[Alternating representation of $\Sym_n$]
	Choosing $G = \Sym_n$, another degree 1 representation can be found by studying the signs, or parities, of the elements of $\Sym_n$. By Proposition~\ref{prop:signwelldefined}, the sign of a permutation is well-defined, so for any two permutations $\sigma$ and $\tau$ with respective signs $(-1)^s$ and $(-1)^t$, their composition has the sign
	\begin{align*}
		\sgn(\sigma\tau) = (-1)^{st} = (-1)^s \cdot (-1)^t = \sgn\sigma \cdot \sgn \tau,
	\end{align*}
	so clearly the map $\sgn: \Sym_n \rightarrow \{\pm 1\}$ is a homomorphism and thus a representation of degree 1, where even permutations are mapped to $+1$ and odd to $-1$.
\end{example}

%\subsubsection{Degree 1 representations of $\Cyc_n$}\label{sect:deg1cycn}

Choose $G$ to be the cyclic group of degree $n$, and let $g$ be a generator of $\Cyc_n$ such that 
\begin{align*}
	\Cyc_n = \{ e, g, g^2, \dots, g^{n-1}\}
\end{align*}
and $g^n = e$. Consider a map $\rho: \Cyc_n \rightarrow \CC$ defined as a homomorphism by $\rho_{g^a}\rho_{g^b} = \rho_{g^{a+b}}$, then by Equation~\ref{eq:rhoIdAndInv} we have that $\rho_e = 1$, but $g^n = e$ and by induction on Equation~\ref{eq:homomorphism} we have that $\rho_{g^n} = (\rho_g)^n$, so then we must have that $\rho_g$ is mapped to a $n$th root of unity.
In conclusion, for $\Cyc_n$ we have found $n$ representations of degree 1, each mapping $g$ to a $n$th root of unity and the powers of $g$ to the corresponding powers of the root of unity.

\begin{example}[Degree 1 representations of $\Cyc_4$]
	
\end{example}
	
	For $n = 4$, the four fourth roots of unity are $1,i,-1$ and $-i$ and the four corresponding representations of $\Cyc_4$ are presented in Table~\ref{mynicetable}.\marginnote{Same example as Sagan? Bad?? More examples n = 3,5?}
	
	\begin{table}[hbt!]
		
		content...
	\end{table}

\begin{figure}[hbt!]
	\begin{tabular}{c | c c cl}
		$\Cyc_4$ & $e$ & $g$  & $g^2$ & $g^3$ \\ \hline
		$\rho_0$           & 1   & 1    & 1     & 1     \\
		$\rho_1$           & 1   & $i$  & $-1$  & $-i$  \\
		$\rho_2$           & 1   & $-1$ & $1$  & $-1$   \\
		$\rho_3$           & 1   & $-i$ & $-1$   & $i$
		
	\end{tabular}
	\centering
	\label{mynicetable}
	\caption{Four representations of $\Cyc_4$.}
\end{figure}

\subsubsection{Permutation representation}

	Given a group $G$, we chose a suitable set $X$ that $G$ in some sense ``naturally'' acts on by permutation. Let $V$ be a vector space spanned by a orthonormal basis $(\bas_x)_{x \in X}$, then we have a homomorphism
	\begin{align*}
		\rho:G \rightarrow \GL(V)
	\end{align*} 
	defined by its action on the basis vectors by, for any $g \in G$, 
	\begin{align*}
		\rho_g: \bas_x \mapsto \bas_{gx},
	\end{align*}
	that is $\rho$ inherited the group action of $G$ on $X$. It is a homomorphism since for any $g,h \in G$ we have
	\begin{align*}
		g(h \cdot \bas_x) = g \cdot \bas_{hx}  = \bas_{ghx} = (gh) \cdot \bas_x
	\end{align*}
	for any $x \in X$.\marginnote{We have imposed a $G$-module structure on $V$.}
	
	Letting $G = \Sym_n$, it would be appropriate to choose the set $X = \{1, 2, \dots, n\}$ and to let any $\sigma \in \Sym_n$ permute any $1 \leq i \leq n$ by $(\sigma, i) \mapsto \sigma(i)$. Choosing a basis $(\bas_i)_{i=1}^n$ to span a vector space $V$, we define a homomorphism
	\begin{align*}
		\rho: \Sym_n \rightarrow \GL(V)
	\end{align*}
	defined for any $\sigma \in \Sym_n$ as
	\begin{align*}
		\rho_\sigma: \bas_i \mapsto \bas_{\sigma(i)}
	\end{align*}
	for any $1 \leq i \leq n$. In this basis, the corresponding set of matrix representations are the permutation matrices
	\begin{align*}
		\rho_\sigma = (r_{ij})_{n \times n} \text{, where } r_{ij} = \delta_{j,\sigma(i)} = \begin{cases}
			1 \text{, if $j = \sigma(i)$,} \\
			0 \text{, otherwise,}
		\end{cases}
	\end{align*}
%	where $\delta_{k,l}$ is the Kronecker delta which vanishes if $k \neq l$ and is equal to one if $k=l$.
	in which the $i$th column has a 1 in the $\sigma(i)$th row, and the rest of the rows have a 0.
	
	Under this action, a vector
	\begin{align*}
		(a_1, a_2, \dots, a_n) = a_1 \bas_1 + a_2 \bas_2 + \dots + a_n \bas_n \in V
	\end{align*}
	is permuted to 
	\begin{align*}
		(a_{\sigma^{-1}(1)}, a_{\sigma^{-1}(2)}, \dots, a_{\sigma^{-1}(n)}) = a_1 \bas_{\sigma(1)} + a_2 \bas_{\sigma(2)} + \dots + a_n \bas_{\sigma(n)}  \in V.
	\end{align*}
	
	\marginnote{Yet another interpretation is that $\sigma$ permutes the $n$ coordinate axes.}
	
	The dimension of $V$ is $|X|$, for example the permutation representation of $\Sym_n$ is of degree $n$.
	
	\begin{example}[Permutation representation of $S_3$]
		content...
	\end{example}
	
	\begin{example}[Permutation representation of $S_4$]
		content...
	\end{example}

\subsubsection{Regular representation}

	Following the same reasoning as in the previous section, but instead, let $G$ act on itself. The corresponding vector space $V$ is spanned by the basis $(\bas_g)_{g \in G}$ constructed from the elements of $G$. A representation of $G$ in $V$ is then a map
	\begin{align*}
		\rho: G \rightarrow \GL(V)
	\end{align*}
	defined by
	\begin{align*}
		g \cdot \bas_h = \bas_{gh}
	\end{align*}
	for any $g,h \in G$.\marginnote{We have created a Group Algebra $\CC[G]$.}
	
	The dimension of $V$ is $|G|$, for example the regular representation of $\Cyc_n$ is of degree $n$ and for $\Sym_n$ it is of degree $n!$, a number which grows increasingly quick for larger $n$, however the regular representation will be shown to be important later.
	
	\begin{example}[Regular representation of $\Sym_3$]
		content...
	\end{example}
	
	

%\subsubsection{Coset representations}
%
%Let $H$ be a subgroup of $G$. Let $\mathcal{H} = \{g_1H, g_2H, \dots, g_kH\}$ be the set of left cosets of $H$ in $G$ and define the action of $G$ on $\mathcal{H}$ by $g(g_iH) = (gg_i)H$ for any $g\in G$ and $g_iH \in \mathcal{H}$. The $g_i$ are called transversals of their respective $g_iH$. Construct a vector space $V$ by spanning it with the transversals as a basis, denoted $(\ghat_i)_{i=1}^k$ and define the group action of $G$ on $V$ by replicating the group action of $G$ on $\mathcal{H}$.\cite[Example 1.3.5.]{Sagan}
%
%\begin{example}[Trivial representation again]
%	Choosing $H = G$ (the non-proper subgroup), the coset representation reduces to the trivial representation.
%\end{example}
%
%\begin{example}[Regular representation again]
%	Choosing $H = \{e\}$ (the trivial subgroup), then $\mathcal{H} = G$ and we construct a vector space $V$ using the elements of $G$ as a basis, hence we have found the regular representation.
%\end{example}
%
%\begin{example}[Choosing a smarter subgroup of $\Sym_3$]
%	As laid out in~\cite[Example 1.3.5.]{Sagan}, choosing the subgroup of $\Sym_3$ to be $H = \{(1), (2,3)\}$, then the cosets of $H$ are $\mathcal{H} = \{ H, (1,2)H, (1,3)H \}$, and computing the matrices will yield the permutation representation of $\Sym_3$.
%\end{example}