\subsection{Examples}\label{sect:basicreprs}

This section follows \cite[1.2]{Serre}.

\subsubsection{Trivial representations}

%	A representation $\rho$ where every $g$ is such that $|\rho_g| = 1$, is called a unital representation.

\begin{example}[Trivial representation]\label{example:trivrepr}
	For any group $G$ there is a trivial representation of degree 1 defined by mapping each element of $G$ to 1, ie.
	\begin{align*}
		\rho_g^\Triv = 1,
	\end{align*}
	for every $g \in G$. It is clearly a homomorphism since for any $g,h \in G$ we have that
	\begin{align*}
		\rho_{gh}^\Triv = 1 = 1 \cdot 1 = \rho_g^\Triv \cdot \rho_h^\Triv.
	\end{align*}
	This could be extended to any vector space by mapping $g$ to the identity transformation of that vector space. For a vector space of dimension $n$, the representation taking every $g$ in a group to the $n \times n$ identity matrix can be described, in the language of Section~\ref{sect:tensorrepr}, as the direct sum of $n$ copies of the trivial representation.
\end{example}

\begin{note}
	The trivial representation (the mapping is trivial) is not to be confused with the trivial zero space (the vector space containing only the zero vector).
\end{note}

\begin{example}[Alternating representation of $\Sym_n$]\label{example:altrepr}
	Choosing $G = \Sym_n$, another degree 1 representation can be found by studying the signs, or parities, of the elements of $\Sym_n$. By \cite[Thm.12.6.1.]{Biggs}, the sign of a permutation is well-defined, so for any two permutations $\sigma$ and $\tau$ with respective signs $(-1)^s$ and $(-1)^t$, their composition has the sign
	\begin{align*}
		\sgn(\sigma\tau) = (-1)^{s+t} = (-1)^s \cdot (-1)^t = \sgn (\sigma) \cdot \sgn (\tau),
	\end{align*}
	so clearly the map $\sgn: \Sym_n \rightarrow \{\pm 1\}$ is a homomorphism and thus a representation of degree 1, where even permutations are mapped to $+1$ and odd to $-1$.
\end{example}

\subsubsection{Degree 1 representations of $\Cyc_n$}\label{sect:deg1cycn}

Choose $G = \Cyc_n$, and let $g$ be a generator of $\Cyc_n$ such that 
\begin{align*}
	\Cyc_n = \{ e, g, g^2, \dots, g^{n-1}\}
\end{align*}
and $g^n = e$. Consider a map $\rho: \Cyc_n \rightarrow \CC$ defined as a homomorphism by $\rho_{g^a}\rho_{g^b} = \rho_{g^{a+b}}$ for some integers $a,b$, then by Equation~\ref{eq:rhoIdAndInv} we have that $\rho_e = 1$, but $g^n = e$ and by induction on Equation~\ref{eq:homomorphism} we have that $\rho_{g^n} = (\rho_g)^n$, so then we must have that $(\rho_g)^n = 1$, that is $\rho_g$ is mapped to a $n$th root of unity.
In conclusion, for $\Cyc_n$ we have found $n$ representations of degree 1, denoted $\rho^0, \rho^1, \dots, \rho^{n-1}$, each mapping $g$ to a $n$th root of unity and the powers of $g$ to the corresponding powers of that root of unity. 

\begin{example}[$\Cyc_3$]
	The three third roots of unity are $1$, $\omega = \frac{-1+i\sqrt{3}}{2}$ and $w^2 = \frac{-1-i\sqrt{3}}{2}$, so the three representations of $\Cyc_3$ are presented in table \ref{table:Cyc3}. %Note that $\rho_1^* = \rho_2$.
	
%	\begin{table}[hbt!]
%		\begin{tabular}{c | c c c}
%			$\Cyc_3$ & $e$ & $g$        & $g^2$      \\ \hline
%			$\rho_0$          & 1   & 1          & 1          \\
%			$\rho_1$          & 1   & $\omega$   & $\omega^2$ \\
%			$\rho_2$          & 1   & $\omega^2$ & $\omega$
%		\end{tabular}
%		\centering
%		\caption{Three representations of $\Cyc_3$. $\omega = e^{2 \pi i/3}$}
%		\label{table:Cyc3}
%	\end{table}
\end{example}

\begin{example}[$\Cyc_4$]
	The four fourth roots of unity are $1,i,-1$ and $-i$ and the four corresponding representations of $\Cyc_4$ are presented in Table~\ref{tbl:cyc4}. %Note that $\rho_1^* = \rho_3$.
%	\begin{table}[hbt!]
%		\centering
%		\begin{tabular}{c | c c cl}
%			$\Cyc_4$ & $e$ & $g$  & $g^2$ & $g^3$ \\ \hline
%			$\rho_0$           & 1   & 1    & 1     & 1     \\
%			$\rho_1$           & 1   & $i$  & $-1$  & $-i$  \\
%			$\rho_2$           & 1   & $-1$ & $1$  & $-1$   \\
%			$\rho_3$           & 1   & $-i$ & $-1$   & $i$
%		\end{tabular}
%		\caption{Four representations of $\Cyc_4$.}
%		\label{tbl:cyc4}
%	\end{table}
\end{example}
	
\begin{example}[$\Cyc_5$]
	The five fifth roots of unity are $\exp{\frac{2\pi im}{5}}$, $0 \leq m \leq 4$, so five representations of $\Cyc_5$ are presented in table \ref{table:Cyc5}. %Note that $\rho_1^* = \rho_4$ and $\rho_2^* = \rho_3$.
	
%	\begin{table}[hbt!]
%		\begin{tabular}{c | c c c c c}
%			
%			$\Cyc_5$ & $e$ & $g$        & $g^2$      & $g^3$      & $g^4$      \\ \hline
%			$\rho_0$            & 1   & 1          & 1          & 1          & 1          \\
%			$\rho_1$            & 1   & $\omega^1$ & $\omega^2$ & $\omega^3$ & $\omega^4$ \\
%			$\rho_2$            & 1   & $\omega^2$ & $\omega^4$ & $\omega^1$ & $\omega^3$ \\
%			$\rho_3$            & 1   & $\omega^3$ & $\omega^1$ & $\omega^4$ & $\omega^2$ \\
%			$\rho_4$            & 1   & $\omega^4$ & $\omega^3$ & $\omega^2$ & $\omega^1$
%		\end{tabular}
%		\centering
%		\caption{Five representations of $\Cyc_5$. $\omega = e^{2 \pi i/5}$.}
%		\label{table:Cyc5}
%	\end{table}
\end{example}



\begin{table}[hbt!]
	\parbox[s]{.31\linewidth}{
		\centering
		\begin{tabular}{c | c c c}
			$\Cyc_3$ & $e$ & $g$        & $g^2$      \\ \hline
			$\rho_0$          & 1   & 1          & 1          \\
			$\rho_1$          & 1   & $\omega$   & $\omega^2$ \\
			$\rho_2$          & 1   & $\omega^2$ & $\omega$
		\end{tabular}
		\vfill
		\caption{Three reprs. of $\Cyc_3$. $\omega = e^{2 \pi i/3}$.}
		\label{table:Cyc3}
	}
	\hfill
	\parbox[s]{.31\linewidth}{
		\centering
		\begin{tabular}{c | c c cl}
			$\Cyc_4$ & $e$ & $g$  & $g^2$ & $g^3$ \\ \hline
			$\rho_0$           & 1   & 1    & 1     & 1     \\
			$\rho_1$           & 1   & $i$  & $-1$  & $-i$  \\
			$\rho_2$           & 1   & $-1$ & $1$  & $-1$   \\
			$\rho_3$           & 1   & $-i$ & $-1$   & $i$
		\end{tabular}
		\caption{Four reprs. of $\Cyc_4$.}
		\label{tbl:cyc4}
	}
	\hfill
	\parbox[s]{.33\linewidth}{
		\centering
		\begin{tabular}{c | c c c c c}
			$\Cyc_5$ & $e$ & $g$        & $g^2$      & $g^3$      & $g^4$      \\ \hline
			$\rho_0$            & 1   & 1          & 1          & 1          & 1          \\
			$\rho_1$            & 1   & $\omega$ & $\omega^2$ & $\omega^3$ & $\omega^4$ \\
			$\rho_2$            & 1   & $\omega^2$ & $\omega^4$ & $\omega$ & $\omega^3$ \\
			$\rho_3$            & 1   & $\omega^3$ & $\omega$ & $\omega^4$ & $\omega^2$ \\
			$\rho_4$            & 1   & $\omega^4$ & $\omega^3$ & $\omega^2$ & $\omega$
		\end{tabular}
		\caption{Five reprs. of $\Cyc_5$. $\omega = e^{2 \pi i/5}$.}
		\label{table:Cyc5}
	}
\end{table}



\subsubsection{Permutation representation}

	Given a group $G$, we chose a set $X$ that $G$ acts on by permutation. Let $V$ be a vector space spanned by a natural basis $(\bas_x)_{x \in X}$, then we have a representation $\rho^\Perm : G \rightarrow \GL(V)$	defined by its action on the basis vectors by, for any $g \in G$, 
	\begin{align*}
		\rho_g^\Perm : \bas_x \mapsto \bas_{gx},
	\end{align*}
	that is $\rho^\Perm$ inherited the group action of $G$ on $X$. It is a homomorphism since for any $g,h \in G$ we have
	{\allowdisplaybreaks\begin{align*}
		g \cdot (h \cdot \bas_x) &= g \cdot \bas_{hx} \\
		&= \bas_{ghx} \\
		&= (gh) \cdot \bas_x
	\end{align*}}
	for any $x \in X$. %\marginnote{We have imposed a $G$-module structure on $V$.}
	In the $(\bas_x)_{x \in X}$-basis, $\rho_g^\Perm$ are permutation matrices, which have one once in every row and column, and the rest of the entries are zero.
	
	Letting $G = \Sym_n$, it would be appropriate to choose the set $X = \{1, 2, \dots, n\}$ and to let any $\sigma \in \Sym_n$ permute any $1 \leq i \leq n$ by $i \mapsto \sigma(i)$. Choosing a basis $(\bas_i)_{i=1}^n$ to span a vector space $V$, we define a representation $\rho^\Perm: \Sym_n \rightarrow \GL(V)$ defined for any $\sigma \in \Sym_n$ as
	\begin{align*}
		\rho_\sigma^\Perm: \bas_i \mapsto \bas_{\sigma(i)}
	\end{align*}
	for any $1 \leq i \leq n$. In this basis, the corresponding set of matrix representations are the permutation matrices
	\begin{align*}
		\rho_\sigma^\Perm = (r_{ij})_{n \times n} \text{, where } r_{ij} = \delta_{j,\sigma(i)} = \begin{cases}
			1 \text{, if $j = \sigma(i)$,} \\
			0 \text{, otherwise,}
		\end{cases}
	\end{align*}
%	where $\delta_{k,l}$ is the Kronecker delta which vanishes if $k \neq l$ and is equal to one if $k=l$.
	in which the $i$th column has a 1 in the $\sigma(i)$th row, and the rest of the rows have a 0.
	
	Under this action, a vector
	\begin{align*}
		(a_1, a_2, \dots, a_n) = a_1 \bas_1 + a_2 \bas_2 + \dots + a_n \bas_n \in V
	\end{align*}
	is permuted to 
	\begin{align*}
		(a_{\sigma^{-1}(1)}, a_{\sigma^{-1}(2)}, \dots, a_{\sigma^{-1}(n)}) = a_1 \bas_{\sigma(1)} + a_2 \bas_{\sigma(2)} + \dots + a_n \bas_{\sigma(n)}  \in V.
	\end{align*}
	
%	\marginnote{Yet another interpretation is that $\sigma$ permutes the coordinate axes of a $n$-dimensional space.}
	
	The dimension of $V$ is $|X|$, for example the permutation representation of $\Sym_n$ is of degree $n$.
	
	\begin{example}[Permutation representation of $\Sym_2$]
		The symmetric group of degree 2 has two elements, $\Sym_2 = \{(1), (1,2)\}$ and their matrix representations in $\bas_1,\bas_2$-space are presented in Table~\ref{table:permS2}.
		\begin{table}[hbt!]
			\centering
			\begin{tabular}{c c}
				$\rho_{(1)}^\Perm =
				\begin{pmatrix}
					1 & 0 \\ 0 & 1
				\end{pmatrix}$, &
				$\rho_{(1,2)}^\Perm =
				\begin{pmatrix}
					0 & 1 \\ 1 & 0
				\end{pmatrix}$
			\end{tabular}
			\caption{Matrix representations of $\Sym_2$}
			\label{table:permS2}
		\end{table}
	\end{example}
	\begin{example}[Permutation representation of $\Sym_3$]\label{ex:permS3}
		Likewise, representations of $\Sym_3$ are presented in Table~\ref{table:permS3}.
		\begin{table}[hbt!]
			\centering
			\begin{tabular}{r r r}
				$\rho_{(1)}^\Perm = 
				\begin{pmatrix}
					1 & 0 & 0 \\
					0 & 1 & 0 \\
					0 & 0 & 1
				\end{pmatrix}$, & 
				$\rho_{(1,2,3)}^\Perm = 
				\begin{pmatrix}
					0 & 0 & 1 \\
					1 & 0 & 0 \\
					0 & 1 & 0
				\end{pmatrix}$, & 
				$\rho_{(1,3,2)}^\Perm = 
				\begin{pmatrix}
					0 & 1 & 0 \\
					0 & 0 & 1 \\
					1 & 0 & 0
				\end{pmatrix}$, \\ & & \\
				$\rho_{(1,2)}^\Perm = 
				\begin{pmatrix}
					0 & 1 & 0 \\
					1 & 0 & 0 \\
					0 & 0 & 1
				\end{pmatrix}$, &
				$\rho_{(1,3)}^\Perm = 
				\begin{pmatrix}
					0 & 0 & 1 \\
					0 & 1 & 0 \\
					1 & 0 & 0
				\end{pmatrix}$, &
				$\rho_{(2,3)}^\Perm = 
				\begin{pmatrix}
					1 & 0 & 0 \\
					0 & 0 & 1 \\
					0 & 1 & 0
				\end{pmatrix}$.
			\end{tabular}
			\caption{Matrix representations of $\Sym_3$}
			\label{table:permS3}
		\end{table}
	\end{example}
	
	\begin{example}[Permutation representation of $\Sym_4$]
		Yet again, matrix representations of some elements of $\Sym_4$ are presented in Table~\ref{table:permS4}. The elements chosen are one representative from every conjugacy class of $\Sym_4$.
		\begin{table}[hbt!]
			\centering
			\begin{tabular}{r r r}
				$\rho_{(1)}^\Perm = \left(\begin{matrix}
					1 & 0 & 0 & 0 \\
					0 & 1 & 0 & 0 \\
					0 & 0 & 1 & 0 \\
					0 & 0 & 0 & 1
				\end{matrix}\right)$,  &
				$\rho_{(1,2)}^\Perm = \left(\begin{matrix}
					0 & 1 & 0 & 0 \\
					1 & 0 & 0 & 0 \\
					0 & 0 & 1 & 0 \\
					0 & 0 & 0 & 1
				\end{matrix}\right)$, &
				$\rho_{(1,2)(3,4)}^\Perm = \left(\begin{matrix}
					0 & 1 & 0 & 0 \\
					1 & 0 & 0 & 0 \\
					0 & 0 & 0 & 1 \\
					0 & 0 & 1 & 0
				\end{matrix}\right)$, \\ & & \\ &
				$\rho_{(1,2,3)}^\Perm = \left(\begin{matrix}
					0 & 0 & 1 & 0 \\
					1 & 0 & 0 & 0 \\
					0 & 1 & 0 & 0 \\
					0 & 0 & 0 & 1
				\end{matrix}\right)$, &
				$\rho_{(1,2,3,4)}^\Perm = \left(\begin{matrix}
					0 & 0 & 0 & 1 \\
					1 & 0 & 0 & 0 \\
					0 & 1 & 0 & 0 \\
					0 & 0 & 1 & 0
				\end{matrix}\right)$.
			\end{tabular}
			\caption{Some matrix representations of $\Sym_4$}
			\label{table:permS4}
		\end{table}
	\end{example}

\subsubsection{Regular representation}

	Following the same reasoning as in the previous section, but instead we let $G$ act on itself, ie. $X = G$. The corresponding vector space $V$ is spanned by the basis $(\bas_g)_{g \in G}$ constructed from the elements of $G$. The regular representation of $G$ in $V$ is then a map	$\rho^\Reg: G \rightarrow \GL(V)$ defined by $g \cdot \bas_h = \bas_{gh}$ for any $g,h \in G$.%\marginnote{We have created a Group Algebra $\CC[G]$.}
	
	The dimension of $V$ is $|G|$, for example the regular representation of $\Cyc_n$ is of degree $n$ and for $\Sym_n$ it is of degree $n!$, a number which grows increasingly quick for larger $n$, however the regular representation will be shown to be key in finding every representation of a group.
	{\allowdisplaybreaks\begin{example}[Regular representation of $\Cyc_3$]
		\label{example:regCyc3}
		For $G = \Cyc_3$ we calculate the action of $\Cyc_3$ on the $\bas, \ghat, \ghat^2$-basis thusly:
		\begin{align*}
			\begin{cases}
				e \cdot \bas = \bas, \\
				e \cdot \ghat = \ghat, \\
				e \cdot \ghat^2 = \ghat^2,
			\end{cases} \quad \begin{cases}
				g \cdot \bas = \ghat, \\
				g \cdot \ghat = \ghat^2, \\
				g \cdot \ghat^2 = \bas,
			\end{cases} \quad \begin{cases}
			g^2 \cdot \bas = \ghat^2, \\
			g^2 \cdot \ghat = \bas, \\
			g^2 \cdot \ghat^2 = \ghat,
			\end{cases}
		\end{align*}
		hence the permutation matrices in the $\bas, \ghat, \ghat^2$-basis are
		\begin{align*}
			\rho_e^\Reg = \id, \quad \rho_g^\Reg = \begin{pmatrix}
				0&0&1 \\ 1&0&0 \\ 0&1&0
			\end{pmatrix}, \quad \rho_{g^2}^\Reg = \begin{pmatrix}
			0&1&0 \\ 0&0&1 \\ 1&0&0 
			\end{pmatrix}.
		\end{align*}
		Let's choose a new basis inspired by the third roots of unity 1, $\omega$ and $\omega^2$:
		\begin{align*}
			\begin{cases}
				\fbas_1 = \bas + \ghat + \ghat^2, \\
				\fbas_2 = \bas + \omega^2\ghat + \omega\ghat^2, \\
				\fbas_3 = \bas + \omega\ghat + \omega^2\ghat^2,
			\end{cases}
		\end{align*}
		where $\omega = e^{2 \pi i/3}$, corresponding to the change-of-basis matrix 
		\begin{align*}
			P = \begin{pmatrix}
				1&1&1 \\
				1&\omega^2&\omega \\
				1&\omega&\omega^2
			\end{pmatrix}, \quad \text{with inverse} \quad P^{-1} = \frac{1}{3}\begin{pmatrix}
				1&1&1 \\
				1&\omega&\omega^2 \\
				1&\omega^2&\omega
			\end{pmatrix}.
		\end{align*}
		Now, the matrices of $\Cyc_3$ in this new basis are
		\begin{align*}
			\rho_e^\Reg &= P^{-1} \cdot \id \cdot P = \id, \\ 
			\rho_g^\Reg &= P^{-1}\cdot\begin{pmatrix}
				0&0&1 \\ 1&0&0 \\ 0&1&0
			\end{pmatrix}\cdot P = \begin{pmatrix}
			1&0&0 \\ 0&\omega&0 \\ 0&0&\omega^2
			\end{pmatrix}, \\
			\rho_{g^2}^\Reg &= P^{-1}\cdot\begin{pmatrix}
				0&1&0 \\ 0&0&1 \\ 1&0&0 
			\end{pmatrix}\cdot P = \begin{pmatrix}
			1&0&0 \\ 0&\omega^2&0 \\ 0&0&\omega
			\end{pmatrix}.
		\end{align*}
		We can clearly see that the regular representation of $\Cyc_3$ is the direct sum of the three degree 1 representations found in Table~\ref{table:Cyc3}, eg. 
		\begin{align*}
			\rho_g^\Reg = (1) \oplus (\omega) \oplus (\omega^2) = \rho^0_g \oplus \rho^1_g \oplus \rho^2_g
		\end{align*}
		after the change of basis.
	\end{example}}
	
%	\begin{example}[Regular representation of $\Sym_3$]\label{example:regSym3}
%		Defining a representation $\rho^\Reg: S_3 \rightarrow \GL(V)$ on the vector space $V$ spanned by the basis\footnote{Explicitly, this is $\left\lbrace \bas_{(1)}, \bas_{(1,2)}, \bas_{(1,3)}, \bas_{(2,3)}, \bas_{(1,2,3)}, \bas_{(1,3,2)} \right\rbrace$.} $(\bas_{\sigma})_{\sigma \in \Sym_3}$ defined for every $\sigma$ by \begin{align*}
%		\rho_\sigma \cdot \bas_\tau = \bas_{\sigma \circ \tau} = \bas_{\sigma\tau}
%		\end{align*} for some $\tau \in \Sym_3$, we arrive at a degree 6 representation of $\Sym_3$. %It is clearly a larger degree representation of $S_n$ than any we have found so far.% Later we will find that for $\Sym_3$, $R$ is the direct sum of the trivial representation, the alternating representation and two copies of the standard representations, that is $S = T \oplus A \oplus 2S$.
%		
%		\textit{To be added: Table of $6 \times 6$-matrices, one per conjugacy class for example. Also: Show that perm.repr. of $\Sym_3$ is completely decomposed to trivial + alternating + two copies of standard, especially since the multiplicity of the standard representation is $>1$.}
%	\end{example}
	
	

%\subsubsection{Coset representations}
%
%Let $H$ be a subgroup of $G$. Let $\mathcal{H} = \{g_1H, g_2H, \dots, g_kH\}$ be the set of left cosets of $H$ in $G$ and define the action of $G$ on $\mathcal{H}$ by $g(g_iH) = (gg_i)H$ for any $g\in G$ and $g_iH \in \mathcal{H}$. The $g_i$ are called transversals of their respective $g_iH$. Construct a vector space $V$ by spanning it with the transversals as a basis, denoted $(\ghat_i)_{i=1}^k$ and define the group action of $G$ on $V$ by replicating the group action of $G$ on $\mathcal{H}$.\cite[Example 1.3.5.]{Sagan}
%
%\begin{example}[Trivial representation again]
%	Choosing $H = G$ (the non-proper subgroup), the coset representation reduces to the trivial representation.
%\end{example}
%
%\begin{example}[Regular representation again]
%	Choosing $H = \{e\}$ (the trivial subgroup), then $\mathcal{H} = G$ and we construct a vector space $V$ using the elements of $G$ as a basis, hence we have found the regular representation.
%\end{example}
%
%\begin{example}[Choosing a smarter subgroup of $\Sym_3$]
%	As laid out in~\cite[Example 1.3.5.]{Sagan}, choosing the subgroup of $\Sym_3$ to be $H = \{(1), (2,3)\}$, then the cosets of $H$ are $\mathcal{H} = \{ H, (1,2)H, (1,3)H \}$, and computing the matrices will yield the permutation representation of $\Sym_3$.
%\end{example}