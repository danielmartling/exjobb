{\allowdisplaybreaks\subsection{Examples}

%\textit{REWRITE THIS SECTION AROUND THE STUDY OF THE REGULAR REPRESENTATION.}

\begin{notation}
	For this section, a representation space may be denoted by its designation in the subscript, that is a trivial representation space is denoted 
	by $V_\Triv$, a permutation by $V_\Perm$, etc.
\end{notation}

\subsubsection{Characters of $\Cyc_n$}

\begin{example}[Characters of $\Cyc_n$]
	By Equation~\ref{eq:regcharacter}, the character of the regular representation of $\Cyc_n$ is $\chi_\Reg = (n, 0, \dots)$. In section \ref{sect:deg1cycn} we described $n$ irreducible representations of degree 1 of  $\Cyc_n$, and by Theorem~\ref{thm:regreprmult}, we have then found every irreducible representation of $\Cyc_n$ since $\sum_{i=1}^n 1^2 = n$.
	
	Remember that the trace of a $1 \times 1$ matrix is its only element, therefore the Tables~\ref{table:Cyc3}, \ref{tbl:cyc4} and \ref{table:Cyc5} are the character tables of respectively $\Cyc_3$, $\Cyc_4$ and $\Cyc_5$.
\end{example}

\subsubsection{Characters of $\Sym_n$}
	
\begin{example}[Characters of $\Sym_3$]
	Returning to Table~\ref{table:chartableSym3}, we have found three irreducible representations of $\Sym_3$. Their degrees are 1, 1 and 2, and the square sum of the degrees are $1 + 1 + 4 = 6$ which is the order of $\Sym_3$, hence by Corollary~\ref{cor:sqsumirreds}, we have found every irreducible representation of $\Sym_3$. 
	
	Now we decompose the tensor powers of the standard representation.
	
	\paragraph{Decomposition of $V_\Stan \otimes V_\Stan$.} Its character is $(4,0,1)$, which by a quick glance on the character table is seen to be the sum of all irreducibles, hence \begin{align*}
		V_\Stan \otimes V_\Stan = V_\Triv \oplus V_\Alt \oplus V_\Stan.
	\end{align*}
	
%	\paragraph{Decomposition of $\SymP^2S$ and $\AltP^2S$.} The character of $\SymP^2S$ is 
%	\begin{align*}
%		\frac{1}{2}\left[ (4,0,1) - (2,2,-1) \right] = (1,-1,1) = \chi_\Alt,
%	\end{align*} hence the symmetric square of the standard representation is isomorphic to the alternating representation. 
%	
%	The character of $\AltP^2S$ is likewise 
%	\begin{align*}
%		\frac{1}{2}\left[ (4,0,1) - (2,2,-1) \right] = (3,1,0) = \chi_\Perm, 
%	\end{align*}
%	hence the exterior square of the standard representation is isomorphic to the permutation representation, compatible with $S \otimes S = \SymP^2S \oplus \AltP^2S$.
	
	\paragraph{Decomposition of $V_\Stan^{\otimes n}$.}\cite[Exercise 2.7.]{FultonHarris} To find the decomposition of larger tensor powers, we study the character $\chi_\Stan^n = (2^n, 0, (-1)^n)$ and take the inner product of it with the irreducibles and find:
	\begin{align*}
		({\chi_\Triv}|\chi_\Stan^n) = ({\chi_\Alt}|\chi_\Stan^n) &= \frac{1}{6}\left(2^n + (-1)^n\right), \quad \text{and}\\
		({\chi_\Stan}|\chi_\Stan^n) &= \frac{1}{6}\left(2^{n+1} + (-1)^{n+1}\right), 
	\end{align*}
	hence \begin{align*}
		V_\Stan^{\otimes n} = a_n (V_\Triv \oplus V_\Alt) \oplus a_{n+1} V_\Stan, \quad \text{where} \quad a_n = \frac{1}{6}\left(2^n + (-1)^n\right).
	\end{align*} 
\end{example}

	\paragraph{} We arrive at a ``complete'' character table for $\Sym_3$, presented in Table~\ref{table:completecharS3}.
	
	\begin{table}[hbt!]
		\centering
		
		\begin{tabular}{c | c c c}
			$S_3$         & $[(1)]$   & $[(1,2)]$  & $[(1,2,3)]$ \\
			$\Ss|[\sigma]|$    & $\Ss 1$ & $\Ss 3$ & $\Ss 2$ \\ \hline
			$\chi_\Triv$       & 1       & 1       & 1       \\
			$\chi_\Alt$       & 1       & -1      & 1       \\
			$\chi_\Stan$       & 2       & 0       & -1      \\ \hline\hline
			$\chi_\Perm$       & 3       & 1       & 0       \\
			$\chi_\Stan^2$ & 4       & 0       & 1       \\
			$\chi_\Reg$       & 6       & 0       & 0       		\end{tabular}
			
		\caption{Complete character table of $S_3$. The representations above the doublestruck line are irreducibles, and those below are composed.}
		\label{table:completecharS3}
	\end{table}
%\end{example}

\begin{example}[Characters of $\Sym_4$]
	This section follows the same methods from the previous section on $\Sym_3$. So far, we have that the trivial, the alternating and the standard representations are irreducibles, see Table~\ref{table:chartableSym4}. Their respective degrees are 1, 1 and 3, with the square sum $1+1+9 = 11$, which is less than $|\Sym_4|=24$, hence by Corollary~\ref{cor:sqsumirreds} we are expected to find additional irreducible representations of $\Sym_4$. We will find them while constructing and decomposing new representations.
	
	\paragraph{Decomposition of $V_\Alt \otimes V_\Stan$.} Consider an ``alternating version of the standard representation'', denoted with the space $V'_\Stan := V_\Alt \otimes V_\Stan$ and the character $\chi'_\Stan := \chi_\Alt\chi_\Stan = (3,-1,0,1,-1)$. It clearly has the same inner product with itself as $\chi_\Stan$, hence it is also irreducible. It increments the square sum of degrees to $11 + 3^2 = 20$.
	
	\paragraph{Decomposition of $V_\Stan \otimes V_\Stan$.} To find the next irreducible, we study the tensor square of the standard representation. Its character is $\chi_\Stan^2 = (9,1,0,1,1)$ and calculations will show that
	\begin{align*}
		(\chi_\Triv | \chi_\Stan^2) & = \frac{1}{24}(9 + 6 + 0 + 6 + 3) = 1, \\
		(\chi_\Alt | \chi_\Stan^2)  &= \frac{1}{24}(9-6+0-6+2) = 0, \\
		(\chi_\Stan | \chi_\Stan^2)  &= \frac{1}{24}(27+6+0-6-3) = 1, \quad \text{and} \\
		(\chi_\Alt\chi_\Stan | \chi_\Stan^2)  &= \frac{1}{24}(27-6+0+6-3) = 1.
	\end{align*}
	However these subrepresentations of non-zero multiplicity are of degrees 1, 3, and 3, and then their direct sum is of degree 7, which is less than the dimension of $V_\Stan \otimes V_\Stan$, hence by Proposition~\ref{prop:complsubrepr} there is another (not necessarily irreducible) subrepresentation of degree 2 complementing them. Denoting this representation by $W$, its character is $\chi_W = \chi_\Stan^2 - \chi_\Triv - \chi_\Stan - \chi'_\Stan = (2,0,-1,0,2)$, which is found to be such that 
	\begin{align*}
		(\chi_W|\chi_W) = \frac{1}{24}(4 + 0 + 8 + 0 + 12) = 1,
	\end{align*}
	hence $W$ is irreducible and the tensor square of the standard representation is decomposed to 
	\begin{align*}
		V_\Stan \otimes V_\Stan = V_\Triv \oplus W \oplus V_\Stan \oplus V'_\Stan.
	\end{align*}
	We have now found all five irreducible representations of $\Sym_4$, since $1^2 + 1^2 +  2^2 + 3^2 + 3^2 = 24$, the order of $\Sym_4$.
%	\paragraph{Decomposition of $\SymP^2S$ and $\AltP^2S$.} The character of $\SymP^2S$ is 
%	\begin{align*}
%		\frac{1}{2}\left[ (9,1,0,1,1)-(3,3,0,-1,3) \right] = (3,-1,0,1,-1)
%	\end{align*}
%	which is the character of $S'$. Likewise, the character of $\AltP^2S$ is $(6,2,0,0,2)$ which is found to be the sum of the characters of $T, V$ and $S$, hence $\AltP^2S = T \oplus V \oplus S$.
	\paragraph{Decomposition of the $n$th tensor power of $W$.}	The character of $W^{\otimes n}$ is
	\begin{align*}
		\chi_{W}^n = (2^n, 0, (-1)^n, 0, 2^n).
	\end{align*}
	
	Its inner product with the irreducible characters are found to be
	\begin{align*}
		(\chi_\Triv| \chi_{W}^n) = (\chi_\Alt | \chi_{W}^n) &= \frac{1}{6} \left(2^n + 2 \cdot (-1)^n\right), \\
		(\chi_W | \chi_{W}^n) &= \frac{1}{6}\left(2^{n+1} + 2 \cdot (-1)^{n+1}\right), \text{ and}\\
		(\chi_\Stan | \chi_{W}^n) = (\chi'_\Stan | \chi_{W}^n) &= 0,
	\end{align*}
	hence the decomposition is
	\begin{align*}
		W^{\otimes n} = a_n (T \oplus A) \oplus a_{n+1} W, \quad \text{where} \quad a_n := \frac{1}{6}\left( 2^n + 2 \cdot (-1)^n \right).
	\end{align*}
	
	\paragraph{Decomposition of the $n$th tensor power of $V_\Stan$.} The character of $V_\Stan^{\otimes n}$ is 
	\begin{align*}
		\chi_\Stan^n = (3^n, 1, 0, (-1)^n, (-1)^n),
	\end{align*}
	and after projecting it on the irreducible characters we find that
	\begin{align}\label{eq:nthpowerS}
		V_\Stan^{\otimes n} = a_n V_\Triv \oplus b_n V_\Alt \oplus c_n W \oplus a_{n+1} V_\Stan \oplus b_{n+1} V'_\Stan,
	\end{align}
	where 
	\begin{align*}
		\begin{cases}
			a_n = \frac{1}{24}(3^n + 9\cdot(-1)^n+6), \\
			b_n = \frac{1}{24}(3^n - 3\cdot(-1)^n -6), \text{ and} \\
			c_n = \frac{1}{24}(2\cdot 3^n + 6\cdot(-1)^n).
		\end{cases}
	\end{align*}
	
	\paragraph{Decomposition of the $n$th tensor powers of $V'_\Stan$.} Since $V'_\Stan = V_\Alt \otimes V_\Stan$, the character of $(V'_\Stan)^{\otimes n}$ is the $n$th power of the character $\chi_\Alt \chi_\Stan$, which is $\chi_\Alt^n\chi_\Stan^n$, hence% is the tensor product of $n$ factors of $A \otimes S$. Since the tensor product is a commutative operation, we can rearrange it as
	\begin{align*}
		(V'_\Stan)^{\otimes n} = V_\Stan^{\otimes n} \otimes V_\Alt^{\otimes n}.
	\end{align*}
	By Example~\ref{example:nthpoweralternating}, the $n$th tensor power of $V_\Alt$ is $V_\Triv$ if $n$ is even and $V_\Alt$ if $n$ is odd, so for even $n$, $(V'_\Stan)^{\otimes n}$ is isomorphic to $V_\Stan^{\otimes n}$, and for odd $n$, it is isomorphic to $ V_\Stan^{\otimes n} \otimes V_\Alt$. Tensor multiplying Equation~\ref{eq:nthpowerS} with $V_\Alt$, we have:
	\begin{align*}
		V_\Stan^{\otimes n} \otimes V_\Alt &= \Big(a_n V_\Triv \oplus b_n V_\Alt \oplus c_n W \oplus a_{n+1} V_\Stan \oplus b_{n+1} V'_\Stan\Big)\otimes V_\Alt \\
		&= b_n V_\Triv \oplus a_n V_\Alt \oplus c_n W \oplus b_{n+1} V_\Stan \oplus a_{n+1} V'_\Stan,
	\end{align*}
	since 
	\begin{align*}
		V_\Triv \otimes V_\Alt &= V_\Alt, \\
		V_\Alt \otimes V_\Alt &= V_\Triv, \\
		V_\Stan \otimes V_\Alt &= V'_\Stan, \quad \text{and} \\
		V'_\Stan \otimes V_\Alt &= V_\Stan \otimes V_\Alt^{\otimes 2} = V_\Stan.
	\end{align*}
	Then we have found the decomposition of all tensor powers of $V'_\Stan$. For even $n$, it is the same as $V_\Stan^{\otimes n}$, and lastly for odd $n$, $V_\Triv$ switch multiplicities with $V_\Alt$, and $V_\Stan$ switch with $V'_\Stan$.
	
	\paragraph{} The findings of the last few paragraphs (except those on larger tensor powers) are presented in table \ref{table:completecharS4}.
	 \vspace{2cm}
%	\vfill
	
	\begin{table}[hbt!]
		\centering
		\begin{tabular}{c | c c c c c}
			$\Sym_4$         & $[(1)]$   & $[(1,2)]$  & $[(1,2,3)]$ & $[(1,2,3,4)]$ & $[(1,2)(3,4)]$ \\
			$\Ss|[\sigma]|$    & $\Ss 1$ & $\Ss 6$ & $\Ss 8$ & $\Ss 6$  & $\Ss 3$      \\ \hline
			$\chi_\Triv$       & 1       & 1       & 1       & 1        & 1            \\
			$\chi_\Alt$       & 1       & -1      & 1       & -1       & 1            \\
			$\chi_W$       & 2       & 0       & -1      & 0        & 2            \\
			$\chi_\Stan$       & 3       & 1       & 0       & -1       & -1           \\
			$\chi'_\Stan$ & 3       & -1      & 0       & 1        & -1           \\ \hline\hline
			$\chi_\Perm$       & 4       & 2       & 1       & 0        & 0            \\
			%$\chi_{\bigwedge^2S}$    & 6       & 2       & 0       & 0        & 2            & $\bigwedge^2S \cong T \oplus S \oplus V$                     \\
			$\chi_\Stan^2$ & 9       & 1       & 0       & 1        & 1            \\
%			$\chi_R$       & 24      & 0       & 0       & 0        & 0            & $R \cong T \oplus A \oplus 2V \oplus 3S \oplus 3S'$
		\end{tabular}
		\caption{Complete character table of $\Sym_4$. The representations above the dashed line are irreducibles, and those below are composed.}
		\label{table:completecharS4}
	\end{table}
\end{example}

%\textit{LOW PRIO: ADD CHARACTER THEORY ON KLEIN'S 4-GROUP AND DIHEDRAL GROUPS}

%\subsection{Characters of the Klein 4-group $V$}
%
%The Klein 4-group, denoted by\footnote{$V$ is for vierer, four in german.} $V$ is, along with $Z_4$, the only group of order 4. It can be described as the set
%\begin{align}
%	V = \{ e, x, y, xy \},
%\end{align}
%where every element is its own inverse and the product of any two distinct non-identity elements is the third one.
%It is an abelian group, which means that every element has its own conjugacy class (SOURCE).
%
%Since every group has a trivial representation $T$ in any vector space, so do $V$. The center of $V$ is trivial, so the regular representation $R$ has the character $(4,0,0,0)$ and is clearly not irreducible.
%
%If $V$ instead is considered as a subgroup of $S_4$, that is
%\begin{align}
%	V = \{ (1), (12)(34), (13)(24), (14)(23) \} < S_4,
%\end{align}
%then each entry in the character of the permutation representation $P$ is the number of fixed points of the associated element. Then one sees that the character of the permutation representation is also $(4,0,0,0)$, it isomorphic to the regular representation.
%
%From theorem \ref{thm:numberirrep} we expect three additional irreducible representations. Since the square sum over all representations of the character of $e$ must be equal to the order of the group (SOURCE), they must be of degree 1. We know that they must be parallel to themselves and orthogonal to $T$ and have a 1 as the character of $e$, so the three additional are alterations of $T$ with two of the ones exchanged with negative ones. Then $R$ is the direct sum of these four irreducibles. These findings are presented in table \ref{table:charV} as the representations $A, B$ and $C$.
%
%
%\begin{table}[hbt!]
%	\centering
%	\begin{tabular}{c | c c c c l}
%		$V$ & $e$ & $x$ & $y$ & $xy$ &             \\ \hline
%		$T$ & 1   & 1   & 1   & 1   &             \\
%		$A$ & 1   & -1  & -1  & 1   &             \\
%		$B$ & 1   & -1  & 1   & -1  &             \\
%		$C$ & 1   & 1   & -1  & -1  &             \\ \hline\hline
%		$R$ & 4   & 0   & 0   & 0   & %$R \cong P \cong T \oplus A \oplus B \oplus C$
%	\end{tabular}
%	\caption{Character table for $V$.}
%	\label{table:charV}
%\end{table}
%
%
%\subsection{Characters of the dihedral group of order 8, $\Dih_8$}
%
%$\Dih_8$ is a 2-Sylow subgroup of $S_4$. Restrict the character table of $S_4$ to $\Dih_8$. 
%
%$\Dih_8$ has its non-trivial                                                                                                                                                                                                                                                              center $Z(\Dih_8) = \{e, r^2\}$ as a normal subgroup. The quotient $\Dih_8/\{e, r^2\}$ is isomorphic to the Klein 4-group. The character table of the Klein 4-group is a ``subset'' of the character table of $\Dih_8$.
}