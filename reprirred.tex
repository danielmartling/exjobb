\subsection{Irreducible representations}

In Section~\ref{sect:subrepr} we found that a given representation can be divided into complementary subrepresentations, now we introduce the notion of an ``indivisible'' representation.

\begin{definition}[Irreducible representation]
	If there are no proper and non-trivial $G$-invariant subspaces of a representation $V$, it is said to be \textit{irreducible}. 
\end{definition}

We have already met a few of these.

\begin{example}[Degree 1 representations are irreducible]\cite[Example 1.4.2.]{Sagan}
	A vector space of dimension 1 has no other subspace other than itself and the zero space, thus it is irreducible. Hence the trivial representation of degree 1 of any group and the alternating representation of $\Sym_n$ discussed in Section~\ref{sect:basicreprs} and the degree 1 representations of $\Cyc_n$ found in Section~\ref{sect:deg1cycn} are irreducible representations of their respective groups.
\end{example}

The results from Proposition~\ref{prop:complsubrepr} invites the notion of \textit{complete reducibility} of an arbitrary representation. This is presented in the following theorem.

\begin{theorem}[Maschke's theorem]\label{thm:maschkes}
	Let $G$ be a finite group and let $V$ be any representation space of $G$. Then $V$ is composed as a direct sum of a number of not necessarily distinct subrepresentations $W_i$ of $V$, or in other words,
	\begin{align*}
		V = \bigoplus_i W_i
	\end{align*}
	\begin{note}
		The direct sum can be expressed in terms of distinct subrepresentation $W_i$ of $V$ and their multiplicities $a_i$ as
		\begin{align*}
			V = \bigoplus_i a_i W_i
		\end{align*}
	\end{note}
\end{theorem}

\begin{example}[Decomposition of permutation representation of $\Sym_n$]
	Let $G = \Sym_n$, and let $P$ be the permutation representation space, $S$ be the standard representation space and $T$ be the trivial representation space, then $V$ is fully decomposed as 
	\begin{align*}
		P = T \oplus V.
	\end{align*}
\end{example}

%\begin{example}[Decomposition of regular representation of $\Sym_n$]
%	Again, let $G = \Sym_n$, and let $R$ be the regular representation space, 
%\end{example}

\begin{theorem}[Schur's lemma]
	content...
\end{theorem}

%	In Example~\ref{example:comptrivpermrepr}, a subrepresentation of the permutation representation complementary to the trivial representation was found. Show it is irreducible? 

%	Consider the action of a group $G$ on the basis $(\bas_i - \bas_{i+1})_{i=1}^n$. For a $g \in G$ we have
%	\begin{align*}
	%		g \cdot (\bas_i - \bas_{i+1}) &= \bas_{g(i)} - \bas_{g(i+1)}
	%	\end{align*}
%	
%	\begin{example}
	%		hello
	%	\end{example}
	
