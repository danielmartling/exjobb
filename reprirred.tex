\subsection{Irreducible representations}\label{sect:irredreprs}

In Section~\ref{sect:subrepr} we found that a given representation can be divided into complementary subrepresentations, now we introduce the notion of an ``indivisible'' representation.

\begin{definition}[Irreducible representation]
	If there are no proper and non-trivial $G$-invariant subspaces of a representation $V$, it is said to be \textit{irreducible}. 
\end{definition}

We have already met a few of these.

\begin{example}[Degree 1 representations are irreducible]\cite[Example 1.4.2.]{Sagan}
	A vector space of dimension 1 has no other subspace other than itself and the zero space, thus it is irreducible. Hence the trivial representation of degree 1 of any group and the alternating representation of $\Sym_n$ discussed in Section~\ref{sect:basicreprs} and the degree 1 representations of $\Cyc_n$ found in Section~\ref{sect:deg1cycn} are irreducible representations of their respective groups.
\end{example}

The results from Proposition~\ref{prop:complsubrepr} invites the notion of \textit{complete reducibility} of an arbitrary representation. This is presented in the following theorem (really a corollary of Proposition~\ref{prop:complsubrepr}).

\begin{theorem}[Maschke's theorem]\label{thm:maschkes}
	Let $G$ be a finite group and let $V$ be any representation space of $G$. Then $V$ is composed as a direct sum of a finite number of subrepresentations $W_i$ of $V$, that is
	\begin{align*}
		V &= W_1 \oplus W_2 \oplus \dots \oplus W_k \\
		&= \bigoplus_{i=1}^k W_i.
	\end{align*}
%	\begin{note}
%		The direct sum can be expressed in terms of distinct subrepresentation $W_i$ of $V$ and their multiplicities $a_i$ as
%		\begin{align*}
%			V = \bigoplus_i a_i W_i
%		\end{align*}
%	\end{note}
\end{theorem}
\begin{proof}
	The theorem will be proved using induction on complementary subrepresentations as described by Proposition~\ref{prop:complsubrepr}. If $V$ itself is irreducible, then we are done. If $V$ is not irreducible, then by definition there exists a non-trivial and non-zero subrepresentation $W$ of $V$, and also by Proposition~\ref{prop:complsubrepr}, there exists a complementary subrepresentation $W'$ of $V$ such that $V = W \oplus W'$. If both $W$ or $W'$ are irreducible, then we are done, if either or both are not, we then apply Proposition~\ref{prop:complsubrepr} on them. By the induction hypothesis and by the fact that the dimension of $V$ is finite, $V$ will be decomposed into a direct sum of a finite number of subrepresentations and we are done.
\end{proof}

%\begin{example}[Decomposition of permutation representation of $\Sym_n$]
%	Let $G = \Sym_n$, and let $P$ be the permutation representation space, $S$ be the standard representation space and $T$ be the trivial representation space, then $V$ is fully decomposed as 
%	\begin{align*}
%		P = T \oplus V.
%	\end{align*}
%\end{example}

%\begin{example}[Decomposition of regular representation of $\Sym_n$]
%	Again, let $G = \Sym_n$, and let $R$ be the regular representation space, 
%\end{example}
Maschke's theorem allows us to decompose any arbitrary representation into the direct sum of irreducible representations, however it does not say anything about the uniqueness of a given composition. We need to study which kinds of $G$-linear maps that are permitted between two irreducible representations, a key insight of Schur's Lemma.

\begin{theorem}[Schur's Lemma]\cite[Lemma.1.7.]{FultonHarris}, \cite[Prop.2.4.]{Serre}\label{thm:schur}
	Let $V$ and $W$ be irreducible representations of a group and let $\varphi: V \rightarrow W$ be a $G$-linear map. Then:
	\begin{itemize}
		\item[i)] Either $\varphi$ is an isomorphism, or $\varphi$ is the zero map, $\varphi = \0$.
		\item[ii)] If $V$ and $W$ are isomorphic, then $\varphi = \lambda \cdot \id$, for some $\lambda \in \CC$.
	\end{itemize}
\end{theorem}
\begin{proof}
	Item i) follows from the $G$-linearity of $\varphi$, since by Proposition~\ref{prop:kernelimagelinearmap}, $\ker \varphi$ is a subrepresentation of $V$, but $V$ is irreducible, hence $\ker \varphi$ must then be all of $V$ ($\varphi = \0$) or $\{\0\}$ ($\varphi$ is injective). Likewise, $\im \varphi$ is a subrepresentation of the irreducible $W$, hence $\im \varphi$ is either $W$ ($\varphi$ is surjective) or $\{\0\}$ ($\varphi = \0$). Combining the two cases, we must have that either $\varphi$ is an isomorphism, or that $\varphi = \0$.
	
	Consider the case where $V$ and $W$ are isomorphic and let $\lambda$ be a eigenvalue of $\varphi$. The map $\varphi - \lambda \cdot \id$ is also a $G$-linear map since it is the sum of two such maps. Since $\lambda$ is chosen to be a eigenvalue of $\varphi$ then $\ker(\varphi - \lambda \cdot \id)$ is non-empty (it contains the eigen vector associated with $\lambda$) so by part i) of this proof we must have that $\varphi - \lambda \cdot \id$ is the zero map, hence $\varphi = \lambda \cdot \id$, and ii) follows.
\end{proof}

In other words, there are no non-trivial maps between two inequivalent irreducible representations. \textit{(In even other words: commutativity in matrices is very rare.)}

We can now conjecture that if $V$ is an arbitrary representation of some group, then it is decomposed into the direct sum of irreducibles $V_i$ by
\begin{align*}
	V &= m_1 V_1 \oplus m_2 V_2 \oplus \dots \oplus m_k V_k \\
	&= \bigoplus_{i=1}^k m_i V_i,
\end{align*}
where $m_i$ is the multiplicity of $V_i$ in $V$.

Schur's lemma also allows us to say something about the irreducible representations of an abelian group, that is a group in which every element commutes with every other element (for example $\Cyc_n$).

\begin{corollary}\label{cor:abelianirred}
	Let $G$ be an abelian group. Then any irreducible representation of $G$ is of degree 1.
\end{corollary}
\begin{proof}\cite[\textit{Mentioned in passing in} Sect.1.3.]{FultonHarris}
	Let $\rho: G \rightarrow \GL(V)$ be an irreducible representation of $G$. Consider the linear map $\rho_h$ of some $h \in G$, then for any $g \in G$ we have
	\begin{align*}
		\rho_h \cdot \rho_g &= \rho_{gh} & \text{($\rho$ is a homomorphism)} \\
		&= \rho_{hg} & \text{($G$ abelian)} \\
		&= \rho_h \cdot \rho_g. & \text{(homomorphism again)}
	\end{align*}
	We have found that $\rho_h: V \rightarrow V$ is a $G$-linear map, provided $G$ is an abelian group. Then, by Schur's Lemma ii) it is a scalar multiple of the identity function, let's say $\rho_h = \lambda \cdot \id$ for some $\lambda \in \CC$. Let $\vvec \in V$, then we have that
	\begin{align*}
		\varphi(\vvec) &= \lambda \id \cdot \vvec \\
		&= \lambda \vvec \in \Span(\vvec) & \text{($\id$ fixes $\vvec$)}
	\end{align*}
	Now, this means that the action on $\rho_g$ on an arbitrary $\vvec \in V$ changes $\vvec$ by a scalar factor, hence the one-dimensional $\Span(\vvec)$ is a $G$-invariant subspace of $V$. However, $V$ was taken to be an irreducible representation, hence $V$ is identical to this subspace and $\dim V = 1$.
\end{proof}

\begin{example}[$\Sym_3$]
	One could show that the matrices of $(1,2)$ and $(2,3)$ in the standard representation can be shown to commute only with a scalar multiple of the identity matrix, however those of $(1,2,3)$ and $(1,3,2)$ commute with each other since the rotation subgroup of $\Sym_3$ is isomorphic to $\Cyc_3$, an abelian group which by Corollary~\ref{cor:abelianirred} only have irreducible representations of degree 1, hence the ``standard representation of $\Sym_3$'' is not irreducible on $\Cyc_3$.
\end{example}

