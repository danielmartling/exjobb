\clearpage{\thispagestyle{empty}}
\section{Introduction}

	Intro to representation theory.
	Replacing groups with groups of matrices behaving the same way as the group. Replacing group operation with matrix multiplication. This is the intuition behind a homomorphism.
	
	Only finite dimensional vector spaces. Note on infinite dimensional spaces (Serre).
	
	Only finite groups. Note on compct and infinite groups (Serre/FultonHarris).
	
	As examples, we study the abelian cyclic group - the group of rotations, and the symmetric group - the group of permutations. Both of small order.
	
	Section 2 introduces some background topics from group theory and linear algebra required before introducing representation theory. Expecially the symmetric group,  recollections of linear algebra and a brief introduction to tensor operations on vector spaces.
	
	The idea of a group representation is introduced in section 3 as a homomorphism (a map respecting the group operation) bringing the elements of a group into the group of automorphisms of some vector space along with a few basic consequences of the definition. Some typical representations follow with examples applied on some small cyclic and symmetric groups. The notion of subrepresentation, analogous to vector subspaces, is introduced along with tensor operations on representations, allowing us to construct new representations out of already known ones. Some representations are indivisible however and are called irreducible, inviting the main theorem of this text that any representation of a finite group in a finite dimensional vector space is composed of a finite number of constituent subrepresentations (Maschke's Theorem~\ref{thm:maschkes}), which along with Schur's Lemma (Theorem~\ref{thm:schur})  allows us to claim that these decompositions are unique. Every concept has an example from the symmetric or cyclic groups.
	
	The last section is on character theory which strips back the ``big'' concepts of linear algebra and matrix calculations into the trace of those matrices, a metric related to the eigen values of a matrix, releasing us from the ambiguities of choices of bases. The typical representations found earlier are reevaluated and we see that the results allows us to find every irreducible representation of a group by prividing a limit for the number of irreducibles, and also completely decompose any arbitrary representation into irreducibles with some ``character calculus''. The cyclic groups are quickly studied, and finally the symmetric groups of degree 3 and 4 are closely studied.