\clearpage{\thispagestyle{empty}}
\section{Introduction}

	The main purpose of this text is to study linear representations of groups, or in other words replacing abstract groups with sets of matrices behaving the same way as the elements of the group and replacing the group operation with matrix multiplication. This is the intuition behind a homomorphism. The reader is assumed to be familiar with first-year linear algebra and the definitions and basic concepts of vector spaces and groups.
	
	Only representations of finite groups in finite dimensional vector spaces will be treated. As \cite[Sect.1.1.]{Serre} notes, we are usually interested in a finite number of elements of a vector space, and then we could span a finite dimensional subspace with those elements. The restriction to finite groups is more severe however, as some results in this text does not apply on infinite groups, or even compact infinite groups (see \cite{FultonHarris, Serre}). 
	
	Calculations performed in this text are my own and the examples are also my own, unless otherwise noted. As practical examples we study the abelian cyclic groups and the symmetric groups of small order extensively.
	
	Section 2 introduces some background topics from group theory and linear algebra required before introducing representation theory. Especially the symmetric group, recollections of linear algebra and a brief introduction to tensor operations on vector spaces are presented.
	
	The idea of a representation is introduced in section 3 as a group homomorphism (a map respecting the group operation) bringing the elements of a group into the set of invertible transformations of some vector space along with a few basic consequences of the definition. Some typical representations follow with examples applied on some small cyclic and symmetric groups. The notion of subrepresentation, analogous to vector subspaces, is introduced along with tensor operations on representations, allowing us to construct new representations out of already known ones. Some representations are indivisible however and are called irreducible, inviting the main theorem of this text that any representation of a finite group in a finite dimensional vector space is composed of a finite number of constituent subrepresentations (Maschke's Theorem~\ref{thm:maschkes}), which along with Schur's Lemma (Theorem~\ref{thm:schur})  allows us to claim that these decompositions are unique.
	
	The last section is on character theory which strips back the ``big'' concepts of linear algebra and matrix calculations into the trace of those matrices, a metric related to the eigenvalues of a matrix, releasing us from the ambiguities of choices of bases. The representations found earlier are reevaluated and we see that the results allows us to find every irreducible representation of a group by providing a limit for the number of irreducibles, and also completely decompose any arbitrary representation into irreducibles with some ``character calculus''. The cyclic groups are quickly studied, and finally the symmetric groups of degree 3 and 4 are closely studied.