\subsection{Character table of $S_4$}

This section follows the same methods from the previous section on $S_3$. So far, we have three irreducible representations of $S_4$: The trivial and the alternating representation from section \ref{sect:trivrepr}; the standard representation from section \ref{sect:standSn} and also we can surmise that there is a regular representation of $S_4$ with character $(24,0,0,0,0)$ from theorem \ref{thm:charregSn}. Along with the composite permutation representation, the character table of $S_4$ is presented in table \ref{table:charS4}. 

\begin{table}[hbt!]
	\centering
	\begin{tabular}{c | c c c c c}
		     $S_3$      & $[1]$   & $[12]$  & $[123]$ & $[1234]$ & $[(12)(34)]$ \\
		$\Ss|[\sigma]|$ & $\Ss 1$ & $\Ss 6$ & $\Ss 8$ & $\Ss 6$  & $\Ss 3$      \\ \hline
		   $\chi_T$     & 1       & 1       & 1       & 1        & 1            \\
		   $\chi_A$     & 1       & -1      & 1       & -1       & 1            \\
		   $\chi_S$     & 3       & 1       & 0       & -1       & -1           \\ \hdashline
		   $\chi_P$     & 4       & 2       & 1       & 0        & 0            \\
		   $\chi_R$     & 24      & 0       & 0       & 0        & 0            	\end{tabular}
	\caption{Character table of $S_4$.}
	\label{table:charS4}
\end{table}

Recall that there are five conjugacy classes in $S_4$, then by theorem \ref{thm:numberirrep} we can expect two more irreducible representations of $S_3$. To find them, we compose new representations out of the ones we have found so far, using the toolbox of section \ref{sect:tensorops}.

\paragraph{Decomposition of $A \otimes S$.} The character of this composition is $(3,-1,0,1,-1)$, an ``alternating'' version of $S$, and taking the inner product of it with itself we find $|\chi_{A \otimes S}|^2 = 1$ and thus $A \otimes S$ is an irreducible representation. From this point on it is denoted as $S'$.

\paragraph{Decomposition of $S \otimes S$.} The character of this composition is $(9,1,0,1,1)$ and it is found to be linearly dependent on $\chi_T$,  $\chi_S$ and $\chi_{A \otimes S}$ with overlaps of 1 and independent of $\chi_A$, however $T$, $S$ and $S'$ together are of degrees 1, 3 and 3 and $S \otimes S$ is of degree 9, so there is either another representation of degree 2 or two of degree 1. \textit{SOURCE, theorem om square sum of $\dim V_i$}. Denoting this representation by $V$, its character is $\chi_V = \chi_{S \otimes S} - \chi_T - \chi_S - \chi_{S'} = (2, 0, -1, 0, 2)$, which is found to be such that $|\chi_V|^2 = 1$, hence $V$ is irreducible and $S \otimes S$ is found to be decomposed to $T \oplus V \oplus S \oplus S'$. We have now found all five irreducible representations of $S_4$.

\paragraph{Decomposition of $R$.} The character of $R$ is $(24,0,0,0,0)$ and its overlaps with $T, A, V, S$ and $S'$ are respectively 1, 1, 2, 3 and 3, so $R = T \oplus A \oplus 2V \oplus 3S \oplus 3S'$.

\paragraph{Decomposition of $\Sym^2S$ and $\bigwedge^2S$.} The character of $\Sym^2S$ is 
\begin{align}
	\frac{1}{2}\left[ (9,1,0,1,1)-(3,3,0,-1,3) \right] = (3,-1,0,1,-1)
\end{align}
which is the character of $S'$. Likewise, the character of $\bigwedge^2S$ is $(6,2,0,0,2)$ which is found to be the sum of the characters of $T, S$ and $V$, hence $\bigwedge^2S = T \oplus S \oplus V$.

\paragraph{Decomposition of the $n$th tensor power of $V$.}

The character of $V^{\otimes n}$ is
\begin{align}
	\chi_{V^{\otimes n}} = (2^n, 0, (-1)^n, 0, 2^n).
\end{align}

Its overlap with $T, A, V, S$ and $S'$ are respectively $a_n, a_n, a_{n+1}, 0$ and $0$, where 
\begin{align}
	a_n := \frac{1}{24}\left( 4 \cdot 2^n + 8 \cdot (-1)^n \right),
\end{align}
so the decomposition is
\begin{align}
	V^{\otimes n} = a_n (T \oplus A) \oplus a_{n+1} V.
\end{align}

\paragraph{Decomposition of the $n$th tensor power of $S$.} The character of $S^{\otimes n}$ is 
\begin{align}
	(\chi_S)^n = (3^n, 1, 0, (-1)^n, (-1)^n),
\end{align}
and after projecting it on the characters of the irreducible representations from table \ref{table:completecharS4}, we find that
\begin{align}\label{eq:nthpowerS}
	S^{\otimes n} = a_n T \oplus b_n A \oplus c_n V \oplus a_{n+1} S \oplus b_{n+1} S',
\end{align}
where 
\begin{align}
	\begin{cases}
		a_n = \frac{1}{24}(3^n + 9(-1)^n+6), \\
		b_n = \frac{1}{24}(3^n - 3(-1)^n -6), \text{ and} \\
		c_n = \frac{1}{24}(2\cdot 3^n + 6(-1)^n).
	\end{cases}
\end{align}

\paragraph{Decomposition of the $n$th tensor powers of $S'$.} Since $S' = A \otimes S$, the $n$th power of $S'$ is the tensor product of $n$ factors of $A \otimes S$. Since the tensor product is a commutative operation, we can rearrange it as
\begin{align}
	(A \otimes S)^{\otimes n} = S^{\otimes n} \otimes A^{\otimes n}
\end{align}
The $n$th tensor power of $A$ is $T$ if $n$ is even and $A$ if $n$ is odd, so for even $n$, $S'^{\otimes n} \cong S^{\otimes n}$, and for odd $n$, $S'^{\otimes n} \cong S^{\otimes n} \otimes A$. Then the character is 
\begin{align}
	\chi_{S'^{\otimes n}} = \begin{cases}
		\big(3^n,1,0,(-1)^{n}, (-1)^n\big), \text{ if } n \text{ is even, or} \\
		\big(3^n,-1,0,(-1)^{n+1}, (-1)^n\big), \text{ if } n \text{ is odd.}
	\end{cases}
\end{align}
Tensor multiplying equation \ref{eq:nthpowerS} with $A$, we have:
\begin{align}
	S^{\otimes n} \otimes A &= \Big(a_n T \oplus b_n A \oplus c_n V \oplus a_{n+1} S \oplus b_{n+1} S'\Big)\otimes A \\
	&= b_n T \oplus a_n A \oplus c_n V \oplus b_{n+1} S \oplus a_{n+1} S',
\end{align}
and then we have found the decomposition of all tensor powers of $S'$. For even $n$, it is the same as $S^{\otimes n}$, for odd $n$, $A$ and $T$, and $S$ and $S'$ switch multiplicities.

\paragraph{} The findings of the last few paragraphs (except those on larger tensor powers) are presented in table \ref{table:completecharS4}.

\begin{table}[hbt!]
	\centering
	\begin{tabular}{c | c c c c c | l}
		       $S_4$         & $[1]$   & $[12]$  & $[123]$ & $[1234]$ & $[(12)(34)]$ &                                                              \\
		  $\Ss|[\sigma]|$    & $\Ss 1$ & $\Ss 6$ & $\Ss 8$ & $\Ss 6$  & $\Ss 3$      & \textit{Alternate compositions}                              \\ \hline
		      $\chi_T$       & 1       & 1       & 1       & 1        & 1            &                                                              \\
		      $\chi_A$       & 1       & -1      & 1       & -1       & 1            &                                                              \\
		      $\chi_V$       & 2       & 0       & -1      & 0        & 2            &                                                              \\
		      $\chi_S$       & 3       & 1       & 0       & -1       & -1           &                                                              \\
		$\chi_{S'}$ & 3       & -1      & 0       & 1        & -1           & $S' :\cong A \otimes S \cong \Sym^2S$                                  \\ \hline\hline
		      $\chi_P$       & 4       & 2       & 1       & 0        & 0            & $P \cong T \oplus S$                                         \\
		   $\chi_{\bigwedge^2S}$    & 6       & 2       & 0       & 0        & 2            & $\bigwedge^2S \cong T \oplus S \oplus V$                     \\
		$\chi_{S \otimes S}$ & 9       & 1       & 0       & 1        & 1            & $S \otimes S \cong T \oplus V \oplus S \oplus S' \cong S' \otimes S'$ \\
		      $\chi_R$       & 24      & 0       & 0       & 0        & 0            & $R \cong T \oplus A \oplus 2V \oplus 3S \oplus 3S'$
	\end{tabular}
	\caption{Complete character table of $S_4$. The representations above the dashed line are irreducibles, and those below are composed. Some compositions are presented in the right-most column.}
	\label{table:completecharS4}
\end{table}