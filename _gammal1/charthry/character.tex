\clearpage{\thispagestyle{empty}}
\section{Character Theory}

%	For a representation $\rho$ of a group $G$ in a vector space $V$, the character of the representation is defined as the trace of the matrix representation in $\GL(V)$ of a element $g$ of $G$. If the matrix representation is given by
%	\begin{align}
%		X(g) = \left( x_{ij} \right)_{n \times n}
%	\end{align}
%	then the character is
%	\begin{align}
%		\chi_V(g) = \Tr X(g) = \sum_{i=1}^{n} x_{ii}.
%	\end{align}
%	Evident from this definition, the character of a degree 1 representation is the same as the representation, since a $1 \times 1$-matrix is usually identified with its only element.
%	
%	The character of the entire group $G$ in a representation $V$ is a vector 
%	\begin{align}
%		\chi_V = \left( \chi_V(g) \right)_{g \in G},
%	\end{align}
%	sometimes abbreviated to just containing a character for every conjugacy class
%	\begin{align}
%		\chi_V = \left( \chi_V([g]) \right)_{[g] \subseteq G},
%	\end{align}
%	since ``character'' is a class function on the elements of $G$.
%	
%	Since the trace of a matrix is invariant under conjugation, the character
	
	\begin{definition}[Character of a representation]\label{def:character}
		Let $\rho$ be a representation of degree $n$ of a group $G$ in a vector space $V$. The \textbf{character} of the representation is defined to be the trace of the representation. In other words, if $\rho(g)$ is the representation of a $g$ in $G$ and it has the matrix representation
		\begin{align}
			\rho_V(g) = (x_{ij})_{n \times n},
		\end{align}
		then the character of this $g$ is
		\begin{align}
			\chi_V(g) := \Tr \rho_V(g) = \sum_{i=1}^{n} x_{ii}.\label{eq:deftrace}
		\end{align}
		When $V$ or $g$ are contextually obvious they are omitted, ie. $\chi = \Tr \rho$.
	\end{definition}
	
	\begin{corollary}[Character of degree 1 representation]\label{cor:chardeg1}
		Evident from definition \ref{def:character}, the character of a degree 1 representation is the same as the representation, since a $1 \times 1$-matrix is usually identified with its only element.
	\end{corollary}
	\begin{proof}
		For a degree 1 representation, $n=1$ and then equation \ref{eq:deftrace} becomes $\chi_V(g) = x_{11}$ which is identified with $(x_{11}) = \rho_V(g)$.
	\end{proof}
	
	Recall from linear algebra that two similar matrices have the same trace \cite[Thm 5.4.]{Holst} and that two matrices $X$ and $Y$ are similar if there exists another matrix $S$ such that $SXS^{-1} = Y$. 
	
	Also, recall that the trace is the sum of the eigenvalues of a matrix. \textit{SOURCE}
	
	\begin{lemma}[Trace function is a class function]
		The trace function = ``character function'' is invariant under conjugation, it is a class function on the elements of $G$, it partitions $G$ into disjoint conjugacy classes where all $g$ in the conjugacy class $[g]$ has the same character.
	\end{lemma}
	
	\begin{theorem}
		The character of a representation is independent of chosen matrix representation and chosen representative of a conjugacy class.
	\end{theorem}
	\begin{proof}
		Consequence of previous definitions and recollections.
	\end{proof}
	
	\begin{theorem}[Expected number of irreducible representations]\label{thm:numberirrep}
		The number of irreducible representations of a group is the same as the number of conjugacy classes of the group.
	\end{theorem}
	\begin{proof}
		SOURCE
	\end{proof}
	
	\subsection{Projection formulae, Schur's lemma}
	
	\textit{TO BE WRITTEN}
	