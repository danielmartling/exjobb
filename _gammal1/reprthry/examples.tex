\subsection{Trivial representations.}\label{sect:trivrepr}

\paragraph{Trivial representation.}


		For any group $G$ there is a trivial representation in any vector space $V$ defined by 
	\begin{align}
		\rho(g) = 1,
	\end{align}
	for every $g$ in $G$. It is an irreducible representation of degree 1. 

\paragraph{Alternating representation of $S_n$.}

	Choosing $G$ to be the symmetric group of degree $n$, denoted $S_n$, another degree 1 representation can be found by studying the signs, or parities, of the elements of $S_n$. For any two permutations $\sigma$ and $\tau$ of $S_n$ with respective sign $(-1)^s$ and $(-1)^t$, the sign of their composition is 
	\begin{align}
		\sgn(\sigma \circ \tau) = (-1)^s \cdot (-1)^t = (-1)^{st} = \sgn(\sigma) \cdot \sgn(\tau),
	\end{align}
	so clearly this sign function is a homomorphism hence the $\sgn$ function is representation of degree 1 and is called the alternating representation.	

\begin{notation}
	From this point on, any trivial representation or its associated vector space may be denoted by $T$ and likewise any alternating representation may be denoted by $A$.
\end{notation}
	

\subsection{Permutation representation}

	We could let the group $G$ act on an appropriately chosen set $X$, and study the action of $G$ on $X$ defined by
	\begin{align}
		G \times X \rightarrow X
	\end{align} 
	by permuting the elements of $X$. This action is defined by any $g$ in $G$ as 
	\begin{align}
		(g,x) \mapsto g \cdot x
%		g \cdot x = gx
	\end{align}
	for any $x$ in $X$.
%	\begin{table}[hbt!]\centering\begin{tabular}{ c c c}
%			 $G \times X$    & $\rightarrow$ & $X$ \\
%			\rotatebox[origin=c]{90}{$\in$}&&\rotatebox[origin=c]{90}{$\in$} \\
%			$(g,x)$ & $\mapsto$     & $gx$.
%	\end{tabular}\end{table}
	
	Further more, define a vector space $V$ spanned by the orthogonal basis $(\ehat_x)_{x \in X}$, then we have a homomorphism
	\begin{align}
		\rho: G \rightarrow \GL(V)
	\end{align}
	defined by its action on the basis $\mathcal{B}$ by, for any $g$ in $G$,
	\begin{align}
		\rho(g): \ehat_x \mapsto g \cdot \ehat_x = \ehat_{gx}.
	\end{align}
%	or on an arbitrary vector $v$ in $V$ as
%	\begin{align}
%		\rho(g): \vvec \mapsto g\vvec.
%	\end{align}
	
	It is a homomorphism since for any $g$ and $h$ in $G$,
	\begin{align}
		\rho(g) \rho(h) \cdot \ehat_x = \rho(g) \cdot \ehat_{hx} = \ehat_{ghx} = \rho(gh) \cdot \ehat_x
	\end{align}
	for any $x$ in $X$. The permutation representation is denoted $P$. %is of degree $|X|$ and
	
	
	\subsubsection{Permutation representation of $S_n$}
	For the symmetric group of order $n$, denoted $S_n$, it would be appropriate to choose the set $X = \{1,2, \dots, n\}$ and to let any $\sigma$ from $S_n$ permute any $i$ from $X$ by $(\sigma,i) \mapsto \sigma(i)$. Choosing a basis $(\ehat_i)_{i=1}^n$ to span a vector space $V$ of dimension $n$, we define a homomorphism 
	\begin{align}
		\rho: S_n \rightarrow \GL(V)
	\end{align}
	defined by any $\sigma$ in $S_n$ as
	\begin{align}
		\rho(\sigma): \ehat_i \mapsto \ehat_{\sigma(i)},
	\end{align}
	for any $i \in \{1,\dots, n\}$. A corresponding set of matrix representations are the permutation matrices
	\begin{align}
		\rho(\sigma) = (x_{ij})_{n \times n}, \text{ where } x_{ij} = \delta_{j,\sigma(i)} = \begin{cases}
			1 \text{ if } j = \sigma(i),\\
			0 \text{ otherwise.}
		\end{cases}
	\end{align}
	where $\delta_{ij}$ is the Kronecker delta.	This representation could be interpreted as vectors in an $n$-dimensional spaces where the action of $S_n$ permutes the coordinate axes. For example, a vector 
	\[
		(a_1, a_2, \dots, a_n) = a_1 \ehat_1 + a_2 \ehat_2 + \dots + a_n \ehat_n
	\]
	of $V$ is permuted to 
	\[
		(a_{\sigma^{-1(1)}}, a_{\sigma^{-1(2)}}, \dots, a_{\sigma^{-1(n)}}) = a_1 \ehat_{\sigma(1)} + a_2 \ehat_{\sigma(2)} + \dots + a_n \ehat_{\sigma(n)}.
	\]
	In other words, $\sigma$ permutes the $i$th axis to the $\sigma(i)$th axis.
	
	\begin{example}[Permutation representation of $S_2$]
		The symmetric group of degree 2 has two elements, $S_2 = \{(1), (12)\}$ and their matrix representations in $\mathbf{\ehat_1},\mathbf{\ehat_2}$-space are presented in table \ref{table:permS2}.
		\begin{table}[hbt!]
			\centering
			\begin{tabular}{c c}
				$\rho(1) =
				\begin{pmatrix}
					1 & 0 \\ 0 & 1
				\end{pmatrix}$, &
				$\rho(12) =
				\begin{pmatrix}
					0 & 1 \\ 1 & 0
				\end{pmatrix}$.
			\end{tabular}
			\caption{Matrix representations of $S_2$}
			\label{table:permS2}
		\end{table}
	\end{example}
	
	\begin{example}[Permutation representation of $S_3$]\label{ex:permS3}
		Likewise, representations of $S_3$ are presented in table \ref{table:permS3}.
		\begin{table}[hbt!]
			\centering
			\begin{tabular}{r r r}
				$\rho(e) = 
				\begin{pmatrix}
					1 & 0 & 0 \\
					0 & 1 & 0 \\
					0 & 0 & 1
				\end{pmatrix}$ & 
				$\rho(123) = 
				\begin{pmatrix}
					0 & 0 & 1 \\
					1 & 0 & 0 \\
					0 & 1 & 0
				\end{pmatrix}$ & 
				$\rho(132) = 
				\begin{pmatrix}
					0 & 1 & 0 \\
					0 & 0 & 1 \\
					1 & 0 & 0
				\end{pmatrix}$ \\ & & \\
				$\rho(12) = 
				\begin{pmatrix}
					0 & 1 & 0 \\
					1 & 0 & 0 \\
					0 & 0 & 1
				\end{pmatrix}$ &
				$\rho(13) = 
				\begin{pmatrix}
					0 & 0 & 1 \\
					0 & 1 & 0 \\
					1 & 0 & 0
				\end{pmatrix}$ &
				$\rho(23) = 
				\begin{pmatrix}
					1 & 0 & 0 \\
					0 & 0 & 1 \\
					0 & 1 & 0
				\end{pmatrix}$
			\end{tabular}
			\caption{Matrix representations of $S_3$}
			\label{table:permS3}
		\end{table}
	\end{example}
		
		%			$$S_3 = \{(1), (12), (13), (23), (123), (132)\}$$
		%			$$S_4 = \{(1), 
		%						(12), (13), (14), (23), (24), (34), 
		%						(12)(34), (13)(24), (14)(23), 
		%						(123), (124), (134), (132), (142), (143), (234), (243),
		%						(1234), (1243), (1324), (1342), (1423), (1432)
		%						\}$$
	
\subsubsection{Standard representation of $S_n$}\label{sect:standSn}

	The question is, is the permutation representation of the symmetric group an irreducible representation? Consider the one-dimensional subspace $W$ of $V$ spanned by the sum of all basis vectors of $V$,
	\begin{align}
		W = \Span\left\lbrace \ehat_1 + \ehat_2 + \dots + \ehat_n \right\rbrace.
	\end{align}
	It is $S_n$-invariant since any $\sigma$ will leave any vector in $W$ fixed. $W$ is an $S_n$-invariant subspace of $V$, thus it is a subrepresentation. Further more, it is irreducible and it is isomorphic to the trivial representation, up to a scalar factor, or in other words $W \cong \C T$.
	
	The complementary $S_n$-invariant subspace of $V$ is 
	\begin{align}
		W^\perp = \Span\left\lbrace \ehat_1 - \ehat_2, \ehat_2 - \ehat_3, \dots, \ehat_{n-1} - \ehat_n \right\rbrace.
	\end{align}
	Since the permutation representation of $S_n$ is of degree 1, the standard representation of $S_n$ is an irreducible\footnote{I have not proved it is irreducible?} representation of degree $n-1$ and is denoted $S$. 
	
	Here we have taken a ``larger'' representation $P$ and decomposed it into two ``smaller'' complementary subspaces $T$ and $S$. In other words, $P$ is the direct sum of the irreducible subrepresentations $T$ and $S$:
	\begin{align}
		P = T \oplus S.
	\end{align}
	
	\begin{example}[Standard representation of $S_3$]\label{ex:standS3}
		Regarding the permutation representation as the action of $S_3$ on the coordinates of points in a three dimensional space, the one-dimensional subrepresentation to the permutation represent found earlier is the diagonal line through the origin parallel to the sum of all basis vectors. The complementary subrepresentation is the two-dimensional plane perpendicular to this diagonal line intersecting it in the origin.
		
		
		\textit{table of 2-dim standard representation matrices of $S_3$. $S$ is action of $A_3$ on a triangle in the plane.}
		
	\end{example}

\subsection{Regular representation}

	Instead of letting the group $G$ act on some arbitrary set by permutation, we could let it act on itself. This action is defined as
	\[
	G \times G \rightarrow G
	\]
	which for every $g$ in $G$ is given by 
	\[
	(g,h) \mapsto g \cdot h,
	\]
	for a $h$ in $G$. The corresponding vector space $V$ is spanned by basis vectors $\{\ehat_g\}_{g \in G}$ constructed from the members of $G$. A representation of $G$ in $V$ is then
	\[
	\rho: G \rightarrow \GL(V)
	\]
	defined by
	\[
	\rho(g)(\ehat_h) = g\ehat_h = \ehat_{gh}.
	\]
	
	To show that it is a homomorphism, one could follow the same steps as above. This is called the regular representation and is denoted by $R$.
	
	\subsubsection{The regular representation of cyclic group $Z_n$}\label{sect:reprZn}
	
		\textit{Is this really the \underline{regular} representation?}
	
		Recall that the cyclic group of order $n$ is generated by an element $g$ of order\footnote{The order of an element $g$ is the smallest possible integer $n$ such that $g^n = e$.} $n$ if $Z_n = \{e, g, g^2, \dots, g^{n-1} \}$. Sometimes we denote the cyclic group as $Z_n = \langle g \rangle$ if $g$ is such a generator. %A cyclic group may have multiple generators that generate the group separately, for example the integers modulo 12, $\Z/12\Z \cong Z_{12}$ has 1, 5, 7 and 11 as generators. 
		
		Consider a representation of $Z_n$ as the homomorphism
		\begin{align}
		\rho: Z_n \longrightarrow \C
		\end{align}
		defined by 
		\begin{align}
			\rho(e) = 1 \text{ and } \rho(g^a g^b) = \rho(g^a)\rho(g^b).
		\end{align}
		Since $g^n = e$ we must have that $\rho(g^n) = 1$, then $\rho(g)$ must be mapped to an $n$th root of unity. Denote by $\omega = e^{2\pi i/n}$ the primitive\footnote{The first non-real $n$th root of unity} root of unity. So for $Z_n$, there are $n$ degree 1 representations $\rho_m: Z_n \rightarrow \C$ that maps $g^k$ to $\omega^{km}$.
		
		\begin{example}[Regular representations of $Z_3$]
			The three third roots of unity are $1$, $\omega = \frac{-1+i\sqrt{3}}{2}$ and $w^2 = \frac{-1-i\sqrt{3}}{2}$, so the three representations of $Z_3$ are presented in table \ref{table:Z3}. Note that $\rho_1^* = \rho_2$.
			
			\begin{table}[hbt!]
				\label{table:Z3}
				\begin{tabular}{c | c c c}
					$Z_3$ & $e$ & $g$        & $g^2$      \\ \hline
					$\rho_0$          & 1   & 1          & 1          \\
					$\rho_1$          & 1   & $\omega$   & $\omega^2$ \\
					$\rho_2$          & 1   & $\omega^2$ & $\omega$
				\end{tabular}
				\centering
				\caption{Three representations of $Z_3$.}
			\end{table}
		\end{example}
		
		\begin{example}[Regular representations of $Z_4$]
			The four fourth roots of unity are $1$, $i$, $-1$ and $-i$, so the four representations of $Z_4$ are presented in table \ref{table:Z4}. It is clearly seen that $\rho_1$ and $\rho_3$ are each others dual representation.  
			
			\begin{table}[hbt!]
				\begin{tabular}{c | c c cl}
					\label{table:Z4}
					$Z_4$ & $e$ & $g$  & $g^2$ & $g^3$ \\ \hline
					         $\rho_0$           & 1   & 1    & 1     & 1     \\
					         $\rho_1$           & 1   & $i$  & $-1$  & $-i$  \\
					         $\rho_2$           & 1   & $-1$ & $1$  & $-1$   \\
					         $\rho_3$           & 1   & $-i$ & $-1$   & $i$
				\end{tabular}
				\centering
				\caption{Four representations of $Z_4$.}
			\end{table}
		\end{example}
		
		\begin{example}[Regular representations of $Z_5$]
			The five fifth roots of unity are $\exp{\frac{2\pi im}{5}}$, so five representations of $Z_5$ are presented in table \ref{table:Z5}. Note that $\rho_1^* = \rho_4$ and $\rho_2^* = \rho_3$.
			
			\begin{table}[hbt!]
				\begin{tabular}{c | c c c c c}
					\label{table:Z5}
					$Z_5$ & $e$ & $g$        & $g^2$      & $g^3$      & $g^4$      \\ \hline
					$\rho_0$            & 1   & 1          & 1          & 1          & 1          \\
					$\rho_1$            & 1   & $\omega^1$ & $\omega^2$ & $\omega^3$ & $\omega^4$ \\
					$\rho_2$            & 1   & $\omega^2$ & $\omega^4$ & $\omega^1$ & $\omega^3$ \\
					$\rho_3$            & 1   & $\omega^3$ & $\omega^1$ & $\omega^4$ & $\omega^2$ \\
					$\rho_4$            & 1   & $\omega^4$ & $\omega^3$ & $\omega^2$ & $\omega^1$
				\end{tabular}
				\centering
				\caption{Five representations of $Z_5$. $\omega = e^{2 \pi i/5}$.}
			\end{table}
		\end{example}
		
	\subsubsection{The regular representation of $S_3$}\label{sect:regS3}
	
		Defining a homomorphism $R: S_3 \rightarrow \GL(V)$ on the vector space spanned by the basis\footnote{Explicitly, this is $\left\lbrace \ehat_1, \ehat_{12}, \ehat_{13}, \ehat_{23}, \ehat_{123}, \ehat_{132} \right\rbrace$.} $(\ehat_{\sigma})_{\sigma \in S_3}$ defined for every $\sigma$ by $R(\sigma) \cdot \ehat_\tau = \ehat_{\sigma \circ \tau} = \ehat_{\sigma\tau}$ for some $\tau$, we arrive at a degree 6 representation of $S_3$. It is clearly a larger degree representation of $S_n$ than any we have found so far. Later we will find that for $S_3$, $S$ is the direct sum of the trivial representation, the alternating representation and two copies of the standard representations, that is $S = T \oplus A \oplus 2S$.
		
		\textit{Table of $6 \times 6$-matrices :-(}

		
			
			