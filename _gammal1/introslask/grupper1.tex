\clearpage{\thispagestyle{empty}}
\section{Groups}

%This section on groups is mostly based on the treatment on groups of \cite{DummitFoote}.
%
%\begin{definition}[Group]
%	A group is a set $G$ together with a binary operation $\star$ fulfilling:
%	\begin{enumerate}[G1.]
%		\setlength\itemsep{0em}
%		\item Closure under the operation.%The set is closed under the operation, that is $gh$ is also an element.
%		\item Associativity under the operation.%The operation is associative, that is $g(hi) = ghi = (gh)i$.
%		\item Existence of a unique identity element.%There exists a unique \textbf{identity element} $e$ in $G$ that leaves every element fixed, that is $ge = g = eg$.
%		\item Existence of unique inverses.%For every element $g$ there exists a unique \textbf{inverse element} $g^-1$ such that $gg^{-1} = e = g^{-1}g$.
%		\textit{\item\label{axiom:G4} (If the group is commutative, it is called abelian.)}%(If every element of $G$ commutes, that is $gh = hg$, the group is said to be commutative, or \textbf{abelian}.)}
%\end{enumerate}
%\end{definition}
%
%Sometimes the group is denoted by $(G,\star)$ but usually the operation is dropped and the set $G$ alone represents the group, especially if the operation is inferred from context. The last axiom G5 is not generally true for all groups. 
%
%\begin{example}[Additive groups]
%The number sets $\Z$, $\Q$, $\R$ and $\C$ are all abelian group under ordinary addition. %Their identity element is $0$ and for a number $n$, its inverse is $-n$. 
%A vector space $V$ over a field $\K$ is an abelian group under vector addition. %The identity element is the zero vector $\vec{0}$ and the inverse of a vector $v \in V$ is $-v$.
%For an integer $n>0$, the set of residue classes of modulo $n$, denoted $\Z/n\Z$, is an group under addition modulo $n$.
%\end{example}
%
%\begin{example}[Multiplicative groups]
%The sets $\Q$, $\R$ and $\C$ without $0$ (denoted $\Q^{\times}$, $\R^{\times}$ and $\C^{\times}$) are  groups under ordinary multiplication. %The identity element is $1$ and the inverse of the number $n$ is $1/n = n^{-1}$.
%The general linear group (for an integer $n>0$, the set of invertible\footnote{Non-zero determinant} $n \times n$-matrices with entries from the field $\K$), denoted $\GL_n(K)$ is a group under matrix multiplication. 
%The set $(\Z/n\Z)^{\times}$ is a group under multiplication modulo $n$.
%\end{example}

%\subsection{Subgroups}
%
%\begin{definition}[Subgroup]
%A non-empty subset $H \subseteq G$ is called a subgroup of $G$ if it is closed under inverses. Then it is denoted by $H \leq G$.
%\end{definition}
%
%Every group $G$ has these subgroups:
%
%\begin{example}
%The group $G$ is a subgroup of itself. However, a proper subgroup $H \neq G$ is denoted by $H < G$. The singleton $\{e\}$ is called the trivial subgroup.
%\end{example}
%
%The examples from the previous section has the following subgroups.
%
%\begin{example}
%The ordinary number sets are related by $\C \subset \R \subset \Q \subset \Z$. The integers has subgroups $n\Z$ for an integer $n>0$. The special linear group with entries from a field $\K$, denoted by $\SL_n(\K)$ is a subgroup of $\GL_n(\K)$.
%\end{example}

%\begin{example}[Center]
%The center of a group $G$ is the set of elements of $G$ that commute and is denoted by $Z(G)$. Since $e$ is in $Z(G)$, it is a subgroup of $G$. If $G$ is abelian, $Z(G) = G$.	
%\end{example}
%
%\begin{example}[Centralizer and Normalizer]
%The centralizer of a subset $A \subseteq G$ is the set of elements that commutes with every element of $A$.
%The normalizer of a subset $A \subseteq G$ is the set of elements that leaves the set $A$ fixed  under conjugation. Clearly, both the centralizer and the normalizer are subgroups of $G$, and the centralizer is a subgroup of the normalizer.
%\end{example}

%\begin{definition}[Normal subgroup]
%A subgroup $N \leq G$ is called normal if it is invariant under conjugation by $G$. It is denoted by $N \trianglelefteq G$. A proper normal subgroup is denoted by $N \triangleleft G$. 
%\end{definition}
%
%\begin{example}
%Any group is a normal subgroup to itself. In an abelian group, every subgroup is normal. In $GL_n(\K)$, $\SL_n(\K)$ is normal.
%\end{example}
%
%\begin{definition}[Simple group]
%A group is called simple if there are no non-trivial and proper normal subgroups.
%\end{definition}
%
%\begin{example}
%The integers modulo $n$ is a simple group if $n$ is a prime. For example, $\Z/7\Z$ has only itself and $\{e\}$ as normal subgroups, and as a non-example the integers have the even integers as a non-trivial and proper normal subgroup.
%\end{example}

%\subsection{The symmetric group $\mathfrak{S}_n$}
%
%The set of all permutations of the numbers $\{1,2,3,\dots,n\}$ is called the symmetric group of degree $n$ and is denoted by\footnote{The strange symbol is a ``fraktur'' $S$.} $\mathfrak{S}_n$. Equivalently, it is the set of bijections from $\{1,2,3,\dots,n\}$ to itself. The order of $\mathfrak{S}_n$ is $n!$. 
%
%\begin{example}
%The elements of $\mathfrak{S}_3$ are 
%\[
%e,(12),(13),(23),(123),(132).
%\]
%There are $3!=6$ elements: the identity element, three transpositions\footnote{2-cycles are called transpositions.} and two 3-cycles.
%\end{example}
%
%
%\subsection{The alternating group $\mathfrak{A}_n$}
%
%The subset of $\mathfrak{S}_n$ consisting of even\footnote{A permutation is considered even if it can be decomposed to an even number of 2-cycles, or equivalently it is of odd length.} permutations is called the alternating group of degree $n$ and is denoted by $\mathfrak{A}_n$. The order of $\mathfrak{A}_n$ is $n!/2$. The alternating group is a normal subgroup of the symmetric group of same degree.
%
%\begin{example}
%The elements of $\mathfrak{A}_3$ are 
%\[
%e,(123),(132).
%\]
%There are $3!/2=3$ elements: the identity element, and two 3-cycles. It is isomorphic to $\Z/3\Z$.
%\end{example}
%
%\begin{example}
%The alternating group of degree 4, $\mathfrak{A}_4$ has a normal subgroup
%\[
%V =\{e,(12)(34),(13)(24),(14)(23)\}
%\]
%This group is called the Klein-four group and it is isomorphic to $\Z_2^2$.
%\end{example}
%
%\subsection{Dihedral groups $\D_{2n}$}
%
%The set of symmetries of a regular $n$-gon centered on the origin in the plane is called the dihedral group of order $2n$ and is denoted by $\D_{2n}$. It can be constructed via two generators $r$ and $s$ where $r$ is the rotation of the $n$-gon by $2\pi/n$ around the origin and $s$ is the reflection along the axis between the origin and an arbitrary ``first'' vertex, and the relations between them. Clearly, the order of $r$ is $n$ and $s$ is 2, so
%\[
%D_{2n} = \left\langle r,s \ \middle\vert \ r^n = s^2 = e, sr = rs^{-1} \right\rangle.
%\] 
%
%\begin{example}
%The symmetries of the triangel is
%\[
%D_6 = \{ e, (123), (132), (12), (23), (13) \},
%\]
%hence it is isomorphic to $\mathfrak{S}_3$.
%\end{example}

%\subsection{Quotient groups}
%
%The normal subgroups subgroups discussed above are important, because they permit the existence of quotient groups.
%
%\begin{example}
%The integers $\Z$ has a normal\footnote{Any subgroup of $\Z$ is normal since $\Z$ is abelian.} subgroup $n\Z$. The quotient group $\Z/n\Z$ are the integers modulo $n$.
%\end{example}
%
%\subsubsection{The first isomorphism theorem}
%
%Something something.
%
%\subsection{Group action}
%
%Somethnig something.
%
%\subsubsection{Action of a group on itself by multiplication}
%
%Something something.
%
%\subsubsection{Action of a group on itself by conjugation}
%
%Something something.
%
%\subsubsection{Conjugacy classes}
%
%Something something.

