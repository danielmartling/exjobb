\clearpage{\thispagestyle{empty}}
\section{Some introductory background topics}

\subsection{Group theory}



%This section on groups is mostly based on the treatment of \cite{DummitFoote}.
%
%\subsubsection{Introduction to Groups}
%
%\begin{definition}[Group]
%	A group is a set $G$ together with a binary operation\footnote{Examples of binary operators are ordinary addition or multiplication, addition $\mod n$ for some integer $n$, vector addition, composition of functions and matrix addition and multiplication.} $\star: G \times G \rightarrow G$ fulfilling:
%	\begin{enumerate}[i)]
%		\item Associativity over the binary operation. For any $a,b,c \in G$, $a \star (b \star c) = (a \star b) \star c = a \star b \star c$.
%		\item Existence of a unique identity element in $G$. There is an element $e \in G$ such that for any $a \in G$, $a \star e = a = e \star a$.
%		\item Existence of unique inverses for every element of $G$. For every $a \in G$, there is an element denoted $a^{-1} \in G$ such that $a \star a^{-1} = e = a^{-1} \star a$.
%	\end{enumerate}
%\end{definition}
%
%Usually the $\star$ symbol is dropped when it is clear from context what operation we are using. In an additive group, the operation is addition. Usually something that looks like zero\footnote{For example 0, the zero vector, the zero matrix...} is the additive identity element and an inverse of an element $a$ is usually denoted by $-a$.
%
%\begin{example}[Some additive groups]
%	Some examples of additive groups are $\Z$, $\Q$, $\R$ and $\C$; any vector space under vector addition and the integers mod $n$, denoted $\Z / n \Z$.
%\end{example}
%
%In a multiplicative group, the operation is usually some sort of multiplication or composition. Usually something that looks like one\footnote{For example 1, the unit matrix $\1_n = (\delta_{ij})_{n \times n}$, the identity permutation $\id = e$...} is the multiplicative identity element.
%
%\begin{example}[Some multiplicative groups]	
%	Some examples of multiplicative groups are $\Z^\times$, $\Q^\times$, $\R^\times$ and $\C^\times$ (these are the subsets of their additive counterpart where zero has been removed since it lacks a multiplicative inverse.); $(\Z / n \Z)^\times$; The set of all invertible square $n \times n$ matrices denoted by $\GL(\C)$; Difference between GL(V) and GL(C)?
%\end{example}
%
%In general, any two elements of a group do not commute, that is for $a,b \in G$, $ab \neq ba$. However the special case when every element is commutative is important.
%
%\begin{definition}[Abelian group]
%	A group $G$ in which every element commutes with every other element (that is $ab = ba$ for every $a,b \in G$) is called a commutative group, or an Abelian group.
%\end{definition}
%
%\subsubsection{Subgroups}
%
%\begin{definition}[Subgroup]
%	A non-empty subset $H$ of $G$ is called a subgroup of $G$ if it fulfills:
%	\begin{enumerate}[i)]
%		\item Closure under the operation of $G$. If $h_1, h_2 \in H$, then $h_1 h_2, h_2 h_1 \in H$.
%		\item Closure under inverses. If $h \in H$, then there is an element $h^{-1} \in H$.
%	\end{enumerate}
%	If $H$ is a subgroup $G$, we denote the relation by $H \leq G$. The trivial subgroup is the subset of $G$ only consisting of the identity element of $G$, that is $\{e\}$. A proper subgroup $H$ of $G$ is a proper subset of $G$ and is denoted by $H < G$.
%\end{definition}

%\begin{definition}[Centralizer]
%	The centralizer of an element $a$ of $G$ is the set of elements of $G$ that commutes with $a$, and is denoted by 
%	\[
%	\Cent_G(a) = \{ g \in G \mid ga = ag \}.
%	\]
%	It is non-empty since it contains the identity element.
%\end{definition}
%
%\begin{definition}[Stabilizer]
%	The stabilizer of an element $a$ of $G$ is the set of elements of $G$ that fixes $a$, and is denoted by 
%	\[
%	\Stab_G(a) = \{ g \in G \mid g \star a = a \}.
%	\]
%	It is non-empty since it contains the identity element.
%\end{definition}

\subsubsection{Some special and interesting groups}

%\begin{definition}[The dihedral group of order $2n$]
%	For every integer $n \geq 3$, the set of all symmetries of a regular planar $n$-gon centered on the origin is called the dihedral group of order $2n$ and is denoted by $\D_{2n}$. The vertices of the regular $n$-gon are labeled clockwise from 1 to $n$. The dihedral group is generated by the two kinds of operations permitted on this rigid $n$-gon: rotation and reflexion. We define relection $s$ about the symmetry line through vertex 1 and the origin. Rotation $r$ is defined by either moving every vertex $i$ to the vertex $i+1 \mod n$, or one step clock-wise or equivalently by rotating the $n$-gon $2 \pi /n$ radians. We see that the order\footnote{The order of an element $g \in G$ is the smallest possible positive integer $m$ such that $g^m = e$.} of $s$ is 2 and of $r$ is $n$.
%	
%	The dihedral group can be constructed using $s$ and $r$ as generators along with some information on how $s$ and $r$ relate:
%	$$ \D_{2n} = \left\langle r, s \mid r^n = s^2 = 1, rs = sr^{-1} \right\rangle.$$
%\end{definition}
%
%\begin{definition}[The symmetric group of degree $n$]
%	The set of bijections from the set $\{1, 2, \dots n\}$ to itself if denoted $S_n$ and is called the symmetric group of degree $n$. Clearly $S_n$ has order $n!$. The associated operation is cycle decomposition $\circ$. For two permutations $\sigma, \tau \in S_n$, their decomposition is $\sigma \circ \tau = \sigma \tau$.
%\end{definition}
%
%The elements of $S_n$ can be represented in some different ways, one is to decompose every permutation $\sigma \in S_n$ to a product of pairwise disjoint\footnote{Two cycles are disjoint if they have no numbers in common.} cycles. The congruence class an element of $S_n$ is determined by the shape of its cycle decomposition, for example if an element consist of a 2-cycle and a 3-cycle, it is in the same conjugacy class of every other such cycle. A representative of a congruence can be chosen arbitrarily, and is surrounded by square brackets, for example is $\left[ e \right]$ is the conjugacy class consisting of only $e$.
%
%
%\begin{example}[$S_3$ and $S_4$]
%	The elements of $S_3$ and $S_4$ are presented in the following tables.
%	
%	%			\begin{center}
%		%				\begin{tabular}{r c l }
%			%					                   & $S_3$ &                      \\
%			%					  $\left[e\right]$ & $=$   & $\{e\}$              \\
%			%					 $\left[12\right]$ & $=$   & $\{(12),(13),(23)\}$ \\
%			%					$\left[123\right]$ & $=$   & $\{(123),(132)\}$
%			%				\end{tabular}
%		%				\\
%		%				\begin{tabular}{r c l }
%			%					                        & $S_4$ &                                                       \\
%			%					       $\left[e\right]$ & $=$   & $\{e\}$                                               \\
%			%					      $\left[12\right]$ & $=$   & $\{(12),(13),(14),(23),(24),(34)\}$                   \\
			%					     $\left[123\right]$ & $=$   & $\{(123),(124),(132),(134),(142),(143),(234),(243)\}$ \\
			%					    $\left[1234\right]$ & $=$   & $\{(1234),(1243),(1324),(1342),(1423),(1432)\}$       \\
			%					$\left[(12)(34)\right]$ & $=$   & $\{(12)(34),(13)(24),(14)(23)\}$
			%				\end{tabular}
		%			\end{center}
%\end{example}

%A subgroup of $S_n$ is the subset of even\footnote{A permutation is called even(odd) if its order is even(odd) permutations.} permutations. This is called the alternating group. For the examples above,

%\begin{example}[$A_3$ and $A_4$]
%	The elements of $S_3$ and $S_4$ are presented in the following tables.
%	
	%			\begin{center}
		%				\begin{tabular}{r c l }
			%					                   & $A_3$ &                   \\
			%					  $\left[e\right]$ & $=$   & $\{e\}$           \\
			%					$\left[123\right]$ & $=$   & $\{(123),(132)\}$
			%				\end{tabular}
		%				\\
		%				\begin{tabular}{r c l }
			%					                        & $A_4$ &                                                       \\
			%					       $\left[e\right]$ & $=$   & $\{e\}$                                               \\
			%					     $\left[123\right]$ & $=$   & $\{(123),(124),(132),(134),(142),(143),(234),(243)\}$ \\
			%					$\left[(12)(34)\right]$ & $=$   & $\{(12)(34),(13)(24),(14)(23)\}$
			%				\end{tabular}
		%			\end{center}			
%\end{example}

%Clearly, $A_n$ is a subgroup of $S_n$. $S_4/S_3 \cong A_3$???

%\subsubsection{Quotient groups}
%
%Quotient group???
%
%\subsubsection{Group action}
%
%Group action
%Groups acting on themselves - conjugation - class equation
%
%Automorphisms
%
%\subsection{Rings and fields}
%
%\subsection{Vector spaces}
