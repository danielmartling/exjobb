\clearpage{\thispagestyle{empty}}
\section{Character Theory}

Recall from linear algebra that two similar matrices have the same trace \cite[Thm 5.4.]{Holst} and that two matrices $X$ and $Y$ are similar if there exists a matrix $S$ such that $S X S^{-1} = Y$. Also recall that the trace is the sum along the diagonal of the matrix, as well as the sum of the eigen values of the matrix.

Something on change of basis - trace is basis independent.

The trace seem related to the conjugation of elements of $\GL(V)$, motivating our use of it as a ``metric/measure'' to compare different representations. 

\begin{definition}\label{def:char}
	 Let $V$ be a representation space of a group $G$ of dimension $n$ and let $\rho_g = (r_{ij})$ be the matrix representation of a $g \in G$. Then the \emph{character} of $g$ in $V$ is defined to be the trace of the matrix representation, that is 
	 \begin{align*}
	 	\chi_V(g) := \Tr \rho_g = \sum_{i=1}^{n} r_{ij}.
	 \end{align*}
\end{definition}

Whenever $V$ or $\rho$ are contextually obvious they may be omitted, ie. $\chi = \Tr g$. We have already inadvertently studied the characters of some representations:

\begin{example}[Characters of degree 1 representations]
	Evident from Definition~\ref{def:char}, the character of a degree 1 representation is the same as the representation, since a $1 \times 1$-matrix is usually identified with its only element, ie. $\chi_g = \rho_g$, hence the Tables~\ref{table:Cyc3},~\ref{tbl:cyc4} and \ref{table:Cyc5} are the character tables of $\Cyc_3$, $\Cyc_4$ and $\Cyc_5$ respectively.
	
	Further more, the character of the trivial representation of a group is 1 for every element of the group and the character of the alternating representation of the symmetric group is $+1$ for an even permutation and $-1$ for an odd one.
\end{example}

\begin{example}[Character of the identity element]
	By Equation~\ref{eq:rhoIdAndInv}, the identity element of any group is mapped to the identity transformation of the representation $V$, which is the identity matrix, hence $\chi(e) = \dim V$.
\end{example}

\begin{example}[Character of the permutation representation of $\Sym_n$]
	The matrix representation of a $\sigma \in \Sym_n$ has a 1 in the diagonal if and only if that row corresponds to a fixed point, hence the character of the permutation representation is the same as the number of fixed points of that conjugacy class.
\end{example}
Something on class functions

Something on basis, matrix representation

Projection formulae

character tables 

Sometging on the regular representation.

\begin{example}[Character table of $\Sym_2$]
	content...
\end{example}

\begin{example}[Character table of $\Sym_3$]
	content...
\end{example}

\begin{example}[Character table of $\Sym_4$]
	content...
\end{example}