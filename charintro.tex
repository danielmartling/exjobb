\clearpage{\thispagestyle{empty}}
\section{Character Theory}

%We have so far composed representations with the direct sum and tensor product of other representations, we have decomposed representations into subrepresentations and eventually into irreducible representations.

This section is based on \cite[Ch.2.]{Serre}, as well as \cite[Sect.2.2.]{FultonHarris}.

Let $V$ be a vector space with basis $(\bas_i)_{i=1}^n$ and let $\rho: G \rightarrow \GL(V)$ be a representation. Now, for each $g \in G$, we define a function $\chi: G \rightarrow \CC$ to be the trace of the matrix of $g$ in $V$. In other words $\chi(g) = \Tr(\rho_g)$. This function is called the \emph{character} of a representation. Why is it useful when studying representations?

\subsection{Basic properties of characters}

\begin{proposition}\cite[Prop.2.1.]{Serre}\label{prop:charidinvconj}
	The trace function \emph{characterizes} the representation in some relevant ways:
	\begin{enumerate}
		\item[i)] The character of the identity element of $G$ is the degree of the representation, $\chi(e) = \dim V$.
		\item[ii)] The character of an element in $G$ is the complex conjugate of the character of the inverse element, $\chi(g^{-1}) = \overline{\chi(g)}$.
		\item[iii)] The character is constant under conjugation, if $g,h\in G$ are conjugate it implies $\chi(g) = \chi(h)$.
	\end{enumerate}
\end{proposition}
\begin{proof}
	\begin{enumerate}
		\item[i)] By Proposition~\ref{prop:homoidinv}, $\rho_e$ is the $n \times n$ identity matrix, where $n = \dim V$, hence $\chi(e) = \Tr \id = n = \dim V$.
		\item[ii)] We are free to choose an orthonormal basis, then $\rho_g$ is a unitary matrix with roots of unity $\{\lambda_i\}$ as eigenvalues \cite[Exercise.8.6.15.]{Nicholson}. The character of $g^{-1}$ is then
		\begin{align*}
			\chi(g^{-1}) &= \Tr \rho_{g^{-1}} & \text{(Def. of character)} \\
			&= \Tr \rho_g^{-1} & \text{(Prop.\ref{prop:homoidinv}.)} \\
			&= \sum_i \frac{1}{\lambda_i} & \text{(Eigenvalues of unitary matrix)} \\
			&= \sum_i \overline{\lambda_i} & \text{(Reciprocal of root of unity)} \\
			&= \Tr \overline{\rho_g} & \text{(Trace is sum of eigenvalues)} \\
			&= \overline{\chi(g)}. & \text{(Def. of character)}
		\end{align*}
%		Recall that ${\rho_g}^{-1} = \rho_{g^{-1}}$. We are free to choose a orthonormal basis, then $\rho_g$ is a unitary matrix with roots of unity $\{\lambda_i\}$ as eigenvalues  (CITATION), then the inverse of $\rho_g$ has the eigenvalues $\{1/\lambda_i\}$. The reciprocal of a root of unity is its complex conjugate, hence $\left\lbrace\overline{\lambda_i}\right\rbrace$ are the eigenvalues of $\rho_{g^{-1}}$ and the character of $g^{-1}$ is 
%		\begin{align*}
%			\chi(g^{-1}) = \Tr \rho_{g^{-1}} = \Tr \rho_g^{-1} = \sum_i \frac{1}{\lambda_i} = \sum_i \overline{\lambda_i} = \Tr \overline{\rho_g} = \overline{\chi(g)}.
%		\end{align*} 
		\item[iii)] It is known that the trace is conserved under conjugation (see Prop.\ref{prop:trace}.).\qedhere
	\end{enumerate}
\end{proof}

Now we properly introduce the character.

\begin{definition}\label{def:char}
	Let $V$ be a representation space of a group $G$ of dimension $n$ and let $\rho_g = (r_{ij})$ be the matrix representation of a $g \in G$. Then the \emph{character} of $g$ in $V$ is defined to be the trace of the matrix representation, that is 
	\begin{align*}
		\chi_V(g) := \Tr \rho_g = \sum_{i=1}^{n} r_{ii}.
	\end{align*}
\end{definition}


\begin{notation}[Group character]
	A ``character vector'', simply called the (group) character, of $G$ can be defined as the tuple containing the character of every element of $G$, ie. $\chi = (\chi(g))_{g \in G}$. If the group $G$ is partitioned into the conjugacy classes $[k_1], [k_2], \dots, [k_l]$, then as an ink-saving measure the group character can be abbreviated to contain one representative $k_i$ from every class $[k_i]$, ie. $\chi= (\chi(k_i))_{i=1}^l$, since the character is fixed under conjugation.%, any elements of a group that are conjugate have the same character, ie. for a conjugate pair $g,g' \in G$ there exists an $h \in G$ such that $g' = hgh^{-1}$, and their characters are
%	\begin{align*}%
%		\chi(g') = \chi(h%gh^{-1}) = \chi(g).
%	\end{align*}%
\end{notation}

\begin{example}[Degree 1 representations]
	The  trace of a $1 \times 1$ matrix is of course its only element, hence the character of any representations of degree 1 is the representation itself.
	For example, the trivial character of any group is $(1, \dots, 1)$. 
\end{example}

\begin{example}[Character of a permutation matrix]\label{example:charperm}
	The character is the sum along the diagonal of a matrix, which in a permutation matrix (eg. the permutation and regular representations) correseponds to the fixed points of the group action. For example the element $(1,2) \in \Sym_3$ has one fixed point ($3$) and thus has the character $1$ in the permutation representation. 
\end{example}

\begin{notation}[Character table]
	A character table is an array of characters of a group.  Every column represents a conjugacy class and every row a representation. The trivial representation is placed in the first row. The sizes of every conjugacy class is placed under each class. An example is shown in Table~\ref{table:chartableexample}.
%	\begin{table}[hbt!]
%		\centering
%		\begin{tabular}{c | c c c }
%			$G$    & $\cdots $ & $[g]$        & $\cdots$ \\ 
%			$\Ss|[g]|$ & $\cdots$ & $\Ss|[g]|$ & $\cdots$ \\ \hline
%			Triv.   & $\cdots$  & 1            & $\cdots$ \\
%			$\vdots$ &           & $\vdots$     &          \\
%			$V$    & $\cdots$  & $\chi_V(g) $ & $\cdots$ \\
%			$\vdots$ &           & $\vdots$     &
%		\end{tabular}
%		\caption{Layout of a character table of a group $G$.}
%		\label{table:chartableexample}
%	\end{table}
\end{notation}

\begin{example}[$\Sym_3$ so far]
	So far, we have described the trivial, alternating, permutation and standard representations of $\Sym_3$ (Examples~\ref{example:trivrepr},~\ref{example:altrepr},~\ref{ex:permS3},~\ref{example:stdSym3}). %~and \ref{example:regSym3}) . 
	These representations are presented in Table~\ref{table:chartableSym3}.
\end{example}
%\begin{table}[hbt!]
%	\centering
%	\begin{tabular}{c | c c c }
%		$\Sym_3$  & $[(1)]$ & $[(1,2)]$ & $[(1,2,3)]$ \\ 
%		$\Ss|[\sigma]|$ & $\Ss1$ & $\Ss3$ & $\Ss2$ \\ \hline
%		\Triv. & $1$     & 1         & $1$       \\
%		\Alt.  & 1       & $-1$      & 1         \\
%		\Perm. & $3$     & $1$       & $0$       \\
%		\Stan. & 2       & 0         & -1        
%		%	Reg.  & 4       & 0         & 0
%	\end{tabular}
%	\vfill
%	\caption{Character table of $\Sym_3$.}
%	\label{table:chartableSym3}
%\end{table}

\begin{table}[hbt!]
	\centering
	\parbox[t]{.45\linewidth}{
		\centering
		\begin{tabular}{c | c c c }
			$G$    & $\cdots $ & $[g]$        & $\cdots$ \\ 
			$\Ss|[g]|$ & $\cdots$ & $\Ss|[g]|$ & $\cdots$ \\ \hline
			Triv.   & $\cdots$  & 1            & $\cdots$ \\
			$\vdots$ &           & $\vdots$     &          \\
			$V$    & $\cdots$  & $\chi_V(g) $ & $\cdots$ \\
			$\vdots$ &           & $\vdots$     &
		\end{tabular}
		\caption{Layout of a character table of a group $G$.}
		\label{table:chartableexample}
	}
	\hfill
	\parbox[t]{.45\linewidth}{
		\centering
		\begin{tabular}{c | c c c }
			$\Sym_3$  & $[(1)]$ & $[(1,2)]$ & $[(1,2,3)]$ \\ 
			$\Ss|[\sigma]|$ & $\Ss1$ & $\Ss3$ & $\Ss2$ \\ \hline
			\Triv. & $1$     & 1         & $1$       \\
			\Alt.  & 1       & $-1$      & 1         \\
			\Perm. & $3$     & $1$       & $0$       \\
			\Stan. & 2       & 0         & -1        
			%	Reg.  & 4       & 0         & 0
		\end{tabular}
		\caption{Character table of $\Sym_3$.}
		\label{table:chartableSym3}
	}
\end{table}







\subsection{Characters and tensor operations}

Addition and multiplication of two characters $\chi$ and $\psi$ are defined component-wise thusly:
\begin{align*}
	\chi + \psi &=  (\chi(g) + \psi(g))_{g \in G}, \text{ and} \\
	\chi \cdot \psi &= (\chi(g) \cdot \psi(g))_{g \in G}.
\end{align*}

To apply character theory on the tensor toolbox presented in Section~\ref{sect:tensorrepr}, we propose the following:

\begin{proposition}\label{prop:charplustimes}
	Let $V$ and $W$ be representation spaces of a group $G$ and let $\chi_V$ and $\chi_W$ be its characters in those representations. Then we propose:
	\begin{enumerate}
		\item[i)] The character in $V \oplus W$ is $\chi_V+\chi_W$.
		\item[ii)] The character in $V \otimes W$ is $\chi_V \cdot \chi_W$.
%		\item[iii)] The characters of the symmetric and exterior squares are \begin{align*}
%			\chi_{\SymP^2 V}(g) = \frac{1}{2}(\chi_V(g)^2 + \chi_V(g^2)) \quad \text{and} \quad \chi_{\AltP^2 V}(g) = \frac{1}{2}(\chi_V(g)^2 - \chi_V(g^2)),
%		\end{align*} compatible with $V \otimes V = \SymP^2 V \oplus \AltP^2 V$.
	\end{enumerate}
\end{proposition}
\begin{proof}
	Statement i) and ii) are consequences of Section~\ref{sect:tensoralgebra}. 
	\begin{enumerate}
		\item[i)] The trace of $\rho_g^V \oplus \rho_g^W$ is clearly the sum of the traces of $\rho_g^V$ and $\rho_g^W$.
		\item[ii)] Likewise, the trace of $\rho_g^V \otimes \rho_g^W$ is found to be the product of the traces of $\rho_g^V$ and $\rho_g^W$.\qedhere
%		\item[iii)] \textit{To be written...} \qedhere
	\end{enumerate}
\end{proof}

\begin{example}[$\Sym_3$ again]
	The permutation representation was earlier found to be the direct sum of the trivial and the standard representations, in fact
	\begin{align*}
		\chi_\Triv + \chi_\Stan = (1,1,1) + (2,0,-1) = (3,1,0) = \chi_\Perm.
	\end{align*}
\end{example}

The character of the direct sum or tensor product of an arbitrary number of representations are defined similarly.

\begin{example}[$\Sym_4$ so far]
	For $\Sym_4$, we know the trivial representation and the alternating representation. 
	
	Using Example~\ref{example:charperm}, we can find the permutation character $(4,2,1,0,0)$ by counting the number of fixed points of every class, equivalently we count the number of ones in the integer partitions of Table~\ref{table:elementsSym4}. 
	
	We may also calculate the character of the standard representation by subtracting the trivial character from the permutation character. These representations are presented in Table~\ref{table:chartableSym4}.
\end{example}

\begin{table}[hbt!]
	\centering
	\begin{tabular}{c | c c c c c}
		$\Sym_4$   & $[(1)]$ & $[(1,2)]$ & $[(1,2,3)]$ & $[(1,2,3,4)]$ & $[(1,2)(3,4)]$ \\
		$\Ss|[\sigma]|$ & $\Ss1$       & $\Ss6$         & $\Ss3$           & $\Ss6$             & $\Ss8$              \\ \hline
		\Triv.    & 1       & 1         & 1           & 1             & 1              \\
		\Alt.     & 1       & -1        & 1           & -1            & 1              \\
		\Stan.    & 3       & 1         & 0           & -1            & -1             \\
		\Perm.    & 4       & 2         & 1           & 0             & 0              \\ %$\chi_R$     & 24      & 0       & 0       & 0        & 0
		%$\chi_R$   & 24      & 0         & 0           & 0             & 0
	\end{tabular}
	\caption{Character table of $\Sym_4$.}
	\label{table:chartableSym4}
\end{table}

\begin{example}[Tensor power of alternating representation]\label{example:nthpoweralternating}
	The character of the alternating representation of a symmetric group is $\chi_\Alt = (1, -1, 1, -1, \dots, 1)$, hence the $n$th tensor power of the alternating representation has the character
	\begin{align*}
		\chi_\Alt^n &= (1, (-1)^n), 1, (-1)^n, \dots, 1) \\
		&= \begin{cases}
			\chi_\Triv \text{, if $n$ is even, and } \\
			\chi_\Alt \text{, if $n$ is odd.}
		\end{cases}
	\end{align*}
	That is, an even tensor power is isomorphic to the trivial representation and an odd power is isomorphic to the alternating representation.
\end{example}

\subsection{Orthogonality relations of characters}

This section is based on \cite[Sect.2.2.]{FultonHarris}.

Continuing the discussion of Section~\ref{sect:irredreprs}, we want find characters of irreducible representations. We call those irreducible characters. Let $V$ be an arbitrary representation and let $W$ be an irreducible representation of some group $G$. 

We denote by $\Hom (V,W)$ the set of all homomorphisms from $V$ to $W$, and by the superscript $(\cdot) ^G$ we denote a subset which is fixed by $G$, for example 
\begin{align}\label{eq:defVG}
	V^G = \left\lbrace \vvec \in V \ \middle\vert \ g\cdot \vvec = \vvec, \ \forall g \in G \right\rbrace
\end{align} 
is the subspace of $V$ in which every element is fixed by all of $G$. Also,  $\Hom(V,W)^G$ is the set of all $G$-linear maps $V \rightarrow W$, and the set of linearly independent $G$-linear maps for a basis for $\Hom(V,W)^G$.

\begin{proposition}
	The multiplicity of $W$ in $V$ is $\dim \Hom(W,V)^G$, that is the number of (linearly independent) $G$-linear maps from $W$ to $V$. Conversely, if $W$ is arbitrary and $V$ is irreducible, then $\dim \Hom (V,W)^G$ is the multiplicity of $V$ in $W$. If both $V$ and $W$ are irreducible, then by Schur's Lemma we must have that
	\begin{align}\label{eq:dimhomVWG}
		\dim \Hom (V,W)^G = \begin{cases}
			1, \text{ if $V$ and $W$ are isomorphic, and} \\
			0, \text{ else.}
		\end{cases}
	\end{align}
\end{proposition}

Now, let $V^G$ be defined as the fixed points of $V$ under the action of $G$, as defined in Equation~\ref{eq:defVG}. A map from $V^G$ to itself is called an \textit{endomorphism of $V^G$} and we set $\End(V^G) := \Hom(V^G, V^G)$. Such a map is not just a representation of $G$, but also a trivial representation by definition, then the number of all such maps, denoted $\dim \End(V^G)$ is the multiplicity of the trivial representation in $V$. 

The map associated with any $g$ is not generally a $G$-linear map, however as we have previously seen, we can construct such a map by averaging over $G$, that is we define a map $\varphi: V \rightarrow V$ by
\begin{align*}
	\varphi = \frac{1}{|G|} \sum_{g \in G} g,
\end{align*}
which is $G$-linear since for some $h \in G$, \begin{align*}
	h \cdot \varphi(h^{-1} \cdot \vvec) &= \frac{1}{|G|} \sum_{g \in G} (hgh^{-1}) \cdot \vvec \\
	&= \varphi(\vvec)
\end{align*} for any $\vvec \in V$, also it is projection of $V$ onto $V^G$ since the image of $\varphi$ is $V^G$ and clearly it then also is an endomorphism of $V^G$. 

Now, the eigenvalues of a projection are 1 for every eigenvector in the image and 0 for every eigenvector in the kernel, hence the trace of the projection $\varphi$ is the dimension of $V^G$, that is
%Now, the multiplicity of the trivial representation in $V$ is the trace of $\varphi$ (since the eigenvalues of a projection are 1 for every basis vector in $V^G$ and zero for every basis vector in $V \setminus V^G$),  which is
\begin{align*}
	\dim V^G = \Tr \varphi &= \Tr \left( \frac{1}{|G|} \sum_{g \in G} g \right) \\
	&= \frac{1}{|G|} \sum_{g \in G} \Tr g & \text{(Trace is linear operator)} \\
	&= \frac{1}{|G|} \sum_{g \in G} \chi_{V}(g) & \text{(Def. of character)},
\end{align*}
hence \begin{align}\label{eq:multVG}
	\dim V^G = \frac{1}{|G|} \sum_g \chi_V(g).
\end{align}

Now, let's apply Equation~\ref{eq:multVG} to the set of all maps $V \rightarrow W$, ie. $V$ becomes $\Hom (V,W)$, where both $V$ and $W$ are irreducible representations of $G$. Earlier in Section~\ref{sect:tensoralgebra}, $\Hom(V,W)$ was identified with $V^* \otimes W$, hence the character of $\Hom(V,W)$ is $\overline{\chi_V(g)}\cdot\chi_W(g)$, then by Equations~\ref{eq:dimhomVWG} and~\ref{eq:multVG} we have:
\begin{align*}
	\dim \Hom(V,W)^G = \frac{1}{|G|} \sum_g \overline{\chi_V(g)}\cdot\chi_W(g) = \begin{cases}
		1, \text{ if $V \cong W$, and} \\
		0, \text{ if $V \neq W$.}
	\end{cases}
\end{align*}

We have arrived at an expression that looks suspiciously familiar, interpreting characters as complex-valued vectors in some vector space, we have found an \textit{inner product of characters}.

\begin{definition}[Inner product of characters]
	Let $\varphi$ and $\psi$ be the group characters of a group $G$, then we define the inner product of characters as
	\begin{align*}
		(\varphi | \psi) := \frac{1}{|G|} \sum_{g \in G} \overline{\varphi(g)} {\psi(g)}.
	\end{align*}
\end{definition}

It is an inner product since it satisfies the expected properties:
\begin{enumerate}
	\item[i)] We have 
	\begin{align*}
		\overline{(\varphi|\psi)} &= \frac{1}{|G|} \sum_{g \in G} \overline{\overline{\varphi(g)} {\psi(g)}} & \text{(Def. inner product)}\\
		&= \frac{1}{|G|} \sum_{g \in G} \varphi(g) \overline{\psi(g)} & \text{($\overline{\overline{a}}=a$ for all $a \in \CC$)}\\
		&= \frac{1}{|G|} \sum_{g \in G} \overline{\psi(g)} {\varphi(g)} & \text{(Scalars commute)}\\
		&= (\psi|\varphi). &\text{(Def. inner product)}
	\end{align*}
	\item[ii)] For $a,b \in \CC$ we have 
	\begin{align*}
		(\varphi | a\psi_1 + b\psi_2) &= \frac{1}{|G|} \sum_{g \in G} \overline{\varphi(g)} \left({a\psi_1(g) + b\psi_2(g)}\right) & \text{(Def.)}\\
		&= a\frac{1}{|G|} \sum_{g \in G} \overline{\varphi(g)} {\psi_1(g)}  +b \frac{1}{|G|} \sum_{g \in G} \overline{\varphi(g)} {\psi_2(g)}  \\
		&= a(\varphi | \psi_1 )  +b (\varphi |\psi_2) & \text{(Def.)}
	\end{align*}
	for any characters $\varphi, \psi_1$ and $\psi_2$, hence it is linear in $\psi$.
	\item[iii)] Lastly, we have 
	\begin{align*}
		(\varphi|\varphi) &= \frac{1}{|G|} \sum_{g \in G} \overline{\varphi(g)} {\varphi(g)} \\
		&= \frac{1}{|G|} \sum_{g \in G} |\varphi(g)|^2 & \text{(Def. modulus)}
	\end{align*}
	for any character $\varphi \neq \0$, hence $(\psi|\psi) > 0$ and $(\psi|\psi) \in \RR$ since every summand is positive and real.
\end{enumerate}
%This is an inner product since it is linear in $\varphi$, semilinear in $\psi$, as well is $(\varphi|\varphi) > 0 $ for any $\varphi \neq \0$ and $(\varphi|\varphi)=0$ if and only if $\varphi= \0$. PROVE THESE PROPERTIES.

%\begin{notation}
%	The inner product of a character with itself is sometimes lazily denoted by \begin{align*}
%		|\varphi| := (\varphi|\varphi).
%	\end{align*}
%\end{notation}

We have also shown the following property of irreducible representations:

\begin{theorem}[Irreducibility criterion]\label{thm:irredcrit}
	The characters of irreducible representations are orthonormal. In other words, let $\chi \neq \psi$ be characters of irreducible representations of $G$ (called irreducible characters), then we have that 
	\begin{align*}
		\text{i) }& (\chi|\chi)= 1 \text{, and}  \\
		\text{ii) }& (\chi|\psi) = 0.
	\end{align*}  
\end{theorem}

In other words, irreducible characters create an orthonormal system, with these irreducibles as a basis.

Now, let $V$ be the direct sum of irreducible representations $W_i$ of $G$ such that
\begin{align*}
	V = \bigoplus_i W_i.
\end{align*}
If $\chi_i$ is the character of $W_i$, then by Proposition~\ref{prop:charplustimes} the character of $V$ is $\varphi = \sum_i \chi_i$.
Let $\chi$ be the character of an irreducible representation of $G$, then we have that
\begin{align*}
	(\varphi|\chi) = \sum_i (\chi_i|\chi).
\end{align*}
Since $\chi$ and all of the $\chi_i$ are irreducible characters, all of the inner products $(\chi_i|\chi)$ are either 1 or 0, depending on if $\chi$ and $\chi_i$ are of isomorphic representations. Hence, $(\varphi|\chi)$ will return the multiplicity of $\chi$ in $\varphi$. 

\begin{remark}
	This also means that the composition of a representation $\varphi$ into a direct sum of irreducible subrepresentations $\chi_i$ is unique up to isomorphism, since two isomorphic compositions would have the same decomposition, ie. be constructed the same way of the same irreducibles.
\end{remark}

\begin{remark}
	The converse is also true, if two representations have the same character, then they are isomorphic since they contain the same irreducible representations.
\end{remark}

\begin{note}
	Let $V$ be a representation with the composition 
	\begin{align*}
		V = \bigoplus_{i=1}^k m_i W_i, % = m_1W_1 \oplus m_2W_2 \oplus \dots \oplus m_k W_k,
	\end{align*}
	where the $m_i$ are the multiplicities of the irreducible $W_i$. If the character of $W_i$ is denoted by $\chi_i$, then the character of $V$ is
	\begin{align*}
		\varphi = \sum_{i=1}^k m_i \chi_i,
	\end{align*}
	where $m_i = (\varphi|\chi_i)$.	Taking the inner product of $\varphi$ with itself we have
	\begin{align}\label{eq:innerprodself}
		(\varphi|\varphi) = \sum_{i=1}^k m_i^2.
	\end{align}
\end{note}

\begin{theorem}
	The character $\varphi$ is irreducible if and only if $(\varphi|\varphi) = 1$.
\end{theorem}
\begin{proof}
	If $\varphi$ is irreducible, then by Theorem~\ref{thm:irredcrit} we have that $(\varphi|\varphi) = 1$. For the converse statement, if the sum in Equation~\ref{eq:innerprodself} is equal to one, we must have that only one of the $m_i$ is equal to 1, and the rest are equal to 0. Then $V$ is composed of only \textit{one} irreducible character, hence it is irreducible.
\end{proof}

\begin{example}[$\Sym_3$]
	The standard representation of $\Sym_3$ is irreducible since
	\begin{align*}
		(\chi_\Stan\vert\chi_\Stan)  &= \frac{1}{6}(2^2 + 3\cdot 0^2 + 2 \cdot (-1)^2)  \\
		&= 1.
	\end{align*}
	hence the standard representation of $\Sym_3$ is irreducible. The trivial and the standard representations are inequivalent since
	\begin{align*}
		(\chi_\Triv | \chi_\Stan) = \frac{1}{6}(2 + 3 \cdot 0 -2) = 0.
	\end{align*}
	We already know that the permutation representation is not irreducible.  This is verified by 
	\begin{align*}
		(\chi_\Perm | \chi_\Perm) = \frac{1}{6} ( 9 + 3 \cdot 1 + 0) = 2.
	\end{align*}
\end{example}

\begin{example}[$\Sym_4$]
	Likewise, the standard character of $\Sym_4$, $\chi^\Stan = (3,1,0,-1,-1)$, is also irreducible since 
	\begin{align*}
		(\chi_\Stan \vert \chi_\Stan) &= \frac{1}{24}(3^2 + 3\cdot 0^2 + 2 \cdot (-1)^2)  \\
		&= 1.
	\end{align*}
\end{example}
%\textit{Something on the characters are basis of class function space, hence the number of irreducible representations of a group is at most the number of conjugacy classes in the group.
%Using these criteria and relations as a character calculus toolbox, we can study some groups, determine all of their irreducible representations and decompose any arbitrary representation.
\subsection{Decomposition of the regular representation}

Earlier, the regular representation of any group where found have the trivial representation as a subrepresentation, likewise, the alternating group was found in the regular representation of $\Sym_n$. Now, we will completely decompose the regular representation of any group.

For a $g \in G$, it will act on a basis vector $\bas_h$ of the regular representation space $V$ by $g \cdot \bas_{h} = \bas_{gh}$ and the resulting matrix $\rho_g$ can then be constructed by studying the action of this $g \in G$ on every $\bas_h$, where $h \in G$. However, the trace of this matrix, ie. the character $\chi_\Reg(g)$, only depends on the values on the diagonal, which corresponds to the fixed points under the action of $g$. What this means is that the ``$h$th'' column will have an 1 in the ``$h$th'' row if and only if $gh = h$, which holds only if $g = e$ since only $e$ leaves every other element of $G$ fixed, hence $\rho_g$ is a $|G| \times |G|$ permuation matrix with trace
\begin{align}\label{eq:regcharacter}
	\chi_{\Reg}(g) = \begin{cases}
		|G| \text{, if $g = e$,} \\
		0 \text{, otherwise,}
	\end{cases}
\end{align}
and the character of the regular representation is then
\begin{align*}
	\chi_{\Reg} = (|G|, 0, \dots, 0)
\end{align*}
For any group $G$. Now taking the inner product of it with itself we have,
\begin{align*}
	(\chi_{\Reg}|\chi_{\Reg}) &= \frac{1}{|G|} \sum_{g \in G} \overline{\chi^{\Reg}(g)}\chi^{\Reg}(g) \\
	&=  \frac{1}{|G|}  \overline{\chi^\Reg}\chi^\Reg & \text{(All vanish except $g=e$.)} \\
	&= \frac{1}{|G|} \left(\dim V \right)^2  & \text{(Prop.~\ref{prop:charidinvconj}).}\\
	&= \frac{1}{|G|}|G|^2 & \text{(From def of $V$.)} \\
	&= |G|.
\end{align*}
Applying Equation~\ref{eq:innerprodself}, we have that $(\chi^\Reg|\chi^\Reg) $ is the square sum of the multiplicities of every irreducible subrepresentation of the regular representation, hence we have:
\begin{align}\label{eq:sqsummult}
	\sum_i m_i^2 = |G|.
\end{align}
If $\{W_i\}$ is the family of all irreducible representations of a group with respective characters $\{\chi_i\}$, then the regular representation $V$ is a direct sum of these with (not-necessarily non-zero) multiplicities $\{m_i\}$. Then we have that the multiplicity of some irreducible $W_j$ with character $\chi_j$ in $\chi^\Reg$ is
{\allowdisplaybreaks\begin{align*}
	m_j &= (\chi_\Reg|\chi_j) \\ &=  \sum_i m_i  (\chi_i|\chi_j)   & \text{(Reg. is sum of $m_i\chi_i$)}\\
	&= \frac{1}{|G|} \sum_i  m_i \sum_{g \in G} \overline{\chi_i(g) }{\chi_j(g)}. & \text{(Def. of inner product.)} \\
	&=  \frac{1}{|G|}\sum_i  m_i \overline{\chi_i(e) }{\chi_j(e)} & \text{($\chi_i$ vanish for all $g \neq e$.)} \\
	&= \frac{1}{|G|} \sum_i m_i \dim W_i \dim W_j & \text{(Character of $e$ is degree of repr.)} \\
	&= \dim W_j \frac{1}{|G|}  \sum_i m_i \dim W_i  & \\
		&= \dim W_j \frac{1}{|G|} \dim V & \text{($V = \sum_i m_i W_i$)} \\
	&= \dim W_j \frac{1}{|G|}  |G| & \text{(From def. of regular repr.)} \\
	& = \dim W_j.
\end{align*}}

We have shown the following:
\begin{theorem}\label{thm:regreprmult}
	Every irreducible representation $W_i$ of a group appears exactly $\dim W_i$ times in the regular representation $V$ of the group. In other words, the regular representation is decomposed as
	\begin{align*}
		V = \bigoplus_i \dim (W_i) W_i.
	\end{align*}
\end{theorem}

Returning to Equation~\ref{eq:sqsummult}, we have the following:
\begin{corollary}\label{cor:sqsumirreds}
	The square sum of the degrees of every irreducible representations of a group is the order of the group.
\end{corollary}

\begin{remark}
	By Theorem~\ref{thm:regreprmult} and Corollary~$\ref{cor:sqsumirreds}$, along with the orthogonality relations and irreducibility criterion, we can use character theory to find \textbf{every} irreducible representation of a group. 
\end{remark}

For example, to ensure we have found an irreducible representations of a group, we take the inner product of its character with itself, expecting 1 if it is irreducible. To ensure we have found every irreducible representation, we calculate the square sum of the degrees of those we found so far. If it does not add up to the order of the group, then we can conjecture possible degrees of the missing irreducibles.

%Recall from linear algebra that two similar matrices have the same trace \cite[Thm 5.4.]{Holst} and that two matrices $X$ and $Y$ are similar if there exists an invertible matrix $S$ such that $S X S^{-1} = Y$. Also recall that the trace is the sum along the diagonal of the matrix, as well as the sum of the eigen values of the matrix.
%
%Something on change of basis - trace is basis independent.
%
%The trace seem related to the conjugation of elements of $\GL(V)$, motivating our use of it as a ``metric/measure'' to compare different representations. 
%
%%\begin{definition}\label{def:char}
%%	 Let $V$ be a representation space of a group $G$ of dimension $n$ and let $\rho_g = (r_{ij})$ be the matrix representation of a $g \in G$. Then the \emph{character} of $g$ in $V$ is defined to be the trace of the matrix representation, that is 
%%	 \begin{align*}
%%	 	\chi_V(g) := \Tr \rho_g = \sum_{i=1}^{n} r_{ii}.
%%	 \end{align*}
%%\end{definition}
%
%Whenever $V$ or $\rho$ are contextually obvious they may be omitted, ie. $\chi = \Tr g$. We have already inadvertently studied the characters of some representations:
%
%\begin{example}[Characters of degree 1 representations]
%	Evident from Definition~\ref{def:char}, the character of a degree 1 representation is the same as the representation, since a $1 \times 1$-matrix is usually identified with its only element, ie. $\chi_g = \rho_g$, hence the Tables~\ref{table:Cyc3},~\ref{tbl:cyc4} and \ref{table:Cyc5} are the character tables of $\Cyc_3$, $\Cyc_4$ and $\Cyc_5$ respectively.
%	
%	Further more, the character of the trivial representation of a group is 1 for every element of the group and the character of the alternating representation of the symmetric group is $+1$ for an even permutation and $-1$ for an odd one.
%\end{example}
%
%\begin{example}[Character of the identity element]
%	By Equation~\ref{eq:rhoIdAndInv}, the identity element of any group is mapped to the identity transformation of the representation $V$, which is the identity matrix, hence $\chi(e) = \dim V$.
%\end{example}
%
%\begin{example}[Character of the permutation representation of $\Sym_n$]
%	The matrix representation of a $\sigma \in \Sym_n$ has a 1 in the diagonal if and only if that row corresponds to a fixed point, hence the character of the permutation representation is the same as the number of fixed points of that conjugacy class.
%\end{example}
%Something on class functions
%
%Something on basis, matrix representation
%
%Projection formulae
%
%character tables 
%
%Sometging on the regular representation.