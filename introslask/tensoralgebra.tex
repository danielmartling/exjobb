%\clearpage{\thispagestyle{empty}}
\subsection{Tensor algebra}

	
%	
%	\begin{definition}[Tensor]
%		A $(r,s)$-tensor $T$ is a multi-linear function
%		\[
%			T: \underset{r \text{ times}}{\underbrace{V \times \dots \times V}} \times \underset{s \text{ times}}{\underbrace{V^* \times \dots \times V^*}} \longrightarrow \C.
%		\]
%		Or in other words, it takes $r$ vectors from a vector space and $s$ vectors from the corresponding dual vector space $V^*$ and maps it to a number in $\C$.
%	\end{definition}
%	
%	\begin{example}[Tensors]
%		\begin{itemize}
%			\itemsep0em 
%			\item A $(0,0)$-tensor is per definition a scalar.
%			\item A $(1,0)$-tensor is a vector.
%			\item A $(0,1)$-tensor is a dual vector.
%			\item An example of a $(3,0)$-tensor is the Levi-Civita tensor defined by 
%			\[
%				\epsilon(\mathbf{u}, \mathbf{v}, \mathbf{w}) = (\mathbf{u} \cdot \mathbf{v}) \times \mathbf{w},
%			\]
%			which could be thought of as a measurement of the volume of the parallelepiped spanned by the vectors $\mathbf{u},\mathbf{v}$ and $\mathbf{w}$. When applied on any basis vectors $\mathbf{e}_i, \mathbf{e}_j$ and $\mathbf{e}_k$ of a vector space, we find the Levi-Civita symbol
%			\[
%				\epsilon(\mathbf{e}_i, \mathbf{e}_j, \mathbf{e}_k) := \epsilon_{ijk} = \begin{cases}
%					1, \text{ if $i,j,k$ cyclical} \\
%					-1, \text{ if $i,j,k$ anti-cyclical} \\
%					0, \text{ if $i,j,k$ non-distinct.}
%					\end{cases}.
%			\]
%		\end{itemize}
%	\end{example}
	
%	\begin{definition}[Direct sum of vector spaces]
%		content...
%	\end{definition}
%	
%	\begin{definition}[Tensor product of vector spaces]\label{def:tensorproduct}
%		Let $V$ and $W$ be two vector spaces over the field $\K$. Their \textbf{tensor product} is also a vector space over $\K$ and is denoted by $V \otimes W$. Its elements will look like $\mathbf{v} \otimes \mathbf{w}$ if $\mathbf{v}$ is from $V$ and $\mathbf{w}$ is from $W$. 
%	\end{definition}
%	
%	\begin{definition}[Linear combination of vector spaces]
%		content...
%	\end{definition}
%	
%	\begin{definition}[Tensor power]
%		If $V$ is a vector space over $\K$, then for a positive but not necessarily non-zero integer $n$, the $n$th tensor power of $V$, denoted by $V^{\otimes n}$ is also a vector space, by induction on definition \ref{def:tensorproduct}.
%		\begin{notation}
%			The zeroth power of a vector space is defined as its ground field, that is $V^{\otimes 0} = \K$, and the first power is the vector space itself, that is $V^{\otimes 1} = V$.
%		\end{notation}
%	\end{definition}
	
%	\begin{example}[$\R^3 \otimes \R^3 = {\R^3}^{\otimes 2}$]
%		Choosing a basis $\{\ebas_1, \ebas_2, \ebas_3\}$ for $\R^3$, a basis of $\R^3 \otimes \R^3$ is 
%		\begin{table}[hbt!]\centering\begin{tabular}{r c c c l}
%			$\mathcal{B} = \{\{ \ebas_i \otimes \ebas_j \}_{i=1}^3\}_{j=1}^3 = \{$ & $\ebas_1 \otimes \ebas_1$, & $\ebas_1 \otimes \ebas_2$, & $\ebas_1 \otimes \ebas_3$, &  \\
%			     & $\ebas_2 \otimes \ebas_1$, & $\ebas_2 \otimes \ebas_2$, & $\ebas_2 \otimes \ebas_3$, &  \\
%			     & $\ebas_3 \otimes \ebas_1$, & $\ebas_3 \otimes \ebas_2$, & $\ebas_3 \otimes \ebas_3$ & $\}$.
%		\end{tabular}\end{table}
%	\end{example}