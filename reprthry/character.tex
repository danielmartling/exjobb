\clearpage{\thispagestyle{empty}}
\section{Character Theory}

	For a representation $\rho$ of a group $G$ in a vector space $V$, the character of the representation is defined as the trace of the matrix representation in $\GL(V)$ of a element $g$ of $G$. If the matrix representation is given by
	\begin{align}
		X(g) = \left( x_{ij} \right)_{n \times n}
	\end{align}
	then the character is
	\begin{align}
		\chi_V(g) = \Tr X(g) = \sum_{i=1}^{n} x_{ii}.
	\end{align}
	Evident from this definition, the character of a degree 1 representation is the same as the representation, since a $1 \times 1$-matrix is usually identified with its only element.
	
	The character of the entire group $G$ in a representation $V$ is a vector 
	\begin{align}
		\chi_V = \left( \chi_V(g) \right)_{g \in G},
	\end{align}
	sometimes abbreviated to just containing a character for every conjugacy class
	\begin{align}
		\chi_V = \left( \chi_V([g]) \right)_{[g] \in G},
	\end{align}
	since ``character'' is a class function on the elements of $G$.
	
	Since the trace of a matrix is invariant under conjugation, the character