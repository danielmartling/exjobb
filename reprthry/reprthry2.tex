%	\clearpage{\thispagestyle{empty}}
%\section{Representation Theory 2}

%The layout of this section is based on \cite{Serre}.

%Let $V$ be a finite vector space over $\C$ with dimension $\dim V = n$. The group of automorphisms of $V$ is denoted $\GL(V)$ and any element $f$ therefrom is a linear map $f: V \rightarrow V$. If a basis is provided for $V$ then $\GL(V)$ is isomorphic to $\GL_n(\C)$, the set of invertible $n \times n$-matrices, and then $f$ is such a matrix, where again $n = \dim V$\cite[18.1]{DummitFoote}.

%Let $G$ be a finite group of degree $|G| = n$ written multiplicatively. 

%\begin{definition}[Representation]
%	A representation of a finite group $G$ in the vector space $V$ is a homomorphism 
%	\[
%	\rho: G \rightarrow \GL(V)
%	\]
%	that assign every element of $G$ to an element of $\GL(V)$. If $V$ has a basis, there is for every $g$ in $G$ a matrix representation
%	\[
%		\mathbf{X}(g) = (x_{ij})_{n \times n}.
%	\]
%	\begin{note}
%		As a shorthand, $V$ itself is called the representation.
%	\end{note}
%\end{definition}

%As a consequence of the homomorphism of $\rho$, $\rho(e) = \1_n$ (the identity matrix) for the identity element $e$ of $G$ and $\rho(gh) = \rho(g)\rho(h)$ and $\rho(g)^{-1} = \rho(g^{-1})$ for any $g$ and $h$ in $G$.

%The \textbf{degree of the representation} is the same as the dimension of the vector space.

%\subsection{Tensor operations on representations}
%	
%	To-do here:
%	\begin{enumerate}
%		\setlength\itemsep{-1em}
%		\item direct sums of representations
%		\item definition of degree of representation
%		\item definition of subrepresentation
%	\end{enumerate}
%	
%\subsection{Irreducibility, Schur's and Maschke's}
%	
%	\begin{theorem}[Maschke's theorem]\label{thm:maschke}
%		Let $G$ be a finite group and let $V$ be any arbitrary representation of $G$, then $V$ is composed of a number of not necessarily distinct irreducible subrepresentations $W_i$ of $V$, or in other words,
%		\begin{align}
%			V = \bigoplus_i W_i.
%		\end{align}
%	\end{theorem}
%	
%	\begin{note}
%		The direct sum can be expressed in terms of \underline{distinct} subrepresentations $W_i$ of $V$ and their multiplicities $a_i$ as
%		\begin{align}
%			V = \bigoplus_i a_i W_i,
%		\end{align}
%		where $a_i W_i$ is the direct sum of $a_i$ copies of $W_i$.
%	\end{note}
%	
%	To prove Maschke's theorem, first we establish a few definitions and lemmas.
%	
%	\begin{corollary}
%		If the representation is of degree 1, $V = W_1$ and $\dim V = 1$, it is clearly irreducible.
%	\end{corollary}
%	
%	
%	
%	\begin{proof}[Proof of Maschke's theorem \ref{thm:maschke}]
%		The theorem will be proved using induction on complementary subspaces. If $V$ itself is irreducible, we are done. If $V$ is not irreducible, then there is a non-trivial proper subrepresentation $W$ of $V$ with a basis $\{\ebas_i\}_i$, then according to lemma \ref{lemma:compsubspace}, there is a complementary ``orthogonal'' subrepresentation $W^\perp$ such that $V = W \oplus W^\perp$. If both of $W$ and $W^\perp$ are irreducible, then we are done. If either are not, then after recursion on $W$ and $W^\perp$ and since $V$ is a finite vector space, we are done.
%	\end{proof}

%\subsection{Some naïve examples}

%In this section, $G$ is a finite group of order $|G| = n$. 

%\paragraph*{The trivial and signature representations.}

%	For any $G$, and choosing $V = \R$, a map
%	\[
%		\rho: G \rightarrow \GL(\R)
%	\]
%	defined by $ \rho(g) = 1$ for every $g$ in $G$, in other words mapping every element in $G$ to the trivial element of $V = \R$. It is trivially a homomorphism since $\rho(e)=1$ and both $\rho(gh)=1$ and $\rho(g)\rho(h)=1$ and is called the \textbf{trivial representation}. It is a representation of degree 1.
%	
%	%Letting $\mathbf{X}(g) = 1$ for every $g$ in $G$, we have a trivial representation of degree 1. This is trivially a homomorphism and thus a representation.
%	
%	Any group has a trivial degree 1 representation, but there are many trivial representations of larger degree, for example choosing $V = \R^3$, we define the representation by $\rho(g) = \1_3$, the three-by-three identity matrix. However, this is not an irreducible representation as $\R^3 = \R \oplus \R \oplus \R = 3\R$ and
%	\[
%		\begin{pmatrix} 1&0&0\\0&1&0\\0&0&1 \end{pmatrix} = (1)\oplus(1)\oplus(1).
%	\]	
	
%	The symmetric group has another degree 1 representation. Every element $\sigma$ of the symmetric group of degree $n$, denoted $S_n$, has a sign, or parity, denoted $\sgn(\sigma)$ defined by the number of transpositions\footnote{2-cycles} required to construct it. Letting $k$ be that number, then the parity or sign of $\sigma$ is $\sgn(\sigma)=(-1)^k$, so an \textbf{even} permutation has sign $+1$ and an \textbf{odd} permutation has sign $-1$. Now, with $G = S_n$ and choosing $V=\F_2$, this $\sgn$ function
%	\[
%		\sgn: S_n \rightarrow \{-1, 1\}
%	\]
%	defined as above is also a homomorphism since $\rho(e) = 1$ and for two arbitrary elements $\sigma$ and $\tau$ of $S_n$ with respective sign $(-1)^s$ and $(-1)^t$ we have 
%	\[
%		\rho(\sigma)\rho(\tau) = (-1)^s(-1)^t = (-1)^{st} = \rho(\sigma\tau).
%	\] 
%	This $\sgn$ function is called the \textbf{alternating representation}\footnote{In some literature called the sign or signature representation.} of the symmetric group.
	
%	\paragraph{Permutation and regular representation}
	
%	Let the group $G$ act on an appropriately chosen set $A$ by
%	\[
%		\phi : G \times A \rightarrow A
%	\] % could be replaced by by ``permutating its elements''
%	defined by $\phi: g \mapsto ga$. After constructing a vector space\marginnote{What is the dim?} $V$ with a basis $\mathcal{B} = ( \ebas_a )_{a \in A}$, we can define a representation
%	\[
%		\rho: G \rightarrow \GL(V)
%	\]
%	defined by
%	\[
%		\rho(g)(\ebas_a) = \ebas_{ga}.
%	\]
%	This is a homomorphism since for any $a$ in $A$, $\rho(e)$ acts as a identity map by 
%	\[
%		\rho(e)(\ebas_a) = \ebas_{ea} = \ebas_a
%	\]
%	and also for any $g$ and $h$ in $G$, 
%	\[
%		\rho(g)\rho(h)(\ebas_a) = \rho(g)(\ebas_{ha}) = \ebas_{gha} = \rho(gh)(\ebas_a).
%	\]
%	This homomorphism is called the \textbf{permutation representation}.
	
%	\begin{example}[Some examples]
%		Here are some suggestions for permutation representations
%		\begin{itemize}
%			\item $S_n$ can act on $\{1,2,\ \dots, n\}$, then $V$ is $\R^n$ and for a $\sigma$ in $S_n$, $\rho(\sigma)$ permutes the $i$th coordinate axis tho the $\sigma(i)$th axis.
%			\item $Z_n$ (the cyclic group of size $n$) can act on the $n$th roots of unity (finite subset of $S_1$) by rotation???
%			\item $\D_{2n}$ (the dihedral group of degree $n$, size $2n$) can act on the vertices of a regular $n$-gon (ostensibly the set $\{1, 2, \ \dots, n\}$)(I SUPPOSE??? HOW???)
%		\end{itemize}
%	\end{example}
%	
%	Instead of any general set, we can let $G$ act on itself. Similarly as above, we construct a vector space with basis $( \ebas_g )_{g \in G}$ and define a representation
%	\[
%		\rho: G \rightarrow \GL(V)
%	\]
%	defined by
%	\[
%		\rho(g)(\ebas_h) = \ebas_{gh}.
%	\]
%	To show that it is a homomorphism, one could follow the same steps as above. This is called the \textbf{regular representation}.
%	
%\paragraph{The regular representation}
%
%In some sense, we can let $G$ act on itself by multiplication. Let $G$ be a finite group of order $n$ and $V$ be a vector space with basis $\left( \ebas_g \right)_{g \in G}$ defined by\footnote{Bad notation! $\ebas_1$ is the basis vector associated with the identity element $e$ of $G$.} $\ebas_1 = e$ and $\ebas_g = \rho(g)(\ebas_1)$. 
%
%\paragraph{The permutation representation}
%
%We can let $G$ act on any arbitrary set $A$, then,
%\[
%	\rho: G \times A \rightarrow A 
%\]
%defined by 
%\[
%	\rho(g): a \mapsto ga
%\]
%for any $g$ in $G$ and $a$ in $A$. Then $V$ is a vector space with basis $\left( \ebas_a \right)_{a \in A}$ where $\rho(g)(\ebas_a) = \ebas_{ga}$.
%
%It is ``natural'' to let $S_n$ act on $\{1, 2, \dots, n\}$.
%
%\begin{example}[Representations of $S_n$.]
%	Letting $\mathbf{T}(\sigma) = 1$ for every $\sigma$ in $S_n$, we have a trivial representation $T$ of $S_n$.
%	
%	Since $S_n$ is a group of permutations where every element has a sign property, the sign function
%	\[
%		\sgn(\sigma) = (-1)^k
%	\]
%	where $k$ is the least number of disjoint transpositions needed to construct $\sigma$, is an appropriate representation since $\sgn(\sigma)\sgn(\tau) = \sgn(\sigma\tau)$ for every $\sigma$ and $\tau$ in $S_n$.
%	
%	
%	
%	\begin{table}[hbt!]
%		\centering
%		\begin{tabular}{c | c c c c c c }
%			$S_3$  & $e$ & $(12)$ & $(13)$ & $(23)$ & $(123)$ & $(132)$ \\ \hline
%			 $T$   & 1   & 1      & 1      & 1      & 1       & 1       \\
%			$\sgn$ & 1   & -1     & -1     & -1     & 1       & 1       \\
%			 $R$   &     &        &        &        &         &         \\
%			 $P$   &     &        &        &        &         &
%		\end{tabular}
%	\end{table}
%	
%	
%\end{example}