\subsection{Trivial representation}

	Any group can be represented by a trivial representation of degree 1 defined by 
	\begin{align}
		\rho(g) = 1, \text{ for every } g \in G.
	\end{align}
	It is clearly a homomorphism\footnote{Here I did not state the vector space, should I? ``A vector space over the trivial group/field	''.}. 
	
	For any $G$, and choosing $V = \R$, a map
	\[
	\rho: G \rightarrow \GL(\R)
	\]
	defined by $ \rho(g) = 1$ for every $g$ in $G$, in other words mapping every element in $G$ to the trivial element of $V = \R$. It is trivially a homomorphism since $\rho(e)=1$ and both $\rho(gh)=1$ and $\rho(g)\rho(h)=1$ and is called the \textbf{trivial representation}. It is a representation of degree 1.
	
	%Letting $\mathbf{X}(g) = 1$ for every $g$ in $G$, we have a trivial representation of degree 1. This is trivially a homomorphism and thus a representation.
	
	Any group has a trivial degree 1 representation, but there are many trivial representations of larger degree, for example choosing $V = \R^3$, we define the representation by $\rho(g) = \1_3$, the three-by-three identity matrix. However, this is not an irreducible representation as $\R^3 = \R \oplus \R \oplus \R = 3\R$ and
	\[
	\begin{pmatrix} 1&0&0\\0&1&0\\0&0&1 \end{pmatrix} = (1)\oplus(1)\oplus(1).
	\]
	
	For the symmetric group $S_n$, the \textbf{trivial representation} is the one-dimensional representation that maps every permutation to 1.
	\[
	\rho(\sigma) = \1, \ \forall \sigma \in S_n
	\]
	It is a representation because in particular it maps $e$ to 1 and for any $\sigma$, $\tau$ in $S_n$,
	\[
	\rho(\sigma \circ \tau) = 1 = 1 \cdot 1 = \rho(\sigma) \cdot \rho(\tau)
	\]
	it is a homomorphism.

\subsection{Alternating representation of $S_n$}

	Choosing $G$ to be the symmetric group of degree $n$, denoted $S_n$, we can find another degree 1 representation by studying the sign, or parity, of the elements of $S_n$. For any two permutations $\sigma$ and $\tau$ of $S_n$ with respective sign $(-1)^s$ and $(-1)^t$, the sign of their composition is 
	\begin{align}
		\sgn(\sigma \circ \tau) = (-1)^s \cdot (-1)^t = (-1)^{st} = \sgn(\sigma) \cdot \sgn(\tau),
	\end{align}
	so clearly this sign function is a homomorphism from $S_n$ to a one-dimensional vector space over the two elements\footnote{The vector space is over the field of two elements, $\F_2$.} $+1$ and $-1$, hence the $\sgn$ function is representation of degree 1 and is called the alternating representation, or alternatively the sign or signature representation.
	
	The symmetric group has another degree 1 representation. Every element $\sigma$ of the symmetric group of degree $n$, denoted $S_n$, has a sign, or parity, denoted $\sgn(\sigma)$ defined by the number of transpositions\footnote{2-cycles} required to construct it. Letting $k$ be that number, then the parity or sign of $\sigma$ is $\sgn(\sigma)=(-1)^k$, so an \textbf{even} permutation has sign $+1$ and an \textbf{odd} permutation has sign $-1$. Now, with $G = S_n$ and choosing $V=\F_2$, this $\sgn$ function
	\[
	\sgn: S_n \rightarrow \{-1, 1\}
	\]
	defined as above is also a homomorphism since $\rho(e) = 1$ and for two arbitrary elements $\sigma$ and $\tau$ of $S_n$ with respective sign $(-1)^s$ and $(-1)^t$ we have 
	\[
	\rho(\sigma)\rho(\tau) = (-1)^s(-1)^t = (-1)^{st} = \rho(\sigma\tau).
	\] 
	This $\sgn$ function is called the \textbf{alternating representation}\footnote{In some literature called the sign or signature representation.} of the symmetric group.
	
	Recall that all of $S_n$ can be generated by its subset of adjacent transpositions $\{(12), (23), \dots, (n,1)\}$, so any permutation in $S_n$ can be written as a product of transpositions. If the number of transpositions required to compose the permutation $\sigma$ is $k$, then the sign of $\sigma$ is defined as
	\[
	\sgn(\sigma) = (-1)^k.
	\]
	So for an even(odd) $k$, the permutation is called even(odd). The $\sgn$ function arises naturally from the symmetric group and is called the \textbf{alternating\footnote{It is sometimes called the sign or signature representation} representation}. It is a representation because it maps $e$ to 1 and a composition $\sigma \circ \tau$ has the sign $\sgn(\sigma\tau) =\sgn(\sigma)\sgn(\tau)$, hence it is a homomorphism.
	
	

\subsection{Permutation representation}

	We could let the group $G$ act on an appropriately chosen set $X$, and study its action $f$ by permutation of the elements of $X$. This action is defined by any $g$ in $G$ and $x$ in $X$ as
	\begin{table}[hbt!]\centering\begin{tabular}{c c c c}
			$f:$ & $G \times X$    & $\rightarrow$ & $X$ \\
			&\rotatebox[origin=c]{90}{$\in$}&&\rotatebox[origin=c]{90}{$\in$} \\
			$f:$ & $(g,x)$ & $\mapsto$     & $gx$.
	\end{tabular}\end{table}
	
	Further more, define a vector space $V$ spanned by the orthogonal basis $\mathcal{B} = \{\ehat_x\}_x$ for every $x$ in $X$. Then we have a homomorphism
	\begin{align}
		\rho: G \rightarrow \GL(V)
	\end{align}
	defined by its action on the basis $\mathcal{B}$ by, for any $g$ in $G$,
	\begin{align}
		\rho(g): \ehat_x \mapsto g\ehat_x = \ehat_{gx}.
	\end{align}
	or on an arbitrary vector $v$ in $V$ as
	\begin{align}
		\rho(g): \vvec \mapsto g\vvec.
	\end{align}
	
	\textit{Show that it is a homomorphism!}
	
	\subsubsection{$S_n$ acting on $\N_n$}
	The action is defined as $S_n \times \N_n$ by $(\sigma,i) \mapsto \sigma(i)$. Choosing a basis $\mathcal{B} = \{\ehat_i\}_{i=1}^n$ we span a vector space $V$ of dimension $\dim V = n$. We then define a homomorphism $\rho$ from $S_n$ to $\GL(V)$ by
	\begin{align}
		\rho(\sigma): \ehat_i \mapsto \ehat_{\sigma(i)},
	\end{align}
	that is by permutating a basis vector $\ehat_i$ to another one $\ehat_j = \ehat_{\sigma(i)}$.
	
	\textit{action of $\rho(\sigma)$ on any vector??}
	
	The corresponding matrix representations are the permutation matrices
	\begin{align}
		X(\sigma) = (x_{ij})_{n \times n}, \text{ where } x_{ij} = \delta_{j,\sigma(i)} = \begin{cases}
			1 \text{ if } j = \sigma(i),\\
			0 \text{ otherwise.}
		\end{cases}
	\end{align}
	
	\begin{example}[$S_n$ for $n = 2,3$]
		
		The symmetric group of degree 2 has two elements, $S_2 = \{(1), (12)\}$ and their matrix representations in $\mathbf{\ehat_1},\mathbf{\ehat_2}$-space are
		\begin{table}[hbt!]
			\centering
			\caption{Matrix representations of $S_2$}
			\begin{tabular}{c | c c}
				$\sigma$    & $(1)$ & $(12)$ \\
				\hline
				$X(\sigma)$ & $\begin{pmatrix}
					1 & 0 \\ 0 & 1
				\end{pmatrix}$ & $\begin{pmatrix}
					0 & 1 \\ 1 & 0
				\end{pmatrix}$
			\end{tabular}
		\end{table}
		
		Likewise, representations of $S_3$ follows:
		
		\begin{table}[hbt!]
			\centering
			\caption{Matrix representations of $S_3$}
			\begin{tabular}{c | c c c c c c}
				$\sigma$  &$(1)$& $(12)$&$(13)$&$(23)$&$(123)$&$(132)$ \\
				\hline
				$X(\sigma)$ & 
				$\left(\begin{smallmatrix}
					1 & 0 & 0 \\
					0 & 1 & 0 \\
					0 & 0 & 1
				\end{smallmatrix}\right)$ &
				$\left(\begin{smallmatrix}
					0 & 1 & 0 \\
					1 & 0 & 0 \\
					0 & 0 & 1
				\end{smallmatrix}\right)$ &
				$\left(\begin{smallmatrix}
					0 & 0 & 1 \\
					0 & 1 & 0 \\
					1 & 0 & 0
				\end{smallmatrix}\right)$ &
				$\left(\begin{smallmatrix}
					1 & 0 & 0 \\
					0 & 0 & 1 \\
					0 & 1 & 0
				\end{smallmatrix}\right)$ &
				$\left(\begin{smallmatrix}
					0 & 0 & 1 \\
					1 & 0 & 0 \\
					0 & 1 & 0
				\end{smallmatrix}\right)$ &
				$\left(\begin{smallmatrix}
					0 & 1 & 0 \\
					0 & 0 & 1 \\
					1 & 0 & 0
				\end{smallmatrix}\right)$
			\end{tabular}
		\end{table}
		
		%			$$S_3 = \{(1), (12), (13), (23), (123), (132)\}$$
		%			$$S_4 = \{(1), 
		%						(12), (13), (14), (23), (24), (34), 
		%						(12)(34), (13)(24), (14)(23), 
		%						(123), (124), (134), (132), (142), (143), (234), (243),
		%						(1234), (1243), (1324), (1342), (1423), (1432)
		%						\}$$
	\end{example}
	
	Let the group $G$ act on an appropriately chosen set $A$ by
	\[
	\phi : G \times A \rightarrow A
	\] % could be replaced by by ``permutating its elements''
	defined by $\phi: g \mapsto ga$. After constructing a vector space\marginnote{What is the dim?} $V$ with a basis $\mathcal{B} = ( \ebas_a )_{a \in A}$, we can define a representation
	\[
	\rho: G \rightarrow \GL(V)
	\]
	defined by
	\[
	\rho(g)(\ebas_a) = \ebas_{ga}.
	\]
	This is a homomorphism since for any $a$ in $A$, $\rho(e)$ acts as a identity map by 
	\[
	\rho(e)(\ebas_a) = \ebas_{ea} = \ebas_a
	\]
	and also for any $g$ and $h$ in $G$, 
	\[
	\rho(g)\rho(h)(\ebas_a) = \rho(g)(\ebas_{ha}) = \ebas_{gha} = \rho(gh)(\ebas_a).
	\]
	This homomorphism is called the \textbf{permutation representation}.
	
	Instead of letting $S_n$ act naturally on the set $\{1, 2, \dots, n\}$, let $S_n$ act on the coordinates of points in $\C^n$ by permuting the basis vectors $(\hat{e}_i)_{i=1}^n$ by $\sigma(\hat{e}_i) = \hat{e}_{\sigma(i)}$, or in other words, $\sigma$ permutes the $i$th to the $\sigma(i)$th axis. Then we have a representation
	\[
	\rho: S_n \longrightarrow \GL_n
	\]
	that is defined by 
	\[
	\rho(\sigma) = \left(\delta_{i, \sigma(i)}\right)_{n \times n},
	\]
	where $\delta_{ij}$ is the Kronecker delta. In other words, every $\rho(\sigma)$ is the unit matrix where the $i$th row is moved to the $\sigma(i)$th row. It is a representation since it maps $e$ to $\1_{n  \times n}$ and it is a homomorphism by matrix multiplication.
	
	\begin{example}[Action of $S_3$ on coordinates in $\R^3$.]
		Letting $S_3$ act on coordinates in space by permuting the axes we obtain:
		\begin{table}[hbt!]
			\centering
			\begin{tabular}{r r r}
				$\rho(e) = 
				\begin{pmatrix}
					1 & 0 & 0 \\
					0 & 1 & 0 \\
					0 & 0 & 1
				\end{pmatrix}$ & 
				$\rho(123) = 
				\begin{pmatrix}
					0 & 0 & 1 \\
					1 & 0 & 0 \\
					0 & 1 & 0
				\end{pmatrix}$ & 
				$\rho(132) = 
				\begin{pmatrix}
					0 & 1 & 0 \\
					0 & 0 & 1 \\
					1 & 0 & 0
				\end{pmatrix}$ \\ & & \\
				$\rho(12) = 
				\begin{pmatrix}
					0 & 1 & 0 \\
					1 & 0 & 0 \\
					0 & 0 & 1
				\end{pmatrix}$ &
				$\rho(13) = 
				\begin{pmatrix}
					0 & 0 & 1 \\
					0 & 1 & 0 \\
					1 & 0 & 0
				\end{pmatrix}$ &
				$\rho(23) = 
				\begin{pmatrix}
					1 & 0 & 0 \\
					0 & 0 & 1 \\
					0 & 1 & 0
				\end{pmatrix}$
			\end{tabular}
		\end{table}
	\end{example}
	
	The question is, is this an irreducible representation? It is not since a representation is irreducible if it has no non-zero proper invariant subspace. Coordinates of points in $\R^3$ along the line from the origin through $(1,1,1)$ is invariant, consequently the subspace $$\left\{ (x,y,z) \in \R^3 \ \middle\vert \ x+y+z=0 \right\}$$ is $S_3$-invariant and is irreducible.

\subsection{Standard representation}

	\textit{The subspace found when studying $S_3$ - generalize to $S_n$.}

\subsection{Regular representation}

	Instead of letting the group $G$ act on some arbitrary set by permutation, we could let it act on itself. This action is defined as
	\[
	G \times G \rightarrow G
	\]
	which for every $g$ in $G$ is given by 
	\[
	(g,h) \mapsto gh,
	\]
	for a $h$ in $G$. The corresponding vector space $V$ is spanned by basis vectors $\{\ghat\}_{g \in G}$ constructed from the members of $G$. A representation of $G$ in $V$ is then
	\[
	\rho: G \rightarrow \GL(V)
	\]
	defined by
	\[
	\rho(g)(\hhat) = g\hhat.
	\]
	
	Instead of any general set, we can let $G$ act on itself. Similarly as above, we construct a vector space with basis $( \ebas_g )_{g \in G}$ and define a representation
	\[
	\rho: G \rightarrow \GL(V)
	\]
	defined by
	\[
	\rho(g)(\ebas_h) = \ebas_{gh}.
	\]
	To show that it is a homomorphism, one could follow the same steps as above. This is called the \textbf{regular representation}.
	
	\subsubsection{The regular representation of cyclic group $Z_n$}
	
		Recall that the cyclic group of order $n$ is generated by an element $g$ of order\footnote{The order of an element $g$ is the smallest possible integer $n$ such that $g^n = e$.} $n$ if $Z_n = \{e, g, g^2, \dots, g^{n-1} \}$. Sometimes we denote the cyclic group as $Z_n = \langle g \rangle$ if $g$ is such a generator. A cyclic group may have multiple generators that generate the group separately, for example the integers modulo 12, $\Z/12\Z \cong Z_{12}$ has 1, 5, 7 and 11 as generators. 
		
		Consider a matrix representation of $Z_n$ as a homomorphism
		\[
		\rho: Z_n \longrightarrow \C
		\]
		defined by $\rho(e) = 1$ and $\rho(g^a g^b) = \rho(g^a)\rho(g^b)$. Since $g^n = e$ and $\rho(g^n) = 1$, then $\rho(g)$ must be mapped to an $n$th root of unity. Denote by $\omega = e^{2\pi i/n}$ the first non-real $n$th root of unity. So for $Z_n$, there are $n$ one-dimensional representations $\rho_m: Z_n \rightarrow \C$ that maps $g^k$ to $\omega^{km}$.
		
		\begin{example}[$Z_3$]
			The three third roots of unity are $1$, $\omega = \frac{-1+i\sqrt{3}}{2}$ and $w^2 = \frac{-1-i\sqrt{3}}{2}$, so the three representations of $Z_3$ are presented in table \ref{table:Z3}.
			
			\begin{table}[hbt!]
				\centering
				\caption{Three representations of $Z_3$.}
				\begin{tabular}{c | c c c}
					\label{table:Z3}
					$Z_3$ & $e$ & $g$        & $g^2$      \\ \hline
					$\rho_0$          & 1   & 1          & 1          \\
					$\rho_1$          & 1   & $\omega$   & $\omega^2$ \\
					$\rho_2$          & 1   & $\omega^2$ & $\omega$
				\end{tabular}
			\end{table}
		\end{example}
		
		\begin{example}[$Z_5$]
			The five fifth roots of unity are $e^{\frac{2\pi im}{5}}$, so five representations of $Z_5$ are presented in table \ref{table:Z5}.
			
			\begin{table}[hbt!]
				\centering
				\caption{Five representations of $Z_5$. $\omega = e^{2 \pi i/5}$.}
				\begin{tabular}{c | c c c c c}
					\label{table:Z5}
					$Z_5$ & $e$ & $g$        & $g^2$      & $g^3$      & $g^4$      \\ \hline
					$\rho_0$            & 1   & 1          & 1          & 1          & 1          \\
					$\rho_1$            & 1   & $\omega^1$ & $\omega^2$ & $\omega^3$ & $\omega^4$ \\
					$\rho_2$            & 1   & $\omega^2$ & $\omega^4$ & $\omega^1$ & $\omega^3$ \\
					$\rho_3$            & 1   & $\omega^3$ & $\omega^1$ & $\omega^4$ & $\omega^2$ \\
					$\rho_4$            & 1   & $\omega^4$ & $\omega^3$ & $\omega^2$ & $\omega^1$
				\end{tabular}
			\end{table}
		\end{example}
		
			
			