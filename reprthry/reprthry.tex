\clearpage{\thispagestyle{empty}}
\section{Representation theory}

	\textit{Detta är syntesen av tre tidigare versioner. }
	
	The layout of this section is based on \cite{Serre} and \cite{FultonHarris}.
	
	From this point on, only finite groups and finite vector spaces are considered. 

	Let $G$ be a group written multiplicatively and let its order (or size or cardinality) be denoted by $|G|$.
	
	Let $V$ be a vector space over a field $\K$. Its group of automorphisms is denoted by $\GL(V)$. If $V$ has a dimension $\dim V = n$ and we provide a basis, usually denoted $\{\bas_i\}_{i=1}^n$, then $\GL(V)$ is identified with the set of linear transformations of $V$, which is the set of invertible square matrices of size $n \times n$, denoted $\GL_n(\K)$\cite[18.1]{DummitFoote}. The main purpose of this text is to study linear representations of finite groups, a process explained as:
	
	\textbf{\emph{A representation of a group in a vector space is a homomorphism that takes an element of the group into the group of automorphisms of the vector space.}}
	
	Or more directly in the following definition.
	
	\begin{definition}[Representation]
		Given a finite group $G$, a representation of $G$ in $V$ is a homomorphism
		\begin{align}
			\rho: G \rightarrow \GL(V).
		\end{align}
	\end{definition}
	For every $g$ in $G$, there is a transformation $\rho(g)$ in $\GL(V)$, or again $\rho(g): V \rightarrow V$ for some $g$ in $G$. In other words, $\rho(g)$ acts acts as a linear map on a vector $v$ in $V$ and maps it to the vector $\rho(g)(v)$ in $V$.
	
	Recall that if $\rho$ is a homomorphism, then for any two elements $g$ and $h$ in $G$,
	\begin{align}\label{eq:homomorphism}
		\rho(gh) = \rho(g)\rho(h),
	\end{align}
	and these two consequences follows:	
	\begin{proposition}
		For the identity element $e$ of $G$ and an arbitrary element $g$ with inverse $g^{-1}$ in $G$,
		\begin{align}
			\rho(e) = \id
		\end{align}
		and
		\begin{align}
			\rho(g)^{-1} = \rho(g^{-1}),
		\end{align}
		where $\id$ is the identity transformation and $\rho(g)^{-1}$ is the inverse representation of $g$.
	\end{proposition}
	\begin{proof}
		Take $g$ as an arbitrary element of $G$. The first identity follows from taking $h=e$ in eq.\ref{eq:homomorphism}:
		\begin{align}
			\rho(eg) = \rho(e)\rho(g). %	\rho(g) \overset{!}{=} \rho(eg) = \rho(e)\rho(g)
		\end{align}
		But $eg=g$, so $\rho(g) = \rho(e)\rho(g)$ which is true if and only if $\rho(e) = \id$. Instead taking $h=g^{-1}$:
		\begin{align}
			\rho(g) \rho(g^{-1}) =  \rho(gg^{-1}). %\overset{!}{=} \rho(e) = \id.
		\end{align}
		But $gg^{-1} = e$ and $\rho(e) = \id$, implying $\rho(g^{-1}) = \rho(g)^{-1}$.
	\end{proof}
	
	If the vector space $V$ is provided with a basis, then a matrix representations of $G$ may be found.
	\begin{definition}[Matrix representation]
		Given a finite group $G$, a matrix representation of $G$ in $V$ is a homomorphism
		\begin{align}
			X: G \rightarrow \GL_n(\K),
		\end{align}
		or equivalently, $X(g): V \rightarrow V$.
	\end{definition}
	
	\begin{note}
		Sometimes the vector space $V$ itself is said to be the representation, instead of the set of maps $\{\rho(g)\}_{g \in G}$.
	\end{note}
	
	\subsection{Subrepresentations and irreducibility}
	
		As laid out in \cite{FultonHarris}, we are looking for representations which are said to be ``atomic'' and conversely for any arbitrary representation, which of these indecomposable representation compose it? These concepts translate directly from linear algebra. 
		
		\begin{definition}[Degree of representation, subrepresentation, irreducible representation]
			Analogous to vector spaces, their vector subspaces and their dimensions are representations, subrepresentations and degrees. 		
			\begin{enumerate}
				\item The degree of a representation is said to be the dimension of its vector space.
				\item A subrepresentation $W$ of a representation $V$ of $G$ is a subspace $W$ of $V$ invariant under $G$, that is for every $\wvec$ in $W$, $\rho(g)(\wvec)$ is also in $W$ for every $g$ in $G$.
				\item A representation is irreducible if it has no proper non-zero invariant subspace.
			\end{enumerate}
		\end{definition}

		\begin{corollary}[Degree 1 representations are irreducible]
			If the representation is of degree 1, the vector space has no subspace, other than itself and the zero space, thus it is clearly irreducible.
		\end{corollary}
		
		
		
		Analogously to the existence of orthogonal subspaces in linear algebra, there are also complementary subrepresentations.
		
		\begin{proposition}\cite[Prop.1.5]{FultonHarris}\label{prop:compsubrepr}
			If $W$ is a subrepresentation of a representation $V$ of $G$, then there is a complementary $G$-invariant subspace $W^\perp$ of $V$ such that 
			\begin{align}
				V = W \oplus W^\perp.
			\end{align}
		\end{proposition}
		\begin{proof}
			\textit{Inner products... Sum...} See \ref{def:orthsubspace}? \textit{Definition: $G$-invariant???} Enough to reference orthogonal subspace? Are subspaces G-invariant automatically?
		\end{proof}
		
		The results of proposition \ref{prop:compsubrepr} invites the next theorem.
		
		\begin{theorem}[Maschke's theorem]\label{thm:maschke}
			Let $G$ be a finite group and let $V$ be a representation of $G$, then $V$ is composed of a number of not necessarily distinct irreducible subrepresentations $W_i$ of $V$, or in other words,
			\begin{align}
				V = \bigoplus_i W_i.
			\end{align}
			\begin{note}
				The direct sum can be expressed in terms of \underline{distinct} subrepresentations $W_i$ of $V$ and their multiplicities $a_i$ as
				\begin{align}
					V = \bigoplus_i a_i W_i,
				\end{align}
				where $a_i W_i$ is the direct sum of $a_i$ copies of $W_i$.
			\end{note}
		\end{theorem}
		\begin{proof}
			The theorem will be proved using induction on complementary subspaces, using proposition \ref{prop:compsubrepr}. If $V$ itself is irreducible, we are done. If $V$ is not irreducible, then there is a non-trivial proper subrepresentation $W$ of $V$ with a basis $\{\ebas_i\}_i$, then according to proposition \ref{prop:compsubrepr}, there is a complementary ``orthogonal'' subrepresentation $W^\perp$ such that $V = W \oplus W^\perp$. If both of $W$ and $W^\perp$ are irreducible, then we are done. If either are not, then after recursion on $W$ and $W^\perp$ and since $V$ is a finite vector space, we are done.
		\end{proof}
	
	\subsection{Tensor operations on representations}
	
	\textit{Not good}
	
	Direct sum, tensor product, tensor powers, symmetric and alternating powers, hom(V,W), dual
	
	Let $V$ and $W$ be two representations, then their direct sum $V \oplus W$, their tensor product $V \otimes W$ are representations.
	
	The $n$th tensor power of $V$, \[V^{\otimes n} = \underset{n \text{ times}}{\underbrace{V \otimes \cdots \otimes V}}\] is also a representation with the symmetric powers $\Sym^n V$ and the exterior powers $ \bigwedge^n V$ as subrepresentations. Further more, the dual space $V^* = \Hom(V,\C)$ of $V$ and more generally $\Hom(V,W) = V^* \otimes W$ are also representations. Recall that $\Hom(V,W)$ is the set of all linear functions (functionals??) from $V$ to $W$ and in particular $\Hom(V, \C)$ ``is some sort of scalar product''. Cf. quantum physics, if $v$ is a column vector from a ket-space $V$, then $u^\dagger$ is a row vector from the corresponding bra-space $V^*$.
	
	