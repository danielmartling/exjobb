\clearpage{\thispagestyle{empty}}
\begin{center}
	\subsubsection*{Abstract}
	\addcontentsline{toc}{section}{Abstract}
\end{center}

	This text serves as an introduction to representation theory of finite groups, beginning with a background in group theory and linear algebra before formally defining representations, some representations are shown to be irreducible, while others are composed as a direct sum of the irreducibles (Maschke's Theorem~\ref{thm:maschkes}), culminating in the proof of Schur's Lemma~\ref{thm:schur}, which ensure the uniqueness of these compositions. Finally, character theory is introduced, simplifying representation theory by focusing on the trace of matrices, providing systematic methods to identify and decompose any given representation. Examples to demonstrate the these theoretical concepts are some cyclic and symmetric groups of small order. In particular, $\Cyc_3$, $\Cyc_4$, $\Cyc_5$, $\Sym_3$, and $\Sym_4$ are examined and completely decomposed and the results are presented in Tables~\ref{table:Cyc3},~\ref{tbl:cyc4},~\ref{table:Cyc5},~\ref{table:completecharS3}~and~\ref{table:completecharS4}.
		
\begin{center}
	\item\subsubsection*{Sammanfattning}
	\addcontentsline{toc}{section}{Sammanfattning}
	
\end{center}

	Den här texten ger en introduktion till  ändliga gruppers representationsteori. Utifrån en inledning med ett par bakgrundsämnen inom gruppteori och linjär algebra definieras representationer formellt. Vissa representationer visar sig vara oreducerbara, medan andra visar sig vara den direkta summan av oreducerbara representationer (Maschkes Sats~\ref{thm:maschkes}), vilket kulminerar i beviset av  Schurs Lemma~\ref{thm:schur} som visar att dessa sammansättningar är unika. Slutligen introduceras karaktärsteorin som förenklar arbetet med att analysera representationer genom att undersöka matrisers spår. Karaktärsteorin ger systematiska metoder för att identifiera och sönderdela varje given representation. För att demonstrera dessa teoretiska koncept ges några cykliska och symmetriska grupper som exempel, särskilt så undersöks och sönderdelas $\Cyc_3$, $\Cyc_4$, $\Cyc_5$, $\Sym_3$ och $\Sym_4$ helt och resultaten presenteras i Tabellerna~\ref{table:Cyc3},~\ref{tbl:cyc4},~\ref{table:Cyc5},~\ref{table:completecharS3}~och~\ref{table:completecharS4}.
	
%	Sammanfattningen på svenska hamnar här. %\lipsum[2][1-10]


\begin{center}
	\item\subsubsection*{Acknowledgements}
	\addcontentsline{toc}{section}{Acknowledgements}
\end{center}
	
%	\textit{NOT DONE}
	
	I would like to thank my supervisor Sofia Tirabassi for guiding me through the construction of this text. I also would like to thank Stefan Sigurdsson for proofreading the text.
	
%	Artificial intelligence (AI) services were not used in the process of planning, writing or typesetting this text. 
%	This text was typed with {\LaTeX} in TexStudio versions $\leq4.7.3$.
	%\lipsum*[3][1-10]


