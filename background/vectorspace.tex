\subsection{Vector spaces over a field}

In part based on \cite{Jeevanjee}.

\begin{remark}[Modules]
	If $V$ is a vector space over a field $\K$, then $V$ is the same as a $\K$-module.
\end{remark}

\begin{definition}[Dual space]
	For any vector space $V$ over the field $\K$ there is a corresponding dual vector space denoted $V^*$. If the elements of $V$ are $\{\mathbf{v}_i\}_{i=1}^n$ and a basis is $\{\mathbf{e}_i\}_{i=1}^n$, then the dual space $V^*$ has elements $\{\mathbf{v}^i\}_{i=1}^n$ and a basis is $\{\mathbf{e}^i\}_{i=1}^n$. An element $v^i$ of $V^*$ can also be interpreted as a $(1,0)$-tensor
	\[
	v^i: V \longrightarrow \K,
	\]
	in this sense, $V^*$ is the set of linear maps from $V$ to $\K$ denoted by $V^* = \Hom(V,\K)$.
	The spaces $V$ and $V^*$ carry a ``natural pairing'' that is a bilinear mapping
	\[
	\bra{\mathbf{v}^j}\ket{\mathbf{v}_i}: V \times V^* \longrightarrow \K
	\]
	that for basis vectors is defined as
	\[
	\bra{\mathbf{e}^j}\ket{\mathbf{e}_i} = {\delta_i}^j.
	\]
\end{definition}

\begin{definition}[Orthogonal subspace]
	\cite[12.7]{holst} If $W$ is a subspace in $V$, then its complementary subspace in $V$ is defined by 
	\[
	W^\perp = \left\lbrace \vvec \in V \ \middle\vert \ \wvec^\dagger \vvec = 0, \text{ for any } \wvec \in W \right\rbrace.
	\]
\end{definition}

\begin{lemma}[Complementary subspaces]\label{lemma:compsubspace}\cite[Thm 12.16]{holst}
	Let $W$ be a subspace of a space $V$. Let $\dim V = n$ and $\dim W = k$, then:
	\noindent
	\begin{enumerate}[i)]
		\setlength\itemsep{-0.5em}
		\item $\dim W^\perp = n - k$,
		\item $V = W \oplus W^\perp$,
		\item If $\{\ebas_i\}_{i=1}^{n}$ is the basis in $V$, then $\{\ebas_i\}_{i=1}^{k}$ and $\{\ebas_i\}_{i=k+1}^{n}$ are the bases for $W$ respectively $W^\perp$.
	\end{enumerate}
\end{lemma}

\subsubsection{Tensor operations on vector spaces}

	\cite{Jeevanjee}, but some from \cite{holst}
	
	In this section, multilinearity will be presented. Two tensors, operators, matrices etc. are said to be \textbf{bilinear} if when fixing either one, then the other is linear, and vice versa. This is extended to an arbitrary number of pairwise bilinear tensors, then they are called \textbf{multilinear}.
	
	In the following few definitions, let $U$ and $V$ be two vector spaces over the same field $\K$ with respective dimensions $m$, $n$ and bases $\{\uhat\}$, $\{\vhat\}$.
	
	\begin{definition}[Direct sum of vector spaces]
		The direct sum of $U$ and $V$, denoted $U \oplus V$ is a vector space over $\K$ with a basis constructed by all $\uhat \oplus \vhat$. The dimension of $U \oplus V$ is $m+n$. Let $F \in \GL(U)$ and $G \in \GL(V)$. Then $F \oplus G \in \GL(U \oplus V)$ is a block matrix
		\begin{align}
			\begin{pmatrix}
				F & \0 \\
				\0 & G
			\end{pmatrix},
		\end{align}
		where $\0$ are zero matrices ``fitting in the corners'', acting on a vector $\uvec \oplus \vvec$ of $U \oplus V$ like:
		\begin{align}
			\left(
			\begin{array}{ccc;{2pt/2pt}ccc}
				f_{11}                    & \cdots & f_{1m} & 0      & \cdots & 0      \\
				\vdots                    & \ddots & \vdots & \vdots & \ddots & \vdots \\
				f_{m1}                    & \cdots & f_{mm} & 0      & \cdots & 0      \\
				\hdashline[2pt/2pt]
				0 & \cdots & 0      & g_{11} & \cdots & g_{1n} \\
				\vdots                    & \ddots & \vdots & \vdots & \ddots & \vdots \\
				0                         & \cdots & 0      & g_{n1} & \cdots & g_{nn}
			\end{array}
			\right)\cdot
			\begin{pmatrix}
				u_1                         \\
				\vdots                          \\
				u_{m}                       \\
				\hdashline[2pt/2pt]
				v_1 \\
				\vdots                          \\
				v_{n}
			\end{pmatrix}.
		\end{align}
	\end{definition}
	
	\begin{definition}[Tensor product of vector spaces]\label{def:tensorproduct}
		The tensor product of $U$ and $W$, denoted $U \otimes V$ is a vector space over $\K$ with a basis constructed by all $\uhat \otimes \vhat$. The dimension of $U \otimes V$ is $mn$. Let $F \in \GL(U)$ and $G \in \GL(V)$. Then $F \otimes G \in \GL(U \otimes V)$ is a block matrix
		\begin{align}
			(f_{ij}G)_{m \times n}=
			\left(
			\begin{array}{c c c}
				f_{11}G & \cdots & f_{1m}G \\
				\vdots & \ddots & \vdots \\
				f_{m1}G & \cdots & f_{mm}G
			\end{array}
			\right)			
		\end{align}
		where $f_{ij}G$ is understood to be the scalar product of $f_{ij}$ with $G$.
	\end{definition}
	
	\begin{definition}[Tensor power]
		If $V$ is a vector space over $\K$, then for a positive integer $n$, the $n$th tensor power of $V$, denoted by $V^{\otimes n}$ is also a vector space, by induction on definition \ref{def:tensorproduct}.
		\begin{notation}
			The zeroth power of a vector space is defined as its ground field, that is $V^{\otimes 0} = \K$, and the first power is the vector space itself, that is $V^{\otimes 1} = V$.
		\end{notation}    
	\end{definition}