\subsection{Groups}

\cite{DummitFoote}

%\begin{definition}[Center]
%	The center of a group $G$ is the set of elements of $G$ that commute and is denoted by $Z(G)$. 
%	\[
%	Z(G) = \{ g \in G \mid ga = ag \ \forall a \in G \}.
%	\]
%	Since $e$ is in $Z(G)$, it is a subgroup of $G$. If $G$ is abelian, $Z(G) = G$.	
%\end{definition}
%
%\begin{definition}[Centralizer and Normalizer]
%	The centralizer of a subset $A \subseteq G$ is the set of elements that commutes with every element of $A$.
%	The normalizer of a subset $A \subseteq G$ is the set of elements that leaves the set $A$ fixed  under conjugation. Clearly, both the centralizer and the normalizer are subgroups of $G$, and the centralizer is a subgroup of the normalizer.
%\end{definition}

\begin{definition}[Centralizer]
	The centralizer of an element $a$ of $G$ is the set of elements of $G$ that commutes with $a$, and is denoted by 
	\[
	\Cent_G(a) = \{ g \in G \mid ga = ag \}.
	\]
	It is non-empty since it contains the identity element. The centralizer of the entire group is the same as the center, $\Cent_G(G) = Z(G)$.
\end{definition}

%\begin{definition}[Stabilizer]
%	The stabilizer of an element $a$ of $G$ is the set of elements of $G$ that fixes $a$, and is denoted by 
%	\[
%	\Stab_G(a) = \{ g \in G \mid g \star a = a \}.
%	\]
%	It is non-empty since it contains the identity element.
%\end{definition}