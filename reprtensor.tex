\subsection{Tensor operations on representations}\label{sect:tensorrepr}

If we have one or more representations, we may construct additional ones using tensor operations. Given two representations $\rho^V: G \rightarrow \GL(V)$ and $\rho^W: G \rightarrow \GL(W)$, the following are also representations.

\textbf{The direct sum of $V \oplus W$}, given by 
\begin{align*}
	\rho_{V \oplus W}(\vvec \oplus \wvec) = (\rho_V \oplus \rho_W)(\vvec \oplus \wvec) = \rho_V(\vvec) \oplus \rho_W(\wvec).
\end{align*}

By recursion, for a positive integer $n$, the \textbf{direct sum of $n$ copies of $V$}, denoted
\begin{align*}
	nV := \bigoplus_{i=1}^n V = \underset{\text{$n$ times}}{\underbrace{V \oplus \dots \oplus V}}.
\end{align*}

The \textbf{tensor product $V \otimes W$}, given by
\begin{align*}
	\rho_{V \otimes W}(\vvec \otimes \wvec) = (\rho_V \otimes \rho_W)(\vvec \otimes \wvec) = \rho_V(\vvec) \otimes \rho_W(\wvec).
\end{align*}

By recursion, for a positive integer $n$, the \textbf{$n$th tensor power of $V$}, denoted
\begin{align*}
	V^{\otimes n} := \bigotimes_{i=1}^n V = \underset{\text{$n$ times}}{\underbrace{V \otimes \dots \otimes V}}.
\end{align*}





%If we have two or more representations, we may construct additional ones using the tools of  Section~\ref{sect:tensoralgebra}. Given two representations $\rho^V: G \rightarrow \GL(V)$ and $\rho^W: G \rightarrow \GL(W)$ of a group $G$, the following are also representations.

%\begin{itemize}
%	\item The \emph{direct sum of $V \oplus W$} is a representation given by 
%	\begin{align*}
%		\rho_{V \oplus W}(\vvec \oplus \wvec) = (\rho_V \oplus \rho_W)(\vvec \oplus \wvec) = \rho_V(\vvec) \oplus \rho_W(\wvec).
%	\end{align*}
	
%	\item By recursion, for a positive integer $n$, the \emph{direct sum of $n$ copies of $V$}, denoted
%	\begin{align*}
%		nV := \bigoplus_{i=1}^n V = \underset{\text{$n$ times}}{\underbrace{V \oplus \dots \oplus V}},
%	\end{align*}
%	is a representation.
	
%	\item The \emph{tensor product $V \otimes W$} is a representation given by
%	\begin{align*}
%		\rho_{V \otimes W}(\vvec \otimes \wvec) = (\rho_V \otimes \rho_W)(\vvec \otimes \wvec) = \rho_V(\vvec) \otimes \rho_W(\wvec).
%	\end{align*}
	
%	\item By recursion, for a positive integer $n$, the \emph{$n$th tensor power of $V$}, denoted
%	\begin{align*}
%		V^{\otimes n} := \bigotimes_{i=1}^n V = \underset{\text{$n$ times}}{\underbrace{V \otimes \dots \otimes V}},
%	\end{align*}
%	is a representation. The zeroth power is by definition the ground field $\CC$ and the first power is $V$ itself.
%	\item The $n$th tensor power has two subrepresentations, the symmetric powers $\SymP^n V$ and the exterior powers $\AltP^n V$. \textit{Needs further explanation, especially the squares.}
%	\item After fixing a basis $(\bas_i)_{i=1}^k$ for $V$, a \emph{dual representation}, denoted $V^*$, can be defined by the dual space, spanned by the ...
	
%	\item Set of homomorphisms...
%\end{itemize}