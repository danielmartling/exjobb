\subsection{Tensor operations on representations}\label{sect:tensorrepr}

Building on the vector spaces constructed in Section~\ref{sect:tensoralgebra}, given two representations $V$ and $W$ of a group, the following are also representations.

{\allowdisplaybreaks\begin{itemize}
	\item The \emph{direct sum of $V \oplus W$} is a representation given by 
	\begin{align*}
		\rho_{V \oplus W}(\vvec \oplus \wvec) = (\rho_V \oplus \rho_W)(\vvec \oplus \wvec) = \rho_V(\vvec) \oplus \rho_W(\wvec).
	\end{align*}
	
	\item By recursion, for a positive integer $n$, the \emph{direct sum of $n$ copies of $V$}, denoted
	\begin{align*}
		nV := \bigoplus_{i=1}^n V = \underset{\text{$n$ times}}{\underbrace{V \oplus \dots \oplus V}},
	\end{align*}
	is a representation.
	
	\item The \emph{tensor product $V \otimes W$} is a representation given by
	\begin{align*}
		\rho_{V \otimes W}(\vvec \otimes \wvec) = (\rho_V \otimes \rho_W)(\vvec \otimes \wvec) = \rho_V(\vvec) \otimes \rho_W(\wvec).
	\end{align*}
	
	\item By recursion, for a positive integer $n$, the \emph{$n$th tensor power of $V$}, denoted
	\begin{align*}
		V^{\otimes n} := \bigotimes_{i=1}^n V = \underset{\text{$n$ times}}{\underbrace{V \otimes \dots \otimes V}},
	\end{align*}
	is a representation. The zeroth power is by definition the ground field $\CC$ and the first power is $V$ itself.
	\item The $n$th tensor power has two subrepresentations, the symmetric powers $\SymP^n V$ and the exterior powers $\AltP^n V$. \textit{Needs further explanation, especially the squares.}
	\item After fixing a basis $(\bas_i)_{i=1}^k$ for $V$, a \emph{dual representation}, denoted $V^*$, can be defined by the dual space, spanned by the ...
	
	\item Set of homomorphisms...
\end{itemize}}

{\allowdisplaybreaks\begin{example}[Decomposition of the permutation representation of $\Sym_3$]
	Letting $V$ be a permutation representation space of $\Sym_3$, it was found to have two complementary subrepresentations $W$ and $W'$, denoted the trivial and the standard representations respectively. In the $\bas_1, \bas_2, \bas_3$-basis, the matrix for $(1,2)$ is by Example~\ref{ex:permS3}
	\begin{align*}
		\rho^\Perm_{(1,2)} = \begin{pmatrix}
			0&1&0 \\ 1&0&0 \\ 0&0&1
		\end{pmatrix}.
	\end{align*}
	Let's consider a basis change 
	\begin{align*}
		\begin{cases}
			\fbas_1 = \bas_1 + \bas_2 + \bas_3 \\
			\fbas_2 = \bas_1 - \bas_2 \\
			\fbas_3 = \bas_2 - \bas_3
		\end{cases}
	\end{align*}
	with corresponding change-of-basis matrix
	\begin{align*}
		P = \begin{pmatrix}
			1&1&0 \\ 1&-1&0 \\ 1&0&-1
		\end{pmatrix}
	\end{align*}
	in accordance with the bases of the trivial and standard representations. Then the matrix for $(1,2)$ in this new basis is
	\begin{align*}
		 P^{-1} \cdot \rho^\Perm_{(1,2)} \cdot P &= 
		 \frac{1}{3}\begin{pmatrix}
		 	1&1&1\\2&-1&-1\\1&1&-2
		 \end{pmatrix} \cdot \begin{pmatrix}
			 0&1&0 \\ 1&0&0 \\ 0&0&1
		 \end{pmatrix} \cdot \begin{pmatrix}
			 1&1&0 \\ 1&-1&0 \\ 1&0&-1
		 \end{pmatrix} \\
		 &= \begin{pmatrix}
		 	1&0&0 \\ 0&-1&1 \\ 0&0&1
		 \end{pmatrix}\\
%		 &= (1)  \oplus \begin{pmatrix}
%		 	-1&1 \\ 0&1
%		 \end{pmatrix} \\
		 &= \rho_{(1,2)}^\Triv \oplus \rho_{(1,2)}^\Stan.
	\end{align*}
	The same calculations on the other elements of $\Sym_3$ confirm that the permutation representation of $\Sym_3$ decomposes to the direct sum of the trivial and standard representations.
\end{example}}