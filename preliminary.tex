\clearpage{\thispagestyle{empty}}
\section{Preliminary topics}

\subsection{Group theoretical topics}

\subsubsection{Group action}

\textit{Put in definitions as in \cite{DummitFoote}}

We remember that the action of a group $G$ on a set $X$ is a map
\begin{align*}
	G \times X \rightarrow X
\end{align*}
in which every $g \in G$ acts on $X$ by permuting its elements.

We could let the group $G$ act on a set $X$ and study the action of $G$ on $X$ defined by

and thereby permuting the elements of $X$ by for any $g \in G$,
\begin{align*}
	(g,x) \mapsto g \cdot x = gx,
\end{align*}
for any $x \in X$.

\subsubsection{The symmetric group}

Recall~(\cite[1.3.]{DummitFoote}, \cite[1.1]{Sagan}) that for a positive integer $n$, we denote by $\Sym_n$ the set of permutations of the set $\{1, 2, \dots, n\}$, which is a group under compositions of permutations. The number of elements in $\Sym_n$ is $n!$.

The elements may be represented several ways or notations, one of which is \textit{cycle decomposition}. A \textit{cycle} is a string of integers \begin{align*}
	(a_1, a_2, \dots, a_m)
\end{align*} which represent which elements of $\{1,2, \dots, n\}$ that cycle permutes. The cycle above will permute $a_1$ to $a_2$, $a_2$ to $a_3$ and $a_{i}$ to $a_{i+1}$ for $1 \leq i \leq m-1$. Lastly it will permute $a_m$ back to $a_1$, completing the \textit{cycle}. This cycle is of length $m$, hence it is an $m$-cycle. Usually 1-cycles are omitted. 2-cycles are called transpositions. Two cycles are called \textit{disjoint} if they have no integers in common. A cycle representations of an element of $\Sym_n$ is not unique, however it can be uniquely expressed as a composition of \textit{disjoint} cycles.

\begin{example}
	The elements of $\Sym_3$, expressed in cycle decomposition are
	\begin{align*}
		(1),\quad (1,2), \quad(1,3), \quad(2,3), \quad(1,2,3), \quad(1,3,2).
	\end{align*}
	There are $3! = 6$ elements: the identity permutation (denoted $(1)$), three transpositions and two 3-cycles. For example, the element $(1,2)$ is the permutation that maps $1 \mapsto 2$, $2 \mapsto 1$ and $3 \mapsto 3$. 3 is called a \textit{fixed point of $(1,2)$}.
\end{example}

\begin{note}
	The group $\Sym_n$ is generated by all sequent transpositions $(1,2), (2,3), \dots, (n-1,n)$, ie. any element of $\Sym_n$ can be expressed as the composition of transpositions. In $\Sym_3$ we for example have $(1,2,3) = (1,2)\cdot(2,3)$ and $(1,3,2) = (2,3)\cdot(1,2)$.
\end{note}

\begin{definition}[Sign of a permutation]
	Let $\sigma \in \Sym_n$, and let $s$ be the number of transpositions required to compose $\sigma$. Then the function $\sgn: \Sym_n \rightarrow \{\pm 1\}$ is defined as a function
	\begin{align*}
		\sgn: & \ \sigma \mapsto (-1)^s.
	\end{align*}
	If $s$ is an even integer, then $\sigma$ is called even, and vice versa for an odd $s$.
\end{definition}

Note that if $\sigma$ is composed of $s$ transpositions and $\tau$ is composed of $t$ transposition, then their composition $\sigma\tau$ can be composed of $s+t$ transpositions, in some cycle decomposition of $\sigma\tau$. The sign function is still well-defined, that is, a permutation can not be expressed both as a composition of even number and of a odd number of transpositions \cite[Thm.12.6.1.]{Biggs}.

The \textit{cycle type} of a permutation $\sigma \in \Sym_n$ is an $n$-tuple $(k^{m_k})_{k=1}^n$ where $m_k$ is the number of $k$-cycles in $\sigma$. For example, the cycle type of $(1,2)(3,4,5) \in \Sym_6$ is $(1^1, 2^1, 3^1, 4^0, 5^0, 6^0) = (1^1, 2^1, 3^1)$, where in the last step the cycles of zero multiplicity are omitted.

Another way to classify a permutation in $\Sym_n$ is to compare to an \textit{integer partition of $n$}. The integer partition of $n$ is a sum $\lambda_1+ \lambda_2+ \dots+ \lambda_l = n$, where $l \leq n$ and $\lambda_i \geq \lambda_{i+1}$. The element $(1,2)(3,4,5) \in \Sym_6$ corresponds to the partition $(3,2,1)$.  

\begin{example}[$\Sym_3$]
	The elements of $\Sym_3$ and their cycle types and corresponding integer partitions are presented in Table~\ref{table:elementsSym3}.
\end{example}

\begin{table}[hbt!]
	\centering
	\begin{tabular}{r | c c c c c c }
		$\Sym_3$ & $(1)$   & $(1,2)$     & $(1,3)$     & $(2,3)$     & $(1,2,3)$ & $(1,3,2)$ \\ \hline
		    Type & $(1^3)$ & $(1^1,2^1)$ & $(1^1,2^1)$ & $(1^1,2^1)$ & $(3^1)$   & $(3^1)$   \\
		   Part. & 1+1+1   & 2+1         & 2+1         & 2+1         & 3         & 3         \\
		   Sign  & $+1$    & $-1$        & $-1$        & $-1$        & $+1$      & $+1$
	\end{tabular}
	\caption{Elements of $\Sym_3$.}
	\label{table:elementsSym3}
\end{table}

Since the number of elements of $\Sym_n$ is $n!$, for larger $n$ it becomes increasingly cumbersome to describe every element of $\Sym_n$, however as we will see, we can instead study the conjugacy classes of $\Sym_n$.

Recall that to elements $g, g' \in G$ are said to be \textit{conjugate} if there exists an element $h \in G$ such that $g' = hgh^{-1}$. ``Being conjugate in a group'' is an equivalence relation, so the equivalence classes, called \textit{conjugacy classes} partition $G$ into disjoint subsets. Denote by $[g]$ the conjugacy class in $G$ containing $g$. If another element $g'$ is conjugate to $g$, then they share the same conjugacy class $[g] = [g']$ and both $g$ and $g'$ are said to be \textit{representatives} of their class. The size of the conjugacy class can be calculated with the \textit{centralizer of $g$ in $G$}, defined by
\begin{align*}
	\Cent(g) = \left\lbrace h \in G \middle\vert hgh^{-1} = g \right\rbrace,
\end{align*}
and by the orbit-stabilizer theorem~\cite[Thm.21.3]{Biggs}, the relationship between $\Cent(g)$ and the elements of $[g]$ is 
\begin{align*}
	|[g]| = \frac{|G|}{|\Cent(g)|},
\end{align*}
where $| \cdot |$ denotes the size of a set.

Returning to the symmetric group, two permutations $\sigma$ and $\tau$ share conjugacy class if and only if they are of the same cycle type~\cite[Sect.1.1.]{Sagan}. Since the cycle type was linked to an integer partition of the degree of the symmetric group, there are as many conjugacy classes in $\Sym_n$ as there are integer partitions of $n$. For example, there are three conjugacy classes in $\Sym_3$, five in $\Sym_4$ and seven in $\Sym_5$. Also, if the cycle type of a $\sigma \in \Sym_n$ is $(k^{m_k})_{k=1}^n$, then the size of its centralizer is 
\begin{align*}
	\prod_{k=1}^{n} k^{m_k} m_k!
\end{align*}
(\cite[Prop.1.1.1.]{Sagan} has a nice combinatorical proof).

\begin{example}[$\Sym_4$]
	The conjugacy classes of $\Sym_4$, along with their sizes and ther cycle types are presented in Table~\ref{table:elementsSym4}.
	\begin{table}[hbt!]
		\centering
		\begin{tabular}{r | c c c c c}
			         $\Sym_4$ & $(1)$     & $(1,2)$     & $(1,2)(3,4)$ & $(1,2,3)$   & $(1,2,3,4)$ \\ \hline
			$|\Cent(\sigma)|$ & $24$      & $4$         & $8$          & $3$         & $4$         \\
			     $|[\sigma]|$ & $1$       & $6$         & $3$          & $8$         & $6$         \\
			             Type & $(1^4)$   & $(1^2,2^1)$ & $(2^2)$      & $(1^1,2^1)$ & $(4^1)$     \\
			            Part. & 1+1+1+1+1 & 2+1+1       & 2+2          & 3+1         & 5           \\
			             Sign & $+1$      & $-1$        & $+1$         & $+1$        & $-1$
		\end{tabular}
		\caption{Classes of $\Sym_4$.}
		\label{table:elementsSym4}
	\end{table}
\end{example}


\subsection{Linear algebra topics}

	\subsubsection{Trace}
		
		The trace of a matrix is defined as the sum along the diagonal, ie. for a $n \times n$ matrix $X = (x_{ij})$ we have
		\begin{align*}
			\Tr X = \sum_{i=1}^{n} x_{ii}.
		\end{align*}
		It is also the sum of all eigen values of $X$.
		
		Two matrices $X$ and $Y$ are said to be similar if they represent the same linear operator under different basis. What this means is that there exists a matrix $T$ such that $X = TYT^{-1}$. If $X$ and $Y$ are similar, then they have the same trace and eigenvalues~\cite[Thm.5.5.1.]{Nicholson}

	\subsubsection{Kernel and image}
	
	\begin{definition}[Kernel and image of a linear map]\label{def:kernelimage}
		Let $V$ and $W$ be two vector spaces and let $\varphi: V \rightarrow W$ be a linear map. Then the kernel and the image of the map are defined thusly:
		\begin{itemize}
			\item[i)] $\ker \varphi = \left\lbrace \vvec \in V \ \middle\vert \ \varphi(\vvec) = \0 \right\rbrace$.
			\item[ii)] $\im \varphi = \left\lbrace \wvec \in W \ \middle\vert \ \exists \vvec \in V \text{ s.t. } \varphi(\vvec) = \wvec \right\rbrace$.
		\end{itemize}
	\end{definition}
	
	\begin{remark}~\cite[Sect.5.4.]{Holst}
		The kernel and the image of a linear map are subspaces of the domain and codomain of the map respectively, ie. $\ker \varphi$ is a subspace of $V$ and $\im \varphi$ is a subspace of $W$.
	\end{remark}
	
	\subsubsection{Existence of complementary subspaces}
	
	\begin{theorem}\label{thm:compsubspaces}\cite[Thm.12.16]{Holst}
		Let $V$ be a vector space and $W$ be a vector subspace of $V$. Then there exists a complementary vector subspace $W'$ in $V$ such that $W \cap W' = \emptyset$ and $W \cup W' = V$. This is equivalent to saying that $V$ is the direct sum of $W$ and $W'$, denoted as $V = W \oplus W'$.
	\end{theorem}

	
	\subsubsection{Tensor algebra}\label{sect:tensoralgebra}
	
		\textit{Taken from \cite{Jeevanjee} and \cite{Yokonuma}, and some from \cite{Holst}.}
		
		\textit{Something on bilinearity.} $a(v\otimes w) = (av)\otimes w = v \otimes (aw)$ and $v \otimes (w + w')$.

		Let $V$ and $W$ be vector spaces provided with respective bases $(\vhat_i)_{i=1}^m$ and $(\uhat_i)_{i=1}^n$, where $m = \dim V$ and $n = \dim W$. Let \begin{align*}
			\vvec = (v_1, v_2, \dots v_m)^T \in V
		\quad \text{ and } \quad 
		\wvec = (w_1, w_2, \dots, w_n)^T \in W.
	\end{align*}
	Also, in the bases provided let $F=(f_{ij})_{m \times m} \in \GL_m(\CC)$ and $G = (g_{ij})_{n \times n} \in \GL_n(\CC)$ be linear maps with eigen values $\{\lambda_i\}$ and $\{\mu_i\}$, then the following vector spaces may be constructed, or rather extended bilinearly, from $V$ and $W$:
		
		\begin{itemize}
			\item The \emph{direct sum of $V$ and $W$}, denoted $V \oplus W$.
				\subitem It has the basis $(\vhat_i \oplus \what_j)_{1 \leq i \leq m}^{1 \leq j \leq n}$, and is of dimension $m+n$.
%				\subitem It has the basis $(\vhat_i \oplus \what_j)_{i = i,}^{m} _{j=1}^n$, and is of dimension $m+n$.
				\subitem An element of $V \oplus W$ looks like \begin{align*}
					\vvec \oplus \wvec = \begin{pmatrix}
						\vvec \\ \wvec
					\end{pmatrix} = \begin{pmatrix}
					v_1, v_2, \dots, v_m, w_1, w_2, \dots, w_n
					\end{pmatrix}^T.
				\end{align*}
				\subitem In the basis provided, $F \oplus G \in \GL(V \oplus W)$ is the $m+n \times m+n$ block matrix 
				\begin{align*}
					\begin{pmatrix}
						F & \0 \\ 
						\0 & G
					\end{pmatrix},
				\end{align*}
				and its action on $\vvec \oplus \wvec$ is 
				\begin{align*}
					(F \oplus G ) \cdot (\vvec  \oplus \wvec) &= \begin{pmatrix}
						F & \0 \\ 
						\0 & G
					\end{pmatrix} \cdot \begin{pmatrix}
					\vvec \\ \wvec
					\end{pmatrix} \\ 
					&= \begin{pmatrix}
					F\vvec \\ G\wvec
					\end{pmatrix}  \\ 
					&= F\vvec \oplus G\wvec.
				\end{align*}
				\subitem The trace of $F \oplus G$ is clearly the sum of the traces of $F$ and $G$, hence the eigen values of $f$ and $g$ are also eigen values of $f \oplus g$, that is the eigen values are $\{\lambda_i\} \cup \{\mu_j\}$. \textit{Clarify this}
				
			\item By recursion, the \emph{direct sum of $n$ copies of $V$}, denoted $nV$.
				\begin{example}
					The direct sum of $n$ copies of a field $\KK$, is usually denoted $\KK^n$, eg. $\RR^3$.
				\end{example}
				
			\item The \emph{tensor product of $V$ and $W$}, denoted $V \otimes W$.
				\subitem It has the basis $(\vhat_i \otimes \what_j)_{1 \leq i \leq m}^{1 \leq j \leq n}$, and is of dimension $mn$.
				\subitem An element of $V \otimes W$ looks like \begin{align*}
					\vvec \otimes \wvec = \begin{pmatrix}
						v_1\wvec, v_2\wvec, \dots, v_m\wvec
					\end{pmatrix}^T = \begin{pmatrix}
						v_i\wvec
					\end{pmatrix}_{mn \times 1} 
				\end{align*}
				\subitem In the basis provided, $F \otimes G \in \GL(V \otimes W)$ is the block matrix 
				\begin{align*}
					(f_{ij}G)_{mn \times mn}
				\end{align*}
				and its action on $\vvec \otimes \wvec$ is 
				\begin{align*}
					(F \otimes G ) \cdot (\vvec  \otimes \wvec) &= \begin{pmatrix}
						f_{ij}G
					\end{pmatrix}\cdot \begin{pmatrix}
						v_i\wvec
					\end{pmatrix} \\ 
					&= \begin{pmatrix}
						f_{ij}G v_i \wvec
					\end{pmatrix} \\ 
					&= \begin{pmatrix}
						f_{ij} v_i G\wvec
					\end{pmatrix} \\
					&=  \begin{pmatrix}
						f_{ij}v_i
					\end{pmatrix} \otimes G\wvec \\
					&= F\vvec \otimes G\wvec.
				\end{align*}
				\subitem The trace of $F \otimes G$ is the sum of the traces of the diagonal matrices in the block matrix $(f_{ij}G)$, which is 
				\begin{align*}
					f_{11} \Tr G + f_{22} \Tr G + \dots + f_{mm} \Tr G = \Tr F \cdot \Tr G
				\end{align*}
				hence the eigen values of $f \otimes g$ are $\{\lambda_i \cdot \mu_j\}$.
				
			\item By recursion, the \emph{$n$th tensor power of $V$}, denoted $V^{\otimes n}$. By definition, the first tensor power is the space itself, and the zeroth power is the ground field.
			
			\item The $n$th tensor power of $V$ has two subspaces, the symmetric powers $\SymP^n V$ and the alternating powers $\AltP^n V$. In particular the symmetric and exterior squares are such that
			\begin{align*}
				V \otimes V = \SymP^2 V \oplus \AltP^2 V.
			\end{align*}\marginnote{need more text/proof}
			
			\begin{figure}[hbt!]
				\centering
				\includegraphics[width=0.7\linewidth]{notesontensors}
				\caption{}
				\label{fig:notesontensors}
			\end{figure}
			
			
			\item After fixing the basis $(\bas_i)_{i=1}^m$ for $V$, the dual space $V^*$ can be constructed by the duals $\bas_i^*$ defined by... It is identified with the set of all linear functions from $V$ to $\CC$.\marginnote{need more text/proof}
			
			\item Set of homomorphisms $V$ to $W$.\marginnote{need more text/proof}
			
		\end{itemize} 
		
	

		
		
		