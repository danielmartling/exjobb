\clearpage{\thispagestyle{empty}}
\section{Preliminary topics}

\subsection{Group theoretical topics}

%\subsubsection{Group action}

The action of a group $G$ on a set $X$ is a map
\begin{align*}
	\sigma : G \times X \rightarrow X
\end{align*}
satisfying for the identity element $e \in G$ 
\begin{align*}
	&\sigma(e,x) = x, \text{ for all $x \in X$}
\end{align*}
and for all $g,h \in G$,
\begin{align*}
	&\sigma (g, \sigma(h,x)) = \sigma(gh, x), \text{ for all } x \in X.
\end{align*}
It is easy to see that, for every fixed $g \in G$ the map $\sigma(g, \cdot): X \rightarrow X$ is a permutation, that is bijective. To simplify notation, we write
\begin{align*}
	g \cdot x := \sigma(g,x).
\end{align*}
%where $e$ is the identity element of $G$.
%in which every $g \in G$ acts on $X$ by permuting its elements.

%We could let the group $G$ act on a set $X$ and study the action of $G$ on $X$ defined by
%
%and thereby permuting the elements of $X$ by for any $g \in G$,
%\begin{align*}
%	(g,x) \mapsto g \cdot x = gx,
%\end{align*}
%for any $x \in X$.

%\subsubsection{The symmetric group}

Recall~(\cite[1.3.]{DummitFoote}, \cite[1.1]{Sagan}) that for a positive integer $n$, we denote by $\Sym_n$ the set of permutations of the set $\{1, 2, \dots, n\}$, which is a group under compositions of permutations. The number of elements in $\Sym_n$ is $n!$.

The elements may be represented several ways or notations, one of which is \textit{cycle decomposition}. A \textit{cycle} is a string of integers \begin{align*}
	(a_1, a_2, \dots, a_m)
\end{align*} which represent which elements of $\{1,2, \dots, n\}$ that cycle permutes. The cycle above will permute $a_1$ to $a_2$, $a_2$ to $a_3$ and $a_{i}$ to $a_{i+1}$ for $1 \leq i \leq m-1$. Lastly it will permute $a_m$ back to $a_1$, completing the \textit{cycle}. This cycle is of length $m$, hence it is an $m$-cycle. Usually 1-cycles are omitted. 2-cycles are called transpositions. Two cycles are called \textit{disjoint} if they have no integers in common. A cycle representations of an element of $\Sym_n$ is not unique, however it can be uniquely expressed as a composition of \textit{disjoint} cycles.

\begin{example}
	The elements of $\Sym_3$, expressed in cycle decomposition are
	\begin{align*}
		(1),\quad (1,2), \quad(1,3), \quad(2,3), \quad(1,2,3), \quad(1,3,2).
	\end{align*}
	There are $3! = 6$ elements: the identity permutation (denoted $(1)$), three transpositions and two 3-cycles. For example, the element $(1,2)$ is the permutation that maps $1 \mapsto 2$, $2 \mapsto 1$ and $3 \mapsto 3$. 3 is called a \textit{fixed point of $(1,2)$}.
\end{example}

\begin{note}
	The group $\Sym_n$ is generated by all sequent transpositions $(1,2), (2,3), \dots, (n-1,n)$, ie. any element of $\Sym_n$ can be expressed as the composition of transpositions. In $\Sym_3$ we for example have $(1,2,3) = (1,2)\cdot(2,3)$ and $(1,3,2) = (2,3)\cdot(1,2)$.
\end{note}

\begin{definition}[Sign of a permutation]
	Let $\sigma \in \Sym_n$, and let $s$ be the number of transpositions required to compose $\sigma$. Then the function $\sgn: \Sym_n \rightarrow \{\pm 1\}$ is defined as a function
	\begin{align*}
		\sgn: & \ \sigma \mapsto (-1)^s.
	\end{align*}
	If $s$ is an even integer, then $\sigma$ is called even, and vice versa for an odd $s$.
\end{definition}

Note that if $\sigma$ is composed of $s$ transpositions and $\tau$ is composed of $t$ transposition, then their composition $\sigma\tau$ can be composed of $s+t$ transpositions, in some cycle decomposition of $\sigma\tau$. The sign function is still well-defined, that is, a permutation can not be expressed both as a composition of even number and of a odd number of transpositions \cite[Thm.12.6.1.]{Biggs}.

The \textit{cycle type} of a permutation $\sigma \in \Sym_n$ is an $n$-tuple $(k^{m_k})_{k=1}^n$ where $m_k$ is the number of $k$-cycles in $\sigma$. For example, the cycle type of $(1,2)(3,4,5) \in \Sym_6$ is $(1^1, 2^1, 3^1, 4^0, 5^0, 6^0) = (1^1, 2^1, 3^1)$, where in the last step the cycles of zero multiplicity are omitted.

Another way to classify a permutation in $\Sym_n$ is to compare it to an \textit{integer partition of $n$}. An integer partition of $n$ is a sum $\lambda_1+ \lambda_2+ \dots+ \lambda_l = n$, where $l \leq n$ and $\lambda_i \geq \lambda_{i+1}$. The element $(1,2)(3,4,5) \in \Sym_6$ corresponds to the partition $(3,2,1)$~\cite[??]{Sagan}.  

\begin{example}[$\Sym_3$]
	The elements of $\Sym_3$ and their signs,  cycle types and corresponding integer partitions are presented in Table~\ref{table:elementsSym3}.
\end{example}

\begin{table}[hbt!]
	\centering
	\begin{tabular}{r | c c c c c c }
		$\Sym_3$ & $(1)$   & $(1,2)$     & $(1,3)$     & $(2,3)$     & $(1,2,3)$ & $(1,3,2)$ \\ \hline
		    Type & $(1^3)$ & $(1^1,2^1)$ & $(1^1,2^1)$ & $(1^1,2^1)$ & $(3^1)$   & $(3^1)$   \\
		   Part. & 1+1+1   & 2+1         & 2+1         & 2+1         & 3         & 3         \\
		   Sign  & $+1$    & $-1$        & $-1$        & $-1$        & $+1$      & $+1$
	\end{tabular}
	\caption{Elements of $\Sym_3$.}
	\label{table:elementsSym3}
\end{table}

Since the number of elements of $\Sym_n$ is $n!$, for larger $n$ it becomes increasingly cumbersome to describe every element of $\Sym_n$, however as we will see, we can instead study the conjugacy classes of $\Sym_n$.

Recall that to elements $g, g' \in G$ are said to be \textit{conjugate} if there exists an element $h \in G$ such that $g' = hgh^{-1}$. ``Being conjugate in a group'' is an equivalence relation, so the equivalence classes, called \textit{conjugacy classes} partition $G$ into disjoint subsets~\cite[??]{Biggs}. Denote by $[g]$ the conjugacy class in $G$ containing $g$. If another element $g'$ is conjugate to $g$, then they share the same conjugacy class, ie. $[g] = [g']$, and both $g$ and $g'$ are said to be \textit{representatives} of their class. The size of the conjugacy class can be calculated with the \textit{centralizer of $g$ in $G$}, defined by
\begin{align*}
	\Cent(g) = \left\lbrace h \in G \ \middle\vert \ hgh^{-1} = g \right\rbrace,
\end{align*}
and by the orbit-stabilizer theorem~\cite[Thm.21.3]{Biggs}, the relationship between $\Cent(g)$ and the elements of $[g]$ is 
\begin{align*}
	|[g]| = \frac{|G|}{|\Cent(g)|},
\end{align*}
where $| \cdot |$ denotes the size of a set.

Returning to the symmetric group, two permutations $\sigma$ and $\tau$ share conjugacy class if and only if they are of the same cycle type~\cite[Sect.1.1.]{Sagan},~\cite[??]{Biggs}. Since the cycle type was linked to an integer partition of the degree of the symmetric group, there are as many conjugacy classes in $\Sym_n$ as there are integer partitions of $n$. For example, there are three conjugacy classes in $\Sym_3$, five in $\Sym_4$ and seven in $\Sym_5$. Also, if the cycle type of a $\sigma \in \Sym_n$ is $(k^{m_k})_{k=1}^n$, then the size of its centralizer is 
\begin{align*}
	\prod_{k=1}^{n} k^{m_k} m_k!,
\end{align*}
(\cite[Prop.1.1.1.]{Sagan} has a nice combinatorical proof).

\begin{example}[$\Sym_4$]
	The conjugacy classes of $\Sym_4$, along with their sizes, cycle types and signs are presented in Table~\ref{table:elementsSym4}.
	\begin{table}[hbt!]
		\centering
		\begin{tabular}{r | c c c c c}
			         $\Sym_4$ & $(1)$   & $(1,2)$     & $(1,2)(3,4)$ & $(1,2,3,4)$ & $(1,2,3)$   \\ \hline
			$|\Cent(\sigma)|$ & $24$    & $4$         & $8$          & $4$         & $3$         \\
			     $|[\sigma]|$ & $1$     & $6$         & $3$          & $6$         & $8$         \\
			             Type & $(1^4)$ & $(1^2,2^1)$ & $(2^2)$      & $(4^1)$     & $(1^1,2^1)$ \\
			            Part. & 1+1+1+1 & 2+1+1       & 2+2          & 5           & 3+1         \\
			             Sign & $+1$    & $-1$        & $+1$         & $-1$& $+1$        
		\end{tabular}
		\caption{Classes of $\Sym_4$.}
		\label{table:elementsSym4}
	\end{table}
\end{example}


\subsection{Linear algebra topics}

%	\subsubsection{Trace}
	
		We recall from linear algebra the concept of the trace of a square matrix $(a_{ij})_{n \times n}$. It is the sum of the elements along the diagonal,
		\begin{align*}
			\Tr(a_{ij})= \sum_{i=1}^{n} a_{ii} = a_{11} + \dots + a_{nn},
		\end{align*}
		and has the following properties:
		\begin{proposition}\label{prop:trace}
			\begin{enumerate}
				\item[i)] It is the sum of the eigenvalues of a matrix~\cite[Cor.8.6.1.]{Nicholson}.
				\item[ii)] Two similar matrices have the same trace~\cite[Thm.5.5.1.]{Nicholson}, consequently the trace is independent of the basis chosen.
				\item[iii)] It is constant under conjugation (Equivalent to ii) and also $\Tr AB = \Tr BA$, see~\cite[Ex.2.3.30.]{Nicholson}).
			\end{enumerate}
		\end{proposition}

%	\subsubsection{Vector space maps and projections}

	Also of importance is the kernel and image of a vector space map, a linear map between two vector spaces. A subset of a vector space that satisfy the vector space axioms is called a vector subspace. 
		
	\begin{definition}[Kernel and image of a linear map]\label{def:kernelimage}
		Let $V$ and $W$ be two vector spaces and let $\varphi: V \rightarrow W$ be a linear map. Then the kernel and the image of the map are defined thusly:
		\begin{align*}
			 \ker \varphi &= \left\lbrace \vvec \in V \ \middle\vert \ \varphi(\vvec) = \0 \right\rbrace, \\  \im \varphi &= \left\lbrace \wvec \in W \ \middle\vert \ \exists \vvec \in V \text{ s.t. } \varphi(\vvec) = \wvec \right\rbrace.
		\end{align*}
	\end{definition}
	
	\begin{remark}
		The kernel and the image of a linear map are subspaces of the domain and codomain of the map respectively, ie. $\ker \varphi$ is a subspace of $V$ and $\im \varphi$ is a subspace of $W$~\cite[Sect.5.4.]{Holst}.
	\end{remark}	
	
	\begin{theorem}\label{thm:compsubspaces}\cite[Thm.12.16]{Holst}
		Let $V$ be a vector space and $W$ be a vector subspace of $V$. Then there exists a complementary vector subspace $W'$ in $V$ such that $W \cap W' = \emptyset$ and $W \cup W' = V$. This is equivalent to saying that $V$ is \textit{the direct sum of $W$ and $W'$}, denoted as $V = W \oplus W'$.
	\end{theorem}
	
\begin{example}
		Letting $V$ be a vector space with a subspace $W$, a linear map $\pi: V \rightarrow W$ is called a projection of $V$ onto $W$. For this $\pi$ we have that
	\begin{enumerate}
		\item[i)]  $\pi^2 = \pi$.
		\item[ii)] For any $\vvec \in W$ we have that $\pi(\vvec) = \vvec$.
		\item[iii)] If $\vvec \notin W$ we have that $\pi(\vvec) = \0$. 
	\end{enumerate}
	Hence $\im \pi = W$ and $\ker \pi$ is a complementary subspace to $W$ in $V$, ie. $V  = W \oplus \ker \pi$.
\end{example}
		
\subsection{Tensor operations}\label{sect:tensoralgebra}

For this section, let $V$ and $W$ be vector spaces of dimensions $m$ and $n$ and bases $(\bas_i)_{i=1}^m$ and $(\fbas_j)_{j=1}^n$
	
In Theorem~\ref{thm:compsubspaces} the \textbf{direct sum of vector spaces} was introduced. More generally, if $V$ and $W$ are any vector spaces, $V \oplus W$ is the set of pairs $(\vvec, \wvec) := \vvec \oplus \wvec$ where $\vvec \in V$ and $\wvec \in W$. The basis of $V \oplus W$ is constructed from the pairs $\bas_i \oplus \bas_j$, hence $\dim V \oplus W = \dim V + \dim W$. In the sense of linear operators, if $F \in \GL(V)$ and $G \in \GL(W)$, then the direct sum of $F$ and $G$ is in some basis the block matrix
\begin{align*}
	F \oplus G = \begin{pmatrix}
	F & \0 \\
	\0 & G
\end{pmatrix}
\end{align*} and it will act on the elements of $V \oplus W$ bilinearly, ie. 
	\begin{align*}
			(F \oplus G) \cdot (\vvec \oplus \wvec) &= \begin{pmatrix}
					F & \0 \\
					\0 & G
				\end{pmatrix} \cdot (\vvec \oplus \wvec)  \\
			&= (F\vvec) \oplus (G \wvec).
		\end{align*} The direct sum of an arbitrary number of vector spaces is defined similarly~\cite[Sect.1.3.]{Serre},~\cite[Sect.1.5.]{Sagan}.
		
\begin{notation}
	The direct sum of $n$ copies of a vector space $V$ is here denoted $nV$.  An element of $nV$ is an $n$-tuple $(\vvec_1, \dots, \vvec_n)$ where every $\vvec_i \in V$.
\end{notation}

\begin{example}
	The direct sum of $n$ copies of a field $\KK$, $n\KK$, is usually denoted $\KK^n$, for example the real space $\RR^3$ is the direct sum of three pair-wise orthogonal lines.
\end{example}

A vector space equipped with a map
\begin{align*}
	(\vvec, \wvec) \mapsto \vvec \otimes \wvec
\end{align*}
is called a \textbf{tensor product of $V$ and $W$} and denoted by $V \otimes W$ if it is linear in each constituent space (this means that $(\vvec_1 + \vvec_2, \wvec_1 + \wvec_2) =  (\vvec_1, \wvec_1 + \wvec_2) + (\vvec_2, \wvec_1 + \wvec_2) = (\vvec_1, \wvec_1) + (\vvec_1, \wvec_2) + (\vvec_2, \wvec_1) + (\vvec_2, \wvec_2)$), and that the set $(\bas_i \otimes \fbas_j)$ is a basis for $V \otimes W$. An element of $V \otimes W$ is a linear combination of tensor products of elements from $V$ and $W$. For the linear maps $F=(f_{ij}) \in \GL(V)$ and $G=(g_{ij}) \in \GL(W)$ we have that their tensor product is the $mn \times mn$ block matrix
\begin{align*}
		F \otimes G = (f_{ij}G) = \begin{pmatrix}
				f_{11}G & \cdots & f_{1m}G  \\
				\vdots & \ddots & \vdots \\
				f_{m1} & \cdots & f_{mm}G
			\end{pmatrix}
	\end{align*}
and its action on an element $\vvec \otimes \wvec \in V \otimes W$ is again bilinear:
\begin{align*}
		(F \otimes G) \cdot (\vvec \otimes \wvec) &= (f_{ij}G)  \cdot (\vvec \otimes \wvec)  \\
		&= (F\vvec) \otimes (G\wvec).
	\end{align*} The tensor product of an arbitrary number of vector spaces is similarly defined~\cite[Sect.1.5.]{Serre},~\cite[Sect.1.7.]{Sagan}.
	
\begin{notation}
	We denote by $V^{\otimes n}$ the tensor product of $n$ copies of $V$. This vector space is called the $n$th tensor power of $V$.
\end{notation}

\begin{example}
	The \textbf{tensor square of $V$} is denoted by $V \otimes V$. An element of $V \otimes V$ is a linear combination of pairs of elements from $V$, that is they look like
	\begin{align*}
		\sum_{ij} a_{ij} \vvec_i \otimes \wvec_j.
	\end{align*}
	Let $\theta$ be a linear map from $V \otimes V$ to itself defined by 
	\begin{align*}
		\theta(\vvec \otimes \wvec) = \wvec \otimes \vvec,
	\end{align*}
	that is it switches the two elements of $V$. It is clearly invertible since it is its own inverse, hence it is an isomorphism. Denote by $\SymP^2 V$ the subset of elements $\vvec \otimes \wvec \in V \otimes V$ such that $\theta(\vvec \otimes \wvec) = \vvec \otimes \wvec$ and by $\AltP^2 V$ the subset of elements $\vvec \otimes \wvec \in V \otimes V$ such that $\theta(\vvec \otimes \wvec) = -\vvec \otimes \wvec$. These two sets are respectively called the \textbf{symmetric square of $V$} and the \textbf{exterior square of $V$}~\cite[Sect.1.6.]{Serre},~\cite[Sects.1.1,B.2.]{FultonHarris}.
	
	The set $(\bas_i \otimes \bas_j + \bas_j \otimes \bas_i)_{i \leq j}$ is a basis of $\SymP^2 V$ and $(\bas_i \otimes \bas_j - \bas_j \otimes \bas_i)_{i \leq j}$ is a basis of $\AltP^2 V$. Letting $\dim V = n$, then we have that 
	\begin{align*}
		\dim \SymP^2 V = \frac{n(n+1)}{2},\quad\text{and}\quad \dim \AltP^2 V = \frac{n(n-1)}{2}.
	\end{align*}
	
	These choices of bases indicates that the direct sum of the symemtric and alternating squares of $V$ precisely is the tensor square of $V$, ie.
	\begin{align*}
		V \otimes V = \SymP^2 V \oplus \AltP^2 V.
	\end{align*}
\end{example}

The set of all linear maps from a space $V$ to $\CC$ is called the \textbf{dual space of $V$}.

%	
%	\paragraph{Direct sum $V \oplus W$.} Let $V$ and $W$ be vector spaces of dimensions $m$ and $n$ respectively. The direct sum of $V$ and $W$, denoted $V \oplus W$ is a vector space consisting of all linear combinations of pairs $(\vvec, \wvec) := \vvec \oplus \wvec$,  where $\vvec \in V$ and $\wvec \in W$. If $(\bas_i)_{i=1}^m$ is a basis for $V$ and $(\fbas_j)_{j=1}^n$ is a basis for $W$, then  $(\bas_i \oplus \0 )_{i=1}^m \cup (\0 \oplus \fbas_j)_{j=1}^n$ is a basis for $V \oplus W$. Clearly $\dim V\oplus W = \dim V + \dim W = m+n$.
%	
%	In the sense of linear operators, if $F \in \GL(V)$ and $G \in \GL(W)$, then the direct sum of $F$ and $G$ is in the basis above the block matrix
%	\begin{align*}
%		F \oplus G = \begin{pmatrix}
%			F & \0 \\
%			\0 & G
%		\end{pmatrix}
%	\end{align*}
%	and it will act on a $\vvec \oplus \wvec$ like
%	\begin{align*}
%		(F \oplus G) \cdot (\vvec \oplus \wvec) &= \begin{pmatrix}
%			F & \0 \\
%			\0 & G
%		\end{pmatrix} \cdot (\vvec \oplus \wvec)  \\
%		&= (F\vvec) \oplus (G \wvec).
%	\end{align*}
%	
%	\paragraph{$n$-fold direct sum $nV$.} By recursion on the last definition, an element of $nV$ is an $n$-tuple $(\vvec_1, \dots, \vvec_n)$ where every $\vvec_i \in V$.
%	
%	\begin{example}
%		The direct sum of $n$ copies of a field $\KK$, $n\KK$, is usually denoted $\KK^n$, for example $\RR^3$ is the direct sum of three pair-wise orthogonal lines.
%	\end{example}
%	
%	\paragraph{Tensor product $V \otimes W$.} An element of $V \otimes W$ is a linear combination of pairs on the form $\vvec \otimes \wvec$, where $\vvec \in V$ and $\wvec \in W$.  A basis for $V \otimes W$ is constructed by every product $\bas_i \otimes \bas_j$, hence $\dim  V \otimes W = \dim V \cdot \dim W = mn$. The tensor product of two linear operators  $F=(f_{ij}) \in \GL(V)$ and $G=(g_{ij}) \in \GL(W)$ is the $mn \times mn$ block matrix
%	\begin{align*}
%		F \otimes G = (f_{ij}G) = \begin{pmatrix}
%			f_{11}G & \cdots & f_{1m}G  \\
%			\vdots & \ddots & \vdots \\
%			f_{m1} & \cdots & f_{mm}G
%		\end{pmatrix}
%	\end{align*}
%	and its action on an element $\vvec \otimes \wvec \in V \otimes W$ is
%	\begin{align*}
%		(F \otimes G) \cdot (\vvec \otimes \wvec) &= (f_{ij}G)  \cdot (\vvec \otimes \wvec)  \\
%		&= 
%	\end{align*}
	
	
%		\textit{Taken from \cite{Jeevanjee} and \cite{Yokonuma}, and some from \cite{Holst}.}
%		
%		\textit{Something on bilinearity.} $a(v\otimes w) = (av)\otimes w = v \otimes (aw)$ and $v \otimes (w + w')$.
%
%		Let $V$ and $W$ be vector spaces provided with respective bases $(\vhat_i)_{i=1}^m$ and $(\uhat_i)_{i=1}^n$, where $m = \dim V$ and $n = \dim W$. Let \begin{align*}
%			\vvec = (v_1, v_2, \dots v_m)^T \in V
%		\quad \text{ and } \quad 
%		\wvec = (w_1, w_2, \dots, w_n)^T \in W.
%	\end{align*}
%	Also, in the bases provided let $F=(f_{ij})_{m \times m} \in \GL_m(\CC)$ and $G = (g_{ij})_{n \times n} \in \GL_n(\CC)$ be linear maps with eigen values $\{\lambda_i\}$ and $\{\mu_i\}$, then the following vector spaces may be constructed, or rather extended bilinearly, from $V$ and $W$:
%		\begin{itemize}
%			\item The \emph{direct sum of $V$ and $W$}, denoted $V \oplus W$ is constructed of every linear combination of $\vvec_i \otimes \wvec_j$.
%				Any basis vector of $V$ and $W$ are also basis vectors of $V \oplus W$ hence it is of dimension $m+n$. An element of $V \oplus W$ looks like \begin{align*}
%					\vvec \oplus \wvec = \begin{pmatrix}
%						\vvec \\ \wvec
%					\end{pmatrix} = \begin{pmatrix}
%					v_1, v_2, \dots, v_m, w_1, w_2, \dots, w_n
%					\end{pmatrix}^T.
%				\end{align*}
%				In the basis provided, $F \oplus G \in \GL(V \oplus W)$ is the $m+n \times m+n$ block matrix 
%				\begin{align*}
%					\begin{pmatrix}
%						F & \0 \\ 
%						\0 & G
%					\end{pmatrix},
%				\end{align*}
%				and its action on $\vvec \oplus \wvec$ is 
%				\begin{align*}
%					(F \oplus G ) \cdot (\vvec  \oplus \wvec) &= \begin{pmatrix}
%						F & \0 \\ 
%						\0 & G
%					\end{pmatrix} \cdot \begin{pmatrix}
%					\vvec \\ \wvec
%					\end{pmatrix} \\ 
%					&= \begin{pmatrix}
%					F\vvec \\ G\wvec
%					\end{pmatrix}  \\ 
%					&= F\vvec \oplus G\wvec.
%				\end{align*}
%				The trace of $F \oplus G$ is clearly the sum of the traces of $F$ and $G$, hence the eigenvalues of $f$ and $g$ are also eigenvalues of $F \oplus G$, that is they are $\{\lambda_i\} \cup \{\mu_j\}$.
%				
%			\item By recursion, the \emph{direct sum of $n$ copies of $V$}, in this text denoted $nV$.
%				\begin{example}
%					The direct sum of $n$ copies of a field $\KK$, is usually denoted $\KK^n$, eg. $\RR^3$.
%				\end{example}
%				
%			\item The \emph{tensor product of $V$ and $W$}, denoted $V \otimes W$.
%				It has a basis constructed from the set $\{\vhat_i \otimes \what_j\}$, and is of dimension $mn$. An element of $V \otimes W$ looks like \begin{align*}
%					\vvec \otimes \wvec = \begin{pmatrix}
%						v_1\wvec, v_2\wvec, \dots, v_m\wvec
%					\end{pmatrix}^T = \begin{pmatrix}
%						v_i\wvec
%					\end{pmatrix}_{mn \times 1} 
%				\end{align*}
%				In the basis provided, $F \otimes G \in \GL(V \otimes W)$ is the block matrix 
%				\begin{align*}
%					(f_{ij}G)_{mn \times mn}
%				\end{align*}
%				and its action on $\vvec \otimes \wvec$ is 
%				\begin{align*}
%					(F \otimes G ) \cdot (\vvec  \otimes \wvec) &= \begin{pmatrix}
%						f_{ij}G
%					\end{pmatrix}\cdot \begin{pmatrix}
%						v_i\wvec
%					\end{pmatrix} \\ 
%					&= \begin{pmatrix}
%						f_{ij}G v_i \wvec
%					\end{pmatrix} \\ 
%					&= \begin{pmatrix}
%						f_{ij} v_i G\wvec
%					\end{pmatrix} \\
%					&=  \begin{pmatrix}
%						f_{ij}v_i
%					\end{pmatrix} \otimes G\wvec \\
%					&= F\vvec \otimes G\wvec.
%				\end{align*}
%				The trace of $F \otimes G$ is the sum of the traces of the diagonal matrices in the block matrix $(f_{ij}G)$, which is 
%				\begin{align*}
%					f_{11} \Tr G + f_{22} \Tr G + \dots + f_{mm} \Tr G = \Tr F \cdot \Tr G
%				\end{align*}
%				hence the eigenvalues of $f \otimes g$ are $\{\lambda_i \cdot \mu_j\}$.
%				
%			\item By recursion, the \emph{$n$th tensor power of $V$}, denoted $V^{\otimes n}$.
%			
%			\item The $n$th tensor power of $V$ has two subspaces, the symmetric powers $\SymP^n V$ and the alternating powers $\AltP^n V$. In particular the symmetric and exterior squares are such that
%			\begin{align*}
%				V \otimes V = \SymP^2 V \oplus \AltP^2 V.
%			\end{align*}\marginnote{need more text/proof}
%			
%			\begin{figure}[hbt!]
%				\centering
%				\includegraphics[width=0.7\linewidth]{notesontensors}
%				\caption{}
%				\label{fig:notesontensors}
%			\end{figure}
%			
%			
%			\item After fixing the basis $(\bas_i)_{i=1}^m$ for $V$, the dual space $V^*$ can be constructed by the duals $\bas_i^*$ defined by... It is identified with the set of all linear functions from $V$ to $\CC$.\marginnote{need more text/proof}
%			
%			\item Set of homomorphisms $V$ to $W$.\marginnote{need more text/proof}
%			
%		\end{itemize} 
%		
%	

		
		
		