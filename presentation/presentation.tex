\documentclass[handout,  12pt]{beamer}
\mode<presentation>
\usepackage[utf8]{inputenc}
\usepackage[T1]{fontenc}
\usepackage{lmodern}
\usetheme{metropolis}




% Blackboard Boldface
\newcommand{\NN}{\mathbb{N}}
\newcommand{\ZZ}{\mathbb{Z}}
\newcommand{\QQ}{\mathbb{Q}}
\newcommand{\RR}{\mathbb{R}}
\newcommand{\CC}{\mathbb{C}}
\newcommand{\KK}{\mathbb{K}}
\newcommand{\FF}{\mathbb{F}}

\newcommand{\1}{\mathbf{1}}
\newcommand{\0}{\mathbf{0}}

% Text
\newcommand{\GL}{\text{GL}}
\newcommand{\SL}{\text{SL}}
\newcommand{\SymP}{\text{Sym}}
\newcommand{\AltP}{{\bigwedge}}
\newcommand{\Hom}{\text{Hom}}
\newcommand{\End}{\text{End}}
\newcommand{\SO}{\text{SO}}
\newcommand{\SU}{\text{SU}}
\newcommand{\id}{\text{id}}
\newcommand{\Stab}{\text{Stab}}
\newcommand{\Cent}{\text{Cent}}
\newcommand{\Norm}{\text{Norm}}
\newcommand{\Type}{\text{Type}}
\newcommand{\Dih}{\text{D}}
\newcommand{\Sym}{\mathcal{S}} %{\mathfrak{S}} % \text{S}
%\newcommand{\Alt}{\text{A}}
\newcommand{\Klein}{\text{V}}
\newcommand{\Cyc}{\mathcal{C}}%{\text{C}}
\newcommand{\sgn}{\text{sgn}}
\newcommand{\Span}{\text{Span}}
\newcommand{\im}{\text{im }}
\newcommand{\Tr}{\text{Tr }}
\newcommand{\Reg}{\text{Reg}}
\newcommand{\Triv}{\text{Triv}}
\newcommand{\Alt}{\text{Alt}}
\newcommand{\Perm}{\text{Perm}}
\newcommand{\Stan}{\text{Stan}}
% Vectors
\newcommand{\bas}{\mathbf{\hat{e}}}
\newcommand{\fbas}{\mathbf{\hat{f}}}

\newcommand{\ihat}{\mathbf{\hat{\imath}}}
\newcommand{\jhat}{\mathbf{\hat{\jmath}}}
\newcommand{\ghat}{\mathbf{\hat{g}}}
\newcommand{\hhat}{\mathbf{\hat{h}}}
\newcommand{\vhat}{\mathbf{\hat{v}}}
\newcommand{\what}{\mathbf{\hat{w}}}
\newcommand{\uhat}{\mathbf{\hat{u}}}
\newcommand{\xhat}{\mathbf{\hat{x}}}

\newcommand{\vvec}{\mathbf{v}}
\newcommand{\uvec}{\mathbf{u}}
\newcommand{\wvec}{\mathbf{w}}


\newcommand{\Ss}{\scriptstyle}

%\newcommand{\barpi}{\overline{\pi}}
\newcommand{\barpi}{\bar{\pi}}

%\allowdisplaybreaks

\begin{document}
	\author{Daniel Martling}
	\title{Introduction to Representation Theory of Finite Groups}
	%\subtitle{}
	\logo{\includegraphics[height=0.5cm]{logo.pdf}}
	%\institute{}
	\date{2024-06-12}
	%\subject{}
%	\setbeamercovered{transparent}
%	\setbeamertemplate{navigation symbols}{}
	\begin{frame}
		\maketitle
	\end{frame}
	\begin{frame}{Dagordning}
		\begin{itemize}
			\item First definitions
			\item Tensor operations
			\item Typical representations
			\item Subrepresentations
			\item Irreducible representations
			\item Decomposition of a representation (Maschke's theorem)
			\item Uniqueness of the decomposition (Schur's Lemma)
			\item Characters
		\end{itemize}
	\end{frame}
	
	
%	\begin{frame}{Initial definitions}
%%		\framesubtitle{Representations}
%		Let $G$ be a finite group and let $V$ be a finite-dimensional vector space over $\mathbb{C}$. \pause
%		
%		\begin{definition}
%			A representation of $G$ in $V$ is a homomorphism
%			\begin{align*}
%				\rho : G \rightarrow \GL(V),
%			\end{align*}
%			that assigns a linear map $\rho_g: V \rightarrow V$ to every $g \in G$.
%		\end{definition} \pause
%		
%		Homomorphism: 
%			$\rho_g \rho_h = \rho_{gh}$ \pause
%		
%		Some consequences:
%		$
%			\rho_e = \text{id}, \quad \text{and} \quad \rho_{g^{-1}} = \rho_g^{-1}.
%		$\pause
%		
%		$\dim V =$ the \textit{degree} of the representation.
%	\end{frame}
%	
%	\begin{frame}{Initial definitions}
%		\begin{definition}
%			A vector space provided with such a map is called a \textit{representation space of $G$}.
%		\end{definition} \pause
%		
%		\textit{Note:} $\rho$, the family $\{\rho_g\}_{g\in G}$ and $V$ are all referred to as the \textit{representation of $G$.} \pause
%		
%		\textit{Note:} A choice of basis gives us a \textit{matrix representation}. It is not canonical and dependent on this basis. 
%
%	\end{frame}
%	
	\begin{frame}{Tensor operations}
		Let $V$ and $W$ be vector spaces. Then:
		\begin{itemize}
			\item their direct sum $V \oplus W$ and
			\item their tensor product $V \otimes W$
		\end{itemize}
		are also vector spaces.
%		\begin{tabular}{c c c}
%			                  & $\ddot{\omega}$ &                             \\
%			$:\hspace{-3pt}3$ &                 & $\varepsilon\hspace{-3pt}:$ \\
%			                  &    $\varpi$             &
%		\end{tabular}
%
%		If $v = (v_i)_{i=1}^{\dim V} \in V$ and $w = (w_j)_{j=1}^{\dim W} \in W$, then:
%		\begin{itemize}
%			\item $v \oplus w = \begin{pmatrix}
%				v \\ w
%			\end{pmatrix} \in V \oplus W \quad \Ss (\dim V \oplus W = \dim V + \dim W)$
%			\item $v \otimes w = (v_iw)_{i=1}^{\dim V} \in V \otimes W \quad \Ss(\dim V \otimes W = \dim V \cdot \dim W)$
%		\end{itemize}
		
		
%		Also, $\dim V \oplus W = \dim V + \dim W$ and $\dim V \otimes W = \dim V \cdot \dim W$.
%
%	\end{frame}
%	\begin{frame}{Tensor operations}
%		Let $\rho^V: G \rightarrow \GL(V)$ and $\rho^W: G \rightarrow \GL(W)$ be representations, then 
%		\begin{itemize}
%			\item $\rho^{V \oplus W} = \rho^V \oplus \rho^W$ is a repr. in $V \oplus W$ by 
%			\begin{align*}
%			(	\rho^V \oplus \rho^W) \cdot (v \oplus w) &= \begin{pmatrix}
%				\rho^V & 0 \\ 0 & \rho^W
%			\end{pmatrix} \cdot \begin{pmatrix}
%			v \\ w
%			\end{pmatrix} \\
%			&= \begin{pmatrix}
%				\rho^V  v \\ \rho^W  w
%			\end{pmatrix} \\
%			&= (\rho^V v) \oplus (\rho^W w).
%			\end{align*}
%		\end{itemize}
%	\end{frame}
%	
%	\begin{frame}{Tensor operations}
%		Let $\rho^V: G \rightarrow \GL(V)$ and $\rho^W: G \rightarrow \GL(W)$ be representations, then 
%		\begin{itemize}
%			\item $\rho^V \otimes \rho^W$ is a repr. in $V \otimes W$ by
%			\begin{align*}
%				(\rho^V \otimes \rho^W) \cdot (v \otimes w) &= (\rho^V_{ij} \rho^W) \cdot (v_i w) \\
%				&= (\rho^V_{ij}v_i \rho^Ww) \\
%				&= (\rho^V v) \otimes (\rho^W w).
%			\end{align*}
%		\end{itemize}
%	\end{frame}
%	
	\begin{frame}{Typical representations}
		\begin{itemize}
			\item For any group, \textbf{Trivial representation}: $\rho_g = 1, \quad \forall g \in G$.
			\item In any vector space: $\rho_g = \id_n = \bigoplus_{i=1}^{n} (1)$.
			\item For $\Sym_n$, \textbf{Alternating representation}:
			\begin{align*}
				\rho_\sigma = \sgn(\sigma), \quad \forall \sigma \in \Sym_n.
			\end{align*}
			It is a homomorphism: for $\sigma, \tau \in \Sym_n$,
			\begin{align*}
				\sgn(\sigma)\sgn(\tau) &= (-1)^s(-1)^t \\
				&= (-1)^{s+t} \\
				&= \sgn(\sigma\tau).
			\end{align*}
		\end{itemize}
	\end{frame}
	
	\begin{frame}{Degree one representations of $\Cyc_n$}
		Recall $\Cyc_n = \{e, g, g^2, \dots, g^{n-1}\}$ where $g^n = e$.
		
		Consider $\rho: \Cyc_n \rightarrow \CC^\times$,
		\begin{align*}
			&\rho_e = 1 \quad \text{implies} \quad \rho_{g^n} = 1 \\
			&\text{but} \quad \rho_{g^n} = (\rho_g)^n \quad \text{by recursion on} \quad \rho_g\rho_g = \rho_{g^2} \\
			&\text{hence} \quad (\rho_g)^n = 1 \\
			&\text{and} \quad g \quad \text{is mapped to an $n$th root of unity.}
		\end{align*}
	\end{frame}
	
	\begin{frame}{Degree one representations of $\Cyc_n$}
		\begin{table}
			\parbox[s]{.45\linewidth}{
			\begin{table}
				\centering
				\begin{tabular}{c | c c c}
					$\Cyc_3$ & $e$ & $g$        & $g^2$      \\ \hline
					$\rho_0$          & 1   & 1          & 1          \\
					$\rho_1$          & 1   & $\omega$   & $\omega^2$ \\
					$\rho_2$          & 1   & $\omega^2$ & $\omega$
				\end{tabular} 
				\caption{$\omega = e^{2 \pi i/3}$.}
				\label{table:Cyc3}
				\begin{tabular}{c | c c cl}
					$\Cyc_4$ & $e$ & $g$  & $g^2$ & $g^3$ \\ \hline
					$\rho_0$           & 1   & 1    & 1     & 1     \\
					$\rho_1$           & 1   & $i$  & $-1$  & $-i$  \\
					$\rho_2$           & 1   & $-1$ & $1$  & $-1$   \\
					$\rho_3$           & 1   & $-i$ & $-1$   & $i$
				\end{tabular}
				\caption{$\omega = e^{\pi i/2}$.}
				\label{tbl:cyc4}
			\end{table}
			}
			\hfill
			\parbox[s]{.45\linewidth}{
				\begin{table}
					\centering
					\begin{tabular}{c | c c c c c}
						$\Cyc_5$ & $e$ & $g$        & $g^2$      & $g^3$      & $g^4$      \\ \hline
						$\rho_0$            & 1   & 1          & 1          & 1          & 1          \\
						$\rho_1$            & 1   & $\omega$ & $\omega^2$ & $\omega^3$ & $\omega^4$ \\
						$\rho_2$            & 1   & $\omega^2$ & $\omega^4$ & $\omega$ & $\omega^3$ \\
						$\rho_3$            & 1   & $\omega^3$ & $\omega$ & $\omega^4$ & $\omega^2$ \\
						$\rho_4$            & 1   & $\omega^4$ & $\omega^3$ & $\omega^2$ & $\omega$
					\end{tabular}
					\caption{$\omega = e^{2 \pi i/5}$.}
					\label{table:Cyc5}
				\end{table}
			}
		\end{table}
	\end{frame}
	
	\begin{frame}{Permutation representations}
		Let $G$ act on a set $X$ by permutation.
		
		Let $V = \Span (\bas_x)_{x \in X}$
		
		Then
		\begin{align*}
			\rho^\Perm &: G \rightarrow \GL(V) \\
			\rho^\Perm_g &: \bas_x \mapsto \bas_{gx}.
		\end{align*}
		is called the \textbf{permutation representation}.
		
		$\rho_g^\Perm$ are \textit{permutation matrices}
	\end{frame}
	
	\begin{frame}{Permutation representation of $\Sym_n$}
		content...
	\end{frame}
	
	\begin{frame}{$\Sym_3$}
		content...
	\end{frame}
	
	\begin{frame}{$\Sym_4$}
		content...
	\end{frame}
	
	\begin{frame}{Regular representation}
		Let $G$ act on itself ($X = G$).
		
		Basis $(\bas_g)_{g \in G}$.
		
		$\rho^\Reg : G \rightarrow \GL(V)$.
		
		For $G = \Cyc_3$:
		\begin{align*}
			\rho^\Reg_e = \id, \quad \rho^\Reg_g = \begin{pmatrix}
				0&0&1\\1&0&0\\0&1&0
			\end{pmatrix}, \quad \text{and} \quad \rho^\Reg_{g^2}\begin{pmatrix}
			0&1&0\\0&0&1\\1&0&0
			\end{pmatrix}.
		\end{align*}		
	\end{frame}
	
	\begin{frame}{$\Cyc_3$ cont'd.}
		Basis change...
	\end{frame}
	
	\begin{frame}{Subrepresentations}
		\begin{definition}
			A subrepresentation of a representation $\rho: G \rightarrow \GL(V)$ is a restriction of $\rho$ to a $G$-invariant subspace $W$ of $V$, that is
			\begin{align*}
				\rho^{V|_W}: G \rightarrow \GL(W)
			\end{align*}
			is a representation of $G$ in $W$.
		\end{definition}
		
		\begin{example}
			Every representation has itself and the zero space as trivial subrepresentations.
		\end{example}
	\end{frame}
	
	\begin{frame}{Subrepresentations}
		\begin{example}
			The trivial representation is a subrepresentation of the permutation representation.
			
			The sum of all basis vectors $\sum_{x \in X} \bas_x$ is fixed by any $g \in G$ (it is a permutation).
		\end{example}
		
		\begin{example}
			Likewise for the regular representation.
		\end{example}
	\end{frame}
	
	\begin{frame}
		content...
	\end{frame}
	
\end{document}