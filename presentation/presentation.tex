\documentclass[12pt]{beamer}
\mode<presentation>
\usepackage[utf8]{inputenc}
\usepackage[T1]{fontenc}
\usepackage{lmodern}
%\usetheme{metropolis}
\begin{document}
	\author{Daniel Martling}
	\title{Introduction to Representation Theory of Finite Groups}
	%\subtitle{}
	\logo{l\includegraphics[height=0.5cm]{logo.pdf}}
	%\institute{}
	\date{2024-06-12}
	%\subject{}
	\setbeamercovered{transparent}
	\setbeamertemplate{navigation symbols}{}
	\begin{frame}[plain]
		\maketitle
	\end{frame}
	
	\begin{frame}[plain]
		\frametitle{Initial definitions}
		\framesubtitle{Representations}
		Let $G$ be a finite group and let $V$ be a finite-dimensional vector space over $\mathbb{C}$. \pause
		
		\begin{definition}
			A representation of $G$ in $V$ is a homomorphism
			\begin{align*}
				\rho : G \rightarrow \text{GL} (V),
			\end{align*}
			that assigns a linear map $\rho_g$ to every $g \in G$.
		\end{definition} \pause
		
		\begin{definition}
			$\dim V$ is referred to as the \textit{degree} of the representation 
		\end{definition} \pause
		
		Some consequences:
		\begin{align*}
			\rho_e = \text{id}, \quad \text{and} \quad \rho_{g^{-1}} = \rho_g^{-1}.
		\end{align*}
		
		
		
	\end{frame}
	
	\begin{frame}
		\frametitle{Hej}
		\begin{align*}
			:\hspace{-3pt}3 \quad
			\varepsilon\hspace{-3pt}: 
		\end{align*}
		
		
	\end{frame}
\end{document}