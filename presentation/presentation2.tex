\documentclass[handout, 10pt]{beamer}
\usepackage[utf8]{inputenc}
\usepackage{mystyle} % See mystyle.sty for packages and own commands
%------------------------------------------------------------
%Title page
\title[Representation Theory]{Introduction to Representation Theory of Finite Groups}
%\subtitle{(Your Sub Title)}
\titlegraphic{\includegraphics[height=2.0cm]{Logos/widelogo.png}}
\author[Daniel Martling]{
	Daniel Martling}% ,
%	2nd Author,
%	3rd Author }
%\institute[]{Department of Physics \\Stockholm University}
\date{2024-06-12}

\newcommand{\GL}{\text{GL}}
\newcommand{\Sym}{\mathcal{S}}
\newcommand{\sgn}{\text{sgn}}

%---------------------TITLE PAGE---------------------------------------
\begin{document}
	\begin{frame}
		\maketitle
	\end{frame}
	%------------------------------------------------------------
	
	\logo{\includegraphics[height=1cm]{Logos/circlelogo.png}~%
	}
	
	
	%-------------------------------------------------------------------
	\section{Introduction}
	
	%-------------------------------------------------------------------
	
	\begin{frame}
		\frametitle{Table of Contents}
		\tableofcontents
	\end{frame}
	
	\section{Representation Theory}
	\subsection{Initial definitions}
	
	\begin{frame}{Initial definitions}
		Let $G$ be a finite group and let $V$ be a finite-dimensional vector space over $\mathbb{C}$. \pause
		
		\begin{definition}
			A representation of $G$ in $V$ is a homomorphism
			\begin{align*}
				\rho : G \rightarrow \GL(V),
			\end{align*}
			that assigns a linear map $\rho_g: V \rightarrow V$ to every $g \in G$.
		\end{definition} \pause
		
		\begin{itemize}
			\item Homomorphism: 
			$\rho_g \rho_h = \rho_{gh} \quad\pause \& \quad \rho_e = \text{id}, \quad \text{and} \quad \rho_{g^{-1}} = \rho_g^{-1}$.
			
			\item $\dim V =$ the \textit{degree} of the representation. 
			
			\item A vector space with such a map is called a \alert{representation space of $G$}.
			
			\item $\rho$, $\{\rho_g\}_{g\in G}$ and $V$ are all referred to as the \alert{representation of $G$.} \pause
			
			\item A choice of basis gives us a \alert{matrix representation}. It is not canonical and dependent on this basis. 
		\end{itemize}		
	\end{frame}
	
	\subsection{Tensor algebra}
	
	\begin{frame}{Tensor operations}
		\framesubtitle{Quick introduction}
		
		Let $V$ and $W$ be vector spaces.
		
		Let $v = (v_1, \dots, v_m)^T \in V$ and $w = (w_1, \dots, w_n)^T \in W$. 
		
		Then the following are also vector spaces:
		
		\vfill
		
		\begin{exampleblock}{Direct sum $V \oplus W$}
			\begin{itemize}
				\item $v \oplus w = \begin{pmatrix}
					v \\ w
				\end{pmatrix} = (v_1, \dots, v_m, w_1, \dots, w_n)^T \in V \oplus W$
				\item $\dim V \oplus W = \dim V + \dim W$
			\end{itemize}
		\end{exampleblock}
		\vfill
		\begin{exampleblock}{Tensor product $V \otimes W$}
			\begin{itemize}
				\item $v \otimes w = (v_iw)_{i=1}^{m} = (v_1 w_1, \dots, v_m w_1, v_1 w_2, \dots, v_m w_2, \dots, v_1 w_n, \dots, v_m w_n)^T \in V \otimes W$
				\item $\dim V \otimes W = \dim V \cdot \dim W$
			\end{itemize}
		\end{exampleblock}		
	\end{frame}
	
	\begin{frame}{Tensor operations on representation spaces}
		Let $\rho^V: G \rightarrow \GL(V)$ and $\rho^W: G \rightarrow \GL(W)$ be representations. 
		
%		Remember: The $\rho^V_g$ and $\rho^W_g$ are matrices. 
		
		Then the following are also representations:
		\vfill
		\begin{exampleblock}{Direct sum of representations}
			\begin{itemize}
				\item $\rho^{V \oplus W} = \rho^V \oplus \rho^W$ is the block matrix $\begin{pmatrix}
					\rho^V & 0 \\ 0 & \rho^W
				\end{pmatrix}$.
				
				\item It is a repr. given by 
				\begin{align*}
%					(	\rho^V \oplus \rho^W) \cdot (v \oplus w) &= 
					\begin{pmatrix}
						\rho^V & 0 \\ 0 & \rho^W
					\end{pmatrix} \cdot \begin{pmatrix}
						v \\ w
					\end{pmatrix} 
					= \begin{pmatrix}
						\rho^V  v \\ \rho^W  w
					\end{pmatrix} 
					= (\rho^V v) \oplus (\rho^W w).
				\end{align*}
			\end{itemize}
		\end{exampleblock}
		
		\begin{exampleblock}{Tensor product of representations}
			\begin{itemize}
				\item $\rho^{V \otimes W} = \rho^V \otimes \rho^W$ is the block matrix $(\rho^V_{ij} \rho^W)$.
				
				\item It is a repr. given by 
				\begin{align*}
				(\rho^V_{ij} \rho^W) \cdot (v_i w)
					= (\rho^V_{ij}v_i \rho^Ww)
					= (\rho^V v) \otimes (\rho^W w).
				\end{align*}
			\end{itemize}
		\end{exampleblock}
	\end{frame}
	
	\subsection{Examples}
	\begin{frame}{Examples}
		\begin{exampleblock}{Trivial representation}
			\begin{itemize}
				\item For any group, the trivial representation is a map
				\begin{align*}
					\rho_g = 1, \quad \text{for all $g \in G$.}
				\end{align*}
				
				\item Trivially a homomorphism.
				
				\item Degree = 1.
			\end{itemize}
			
		\end{exampleblock}
	\end{frame}
	\begin{frame}{Examples cont'd.}
		\begin{exampleblock}{Alternating representation}
			\begin{itemize}
				\item For $\Sym_n$, the alternating representation is the ``sign'' map
				\begin{align*}
						\sgn &: \Sym_n \rightarrow \{\pm 1\} \\
					\rho &: \sigma \mapsto \sgn(\sigma), \quad \text{for all } \sigma \in \Sym_n.
					%				\sgn &: \Sym_n \rightarrow \{\pm 1\} \\
					%				\sgn &: \sigma \mapsto (-1)^s,
				\end{align*}
				
				\item It is a homomorphism: for any $\sigma, \tau \in \Sym_n$,
				\begin{align*}
					\sgn(\sigma)\sgn(\tau) &= (-1)^s(-1)^t \\
					&= (-1)^{s+t} \\
					&= \sgn(\sigma\tau).
				\end{align*}
				
				\item Degree = 1.
			\end{itemize}
		\end{exampleblock}
	\end{frame}
	
	\begin{frame}{Examples cont'd.}
		\begin{exampleblock}{Permutation representation}
			Let $G$ be any finite group and let $X$ be a finite set.
		\end{exampleblock}
	\end{frame}
	
	
	
	%-------------------------------------------------------------------
	
	%---------------------EXEMPELSLIDES---------------------------------------
	\begin{frame}{Equations of motion} 
		
		\begin{columns}
			\column{0.4\textwidth}
			\begin{block}{Newton's second law}
				$m\ddot{x} = F(\dot{x}, x, t)$
			\end{block}
			\begin{block}{Schrödinger's equation}
				${\displaystyle i\hbar {\frac {d}{dt}}\vert \Psi (t)\rangle ={\hat {H}}\vert \Psi (t)\rangle }$
			\end{block}
			\begin{block}{Ampère's circuital law}
				$ \nabla \times \mathbf{B}=\frac{1}{c}\left(4 \pi \mathbf{J}+\frac{\partial \mathbf{E}}{\partial t}\right)$
			\end{block}
		\end{columns}
	\end{frame}
	%---------------------SLIDE---------------------------------------
	\begin{frame}{Important question}
		Important question again
	\end{frame}
	%---------------------SLIDE---------------------------------------
	\section{Important section}
	\begin{frame}{Bullet points}
		\begin{itemize}
			\item Item A
			\item Item B
			\item Item C
		\end{itemize}
	\end{frame}
	%---------------------SLIDE---------------------------------------
	\begin{frame}{Using pause}
		This is a sentence. \pause 
		And this too. \pause
		\alert{Bye}. 
	\end{frame}
	
	%---------------------SLIDE--------------------------------------
	\begin{frame}{A Theorem}
		\begin{theorem}[Freshman's Dream] 
			$(a+b)^p \equiv a^p + b^p \, (mod \,p)$ if p is a prime number.
		\end{theorem} \pause
		\begin{proof}
			A valid proof.
		\end{proof}
		\begin{example}
			Maybe an example?
		\end{example}
	\end{frame}
	
	%-----------------------------SLIDE --------------------------------------
	
	%Below must be compiled using LuLaTex (for emoji package):
	
	% \begin{frame}{Summary}
		
		% \begin{itemize}
			%     \item[\emoji{check-mark-button}] Concept A
			%      \item[\emoji{check-mark-button}] Concept B
			%      \item[\emoji{check-mark-button}] Concept C
			% \end{itemize}
		% \end{frame}
	
	
	%-----------------------------SLIDE --------------------------------------
	
	\begin{frame}{How do you write a thesis?}
		
		\begin{enumerate}
			\item Eat
			\item Sleep
			\item Rave
			\item Repeat
		\end{enumerate}
	\end{frame}
	
	
	%-----------------------------SLIDE --------------------------------------
	
%	\begin{frame}{Frame title}{Frame subtitle}
%		This is how you can cite \cite{Dirac}.
%	\end{frame}
	%-----------------------------SLIDE --------------------------------------
	\begin{frame}{The end.}
		\begin{columns}
			\column{0.4\textwidth}
			This is a column.
			
			\column{0.4\textwidth}
			\includegraphics[width=0.9\textwidth]{Figures/meme.png}    
		\end{columns}    
		
		
	\end{frame}
	
	%---------------------------------------------------------
%	\section{References}
%	\begin{frame}[allowframebreaks]\frametitle{References}
%		\bibliographystyle{apalike}
%		\bibliography{bib}
%	\end{frame}
	
	
\end{document}