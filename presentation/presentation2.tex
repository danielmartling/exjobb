\documentclass[10pt]{beamer}
\usepackage[utf8]{inputenc}
\usepackage{mystyle} % See mystyle.sty for packages and own commands
%------------------------------------------------------------
%Title page
\title[Representation Theory]{Introduction to Representation Theory of Finite Groups}
%\subtitle{(Your Sub Title)}
%\titlegraphic{\includegraphics[height=2.0cm]{Logos/widelogo.png}}
\author[Daniel Martling]{
	Daniel Martling}% ,
%	2nd Author,
%	3rd Author }
%\institute[]{Department of Physics \\Stockholm University}
\date{2024-06-12}

\usepackage{tikz, tikz-cd, quiver}

\newcommand{\GL}{\text{GL}}
\newcommand{\Sym}{\mathcal{S}}
\newcommand{\Cyc}{\mathcal{C}}
\newcommand{\sgn}{\text{sgn}}
\newcommand{\bas}{\mathbf{e}}
\newcommand{\fbas}{\mathbf{f}}
\newcommand{\id}{\text{id}}
\newcommand{\Span}{\text{Span}}
\newcommand{\CC}{\mathbb{C}}
\newcommand{\RR}{\mathbb{R}}
\newcommand{\wvec}{w}

\newcommand{\Reg}{\text{Reg}}
\newcommand{\Triv}{\text{Triv}}
\newcommand{\Alt}{\text{Alt}}
\newcommand{\Perm}{\text{Perm}}
\newcommand{\Stan}{\text{Stan}}

\newtheorem{proposition}{Proposition}[theorem]

%---------------------TITLE PAGE---------------------------------------
\begin{document}
	\begin{frame}
		\maketitle
	\end{frame}
	%------------------------------------------------------------
	
%	\logo{\includegraphics[height=1cm]{Logos/circlelogo.png}~%
%	}
	
	
	%-------------------------------------------------------------------
	\section{Introduction}
	
	%-------------------------------------------------------------------
	
	\begin{frame}
		\frametitle{Table of Contents}
		\tableofcontents
	\end{frame}
	
	\section{Representation Theory}
	\subsection{Initial definitions}
	
	\begin{frame}{Initial definitions}
		Let $G$ be a finite group and let $V$ be a finite-dimensional vector space over $\CC$.\pause
		
		\begin{definition}
			A representation of $G$ in $V$ is a homomorphism
			\begin{align*}
				\rho : G \rightarrow \GL(V),
			\end{align*}
			that assigns a linear map $\rho_g: V \rightarrow V$ to every $g \in G$.
		\end{definition}
	\end{frame}	
	
	\begin{frame}{Initial definitions}
		\begin{itemize}
			\item $\rho$ is a homomorphism: \begin{align*}
				\rho_g \rho_h &= \rho_{gh}, 
			\end{align*}\pause
			\begin{align*}
				\quad \rho_e &= \text{id}, \\
				\rho_{g^{-1}} &= (\rho_g)^{-1}.
			\end{align*}\pause
			\vfill
			\item A vector space with such a map is called a \alert{representation space of $G$}.\pause
			\vfill
			\item $\rho$, $\{\rho_g\}_{g\in G}$ and $V$ are all referred to as the \alert{representation of $G$.}\pause
			\vfill
			\item $\dim V =$ the \textit{degree} of the representation.\pause
			\vfill
			\item A choice of basis gives us a \alert{matrix representation}. It is not canonical and dependent on this basis.
		\end{itemize}		
	\end{frame}
	
	\subsection{Tensor operations}
	
	\begin{frame}{Tensor operations}
%		\framesubtitle{Quick introduction}
		
\begin{itemize}
	\item 		Let $V$ and $W$ be vector spaces.
		
%		\item Let $v = (v_1, \dots, v_m)^T \in V$ and $w = (w_1, \dots, w_n)^T \in W$. 
		
		\item Let $\rho^V: G \rightarrow \GL(V)$ and $\rho^W: G \rightarrow \GL(W)$ be representations.
\end{itemize}\pause

		\begin{block}{Direct sum $V \oplus W$}
			\begin{itemize}
%				\item $v \oplus w = \begin{pmatrix}
%					v \\ w
%				\end{pmatrix} = (v_1, \dots, v_m, w_1, \dots, w_n)^T \in V \oplus W$.
%				
%				\item $\dim V \oplus W = \dim V + \dim W$.
%				
%				\item $\rho^{V \oplus W} = \rho^V \oplus \rho^W$ is the block matrix $\begin{pmatrix}
%					\rho^V & 0 \\ 0 & \rho^W
%				\end{pmatrix}$.
%				
%				\item $\rho^{V \oplus W}: G \rightarrow \GL(V \oplus W)$ is given by 
%				\begin{align*}
%					%					(	\rho^V \oplus \rho^W) \cdot (v \oplus w) &= 
%					\begin{pmatrix}
%						\rho^V & 0 \\ 0 & \rho^W
%					\end{pmatrix} \cdot \begin{pmatrix}
%						v \\ w
%					\end{pmatrix} 
%					= \begin{pmatrix}
%						\rho^V  v \\ \rho^W  w
%					\end{pmatrix} 
%					= (\rho^V v) \oplus (\rho^W w).
%				\end{align*}
				\item $\rho^V \oplus \rho^W$ is a representation of $G$ in $V \oplus W$.
			\end{itemize}
		\end{block}\pause
		
%		\begin{example}[$\RR^3$]
%			\begin{align*}
%				\RR^3 = \RR \oplus \RR \oplus \RR
%			\end{align*}
%		\end{example}
		
%		\begin{block}{Direct sum of representations}
%			\begin{itemize}
%				\item $\rho^{V \oplus W} = \rho^V \oplus \rho^W$ is the block matrix $\begin{pmatrix}
%					\rho^V & 0 \\ 0 & \rho^W
%				\end{pmatrix}$.
%				
%				\item $\rho^{V \oplus W}: G \rightarrow \GL(V \oplus W)$ is a representation given by 
%				\begin{align*}
%					%					(	\rho^V \oplus \rho^W) \cdot (v \oplus w) &= 
%					\begin{pmatrix}
%						\rho^V & 0 \\ 0 & \rho^W
%					\end{pmatrix} \cdot \begin{pmatrix}
%						v \\ w
%					\end{pmatrix} 
%					= \begin{pmatrix}
%						\rho^V  v \\ \rho^W  w
%					\end{pmatrix} 
%					= (\rho^V v) \oplus (\rho^W w).
%				\end{align*}
%			\end{itemize}
%		\end{block}
		
%		\framebreak
		
		\begin{block}{Tensor product $V \otimes W$}
			\begin{itemize}
%				\item $v \otimes w = (v_iw)_{i=1}^{m} = (v_1 w_1, \dots, v_m w_1, v_1 w_2, \dots, v_m w_2, \dots, v_1 w_n, \dots, v_m w_n)^T \in V \otimes W$.
%				
%				\item $\dim V \otimes W = \dim V \cdot \dim W$.
%				
%				\item $\rho^{V \otimes W} = \rho^V \otimes \rho^W$ is the block matrix $(\rho^V_{ij} \rho^W)$.
%				
%				\item $\rho^{V \otimes W}: G \rightarrow \GL(V \otimes W)$ is given by 
%				\begin{align*}
%					(\rho^V_{ij} \rho^W) \cdot (v_i w)
%					= (\rho^V_{ij}v_i \rho^Ww)
%					= (\rho^V v) \otimes (\rho^W w).
%				\end{align*}
				\item $\rho^V \otimes \rho^W$ is a repesentation of $G$ in $V \otimes W$.
			\end{itemize}
		\end{block}		

		

		
		
%		\begin{block}{Tensor product of representations}
%			\begin{itemize}
%				\item $\rho^{V \otimes W} = \rho^V \otimes \rho^W$ is the block matrix $(\rho^V_{ij} \rho^W)$.
%				
%				\item It is a repr. given by 
%				\begin{align*}
%				(\rho^V_{ij} \rho^W) \cdot (v_i w)
%					= (\rho^V_{ij}v_i \rho^Ww)
%					= (\rho^V v) \otimes (\rho^W w).
%				\end{align*}
%			\end{itemize}
%		\end{block}
	\end{frame}
	
	\subsection{Examples}

	\begin{frame}{Examples}
		\begin{block}{Trivial representation}
			\begin{itemize}
				\item For any group, the trivial representation is a map
				\begin{align*}
					\rho_g = 1, \quad \text{for all $g \in G$.}
				\end{align*}
				
				\item Trivially a homomorphism.
			\end{itemize}
			
		\end{block}
	\end{frame}
	\begin{frame}{Examples}
%		\begin{block}{Symmetric group $\Sym_n$}
%			\begin{itemize}
%				\item Set of permutations of $\{1, 2, \dots, n\}$.
%				
%				\item Group under composition of permutations. 
%				
%				\item $|\Sym_n| = n!$.
%			\end{itemize}
%		\end{block}

		\begin{block}{Alternating representation}
			\begin{itemize}
				\item For $\Sym_n$, recall the ``sign'' map
				\begin{align*}
%						\sgn &: \Sym_n \rightarrow \{\pm 1\} \\ 
					\sgn &: \sigma \mapsto \{\pm 1\}, \quad \text{for any } \sigma \in \Sym_n.
					%				\sgn &: \Sym_n \rightarrow \{\pm 1\} \\
					%				\sgn &: \sigma \mapsto (-1)^s,
				\end{align*}\pause
				
				\item It is a homomorphism: for any $\sigma, \tau \in \Sym_n$,
				\begin{align*}
					\sgn(\sigma)\sgn(\tau) &= (-1)^s(-1)^t \\
					&= (-1)^{s+t} \\
					&= \sgn(\sigma\tau).
				\end{align*}\pause
				
				\item $\rho_\sigma = \sgn(\sigma)$ is a representation.
			\end{itemize}
		\end{block}
	\end{frame}
	\begin{frame}{Examples}
%		\begin{block}{Cyclic group $\Cyc_n$}
%			\begin{itemize}
%				\item $\Cyc_n = \{ e, g, g^2, \dots, g^{n-1}\}$.
%				
%				\item $g \cdot g^a = g^{a+1}$ and $g^n = e$.
%			\end{itemize}
%		\end{block}
%		
		\begin{block}{Degree 1 representations of $\Cyc_n$}
			\begin{itemize}
				\item Consider $\rho: \Cyc_n \rightarrow \CC^\times$.\pause
				
				\item For $g$, $g^n = e$, hence $\rho_{g^n} = 1$. 
				
				\item Also, $(\rho_g)^n = \rho_{g^n}$, hence $(\rho_g)^n = 1$.\pause
				
				\item That is: $g$ is mapped to $n$th root of unity.
			\end{itemize}
		\end{block}\pause
		
		\begin{example}[$\Cyc_3$]
			The third roots of unity are $1$, $\omega = \frac{-1+i\sqrt{3}}{2}$ and $\omega^2 = \frac{-1-i\sqrt{3}}{2}$.
			\begin{table}
				\centering
				\begin{tabular}{c | c c c}
					$\Cyc_3$ & $e$ & $g$        & $g^2$      \\ \hline
					$\rho_0$          & 1   & 1          & 1          \\
					$\rho_1$          & 1   & $\omega$   & $\omega^2$ \\
					$\rho_2$          & 1   & $\omega^2$ & $\omega$
				\end{tabular}
			\end{table}
		\end{example}
		
%		\begin{example}[$\Cyc_4$]
%			The fourth roots of unity are $1,i,-1$ and $-i$.
%			\begin{table}
%				\centering
%				\begin{tabular}{c | c c c c}
%					$\Cyc_4$ & $e$ & $g$  & $g^2$ & $g^3$ \\ \hline
%					$\rho_0$           & 1   & 1    & 1     & 1     \\
%					$\rho_1$           & 1   & $i$  & $-1$  & $-i$  \\
%					$\rho_2$           & 1   & $-1$ & $1$  & $-1$   \\
%					$\rho_3$           & 1   & $-i$ & $-1$   & $i$
%				\end{tabular}
%			\end{table}
%		\end{example}
		
%		\begin{example}[$\Cyc_5$]
%			The fifth roots of unity are $e^{{2\pi im}/{5}}$, $0 \leq m \leq 4$.
%			\begin{table}
%				\centering
%				\begin{tabular}{c | c c c c c}
%					$\Cyc_5$ & $e$ & $g$        & $g^2$      & $g^3$      & $g^4$      \\ \hline
%					$\rho_0$            & 1   & 1          & 1          & 1          & 1          \\
%					$\rho_1$            & 1   & $\omega$ & $\omega^2$ & $\omega^3$ & $\omega^4$ \\
%					$\rho_2$            & 1   & $\omega^2$ & $\omega^4$ & $\omega$ & $\omega^3$ \\
%					$\rho_3$            & 1   & $\omega^3$ & $\omega$ & $\omega^4$ & $\omega^2$ \\
%					$\rho_4$            & 1   & $\omega^4$ & $\omega^3$ & $\omega^2$ & $\omega$
%				\end{tabular}
%			\end{table}
%		\end{example}
	\end{frame}
	\begin{frame}{Examples}
		\begin{block}{Permutation representation}
			\begin{itemize}
				\item Let $G$ be a finite group and let $X$ be a finite set.
				
				\item $G$ acts on $X$ by permutation.\pause
%				\begin{align*}
%					\sigma &: G \times X \rightarrow X, \\
%					\sigma_g &: x \mapsto gx.
%				\end{align*}
				
				\item Let the basis $(\bas_x)_{x \in X}$ span $V$.\pause
				
				\item Then the permutation representation is
				\begin{align*}
					\rho &: G \rightarrow \GL(V), \\
					\rho_g &: \bas_x \mapsto \bas_{gx}.
				\end{align*}\pause
				
				\item The matrices $\rho_g$ are \alert{permutation matrices}.
			\end{itemize}
		\end{block}
		
	\end{frame}
	\begin{frame}{Examples}
		
		\begin{block}{Symmetric group $\Sym_n$}
			\begin{itemize}
				\item Let $G = \Sym_n$ act on $X = \{1, 2, \dots, n\}$.
				
				\item $\rho : \Sym_n \rightarrow \GL(V)$ is defined by $\rho_\sigma : \bas_i \rightarrow \bas_{\sigma(i)}$.
				
%				\item $\rho_\sigma = (r_{ij})_{n \times n}$, where $r_{ij} = \begin{cases}
%					1, \quad \text{if} \quad j = \sigma(i), \\
%					0, \quad \text{otherwise.}
%				\end{cases}$
			\end{itemize}
		\end{block}\pause
		\begin{example}[Matrix representations of $\Sym_3$]
%			\begin{table}[hbt!]
				\centering
				\begin{tabular}{r r r}
					$\rho_{(1)} = 
					\begin{pmatrix}
						1 & 0 & 0 \\
						0 & 1 & 0 \\
						0 & 0 & 1
					\end{pmatrix}$, & 
					$\rho_{(1,2,3)} = 
					\begin{pmatrix}
						0 & 0 & 1 \\
						1 & 0 & 0 \\
						0 & 1 & 0
					\end{pmatrix}$, & 
					$\rho_{(1,3,2)} = 
					\begin{pmatrix}
						0 & 1 & 0 \\
						0 & 0 & 1 \\
						1 & 0 & 0
					\end{pmatrix}$, \\ & & \\
					$\rho_{(1,2)} = 
					\begin{pmatrix}
						0 & 1 & 0 \\
						1 & 0 & 0 \\
						0 & 0 & 1
					\end{pmatrix}$, &
					$\rho_{(1,3)} = 
					\begin{pmatrix}
						0 & 0 & 1 \\
						0 & 1 & 0 \\
						1 & 0 & 0
					\end{pmatrix}$, &
					$\rho_{(2,3)} = 
					\begin{pmatrix}
						1 & 0 & 0 \\
						0 & 0 & 1 \\
						0 & 1 & 0
					\end{pmatrix}$.
				\end{tabular}
%				\caption{Matrix representations of $\Sym_3$}
%				\label{table:permS3}
%			\end{table}
		\end{example}

%		\begin{example}[Matrix representations of $\Sym_4$]
%		
%%		\begin{table}[hbt\Perm!]
%			\centering
%			\begin{tabular}{ r }
%				$\rho_{(1)} = \left(\begin{matrix}
%					1 & 0 & 0 & 0 \\
%					0 & 1 & 0 & 0 \\
%					0 & 0 & 1 & 0 \\
%					0 & 0 & 0 & 1
%				\end{matrix}\right)$,  \\ \\
%			\end{tabular}
%			\begin{tabular}{ r r }
%				$\rho_{(1,2)} = \left(\begin{matrix}
%					0 & 1 & 0 & 0 \\
%					1 & 0 & 0 & 0 \\
%					0 & 0 & 1 & 0 \\
%					0 & 0 & 0 & 1
%				\end{matrix}\right)$, &
%				$\rho_{(1,2)(3,4)} = \left(\begin{matrix}
%					0 & 1 & 0 & 0 \\
%					1 & 0 & 0 & 0 \\
%					0 & 0 & 0 & 1 \\
%					0 & 0 & 1 & 0
%				\end{matrix}\right)$, \\ & \\
%
%				$\rho_{(1,2,3)} = \left(\begin{matrix}
%					0 & 0 & 1 & 0 \\
%					1 & 0 & 0 & 0 \\
%					0 & 1 & 0 & 0 \\
%					0 & 0 & 0 & 1
%				\end{matrix}\right)$, &
%				$\rho_{(1,2,3,4)} = \left(\begin{matrix}
%					0 & 0 & 0 & 1 \\
%					1 & 0 & 0 & 0 \\
%					0 & 1 & 0 & 0 \\
%					0 & 0 & 1 & 0
%				\end{matrix}\right)$.
%			\end{tabular}
%%			\caption{Some matrix representations of $\Sym_4$}
%%			\label{table:permS4}
%%		\end{table}
%\end{example}
	\end{frame}
	\begin{frame}{Examples}
		\begin{block}{Regular representation}
			\begin{itemize}
				\item Special case: Let $X = G$. \textit{($G$ acts on itself.)}\pause
				
				\item The regular representation is a map
				\begin{align*}
					\rho &: G \rightarrow \GL(V), \\
					\rho_g &: \bas_h \mapsto \bas_{gh}.
				\end{align*}
			\end{itemize}
		\end{block}\pause
		

		\begin{example}[Matrix representations of $\Cyc_3$]
			\begin{itemize}
				\item In $e, g, g^2$-basis:
%				\begin{align*}
	$
					\rho_e = \id, \quad \rho_g = \begin{pmatrix}
						0&0&1 \\ 1&0&0 \\ 0&1&0
					\end{pmatrix}, \quad \rho_{g^2} = \begin{pmatrix}
						0&1&0 \\ 0&0&1 \\ 1&0&0 
					\end{pmatrix}.
					$
%				\end{align*}
				
%				\item Switch basis: $\begin{cases}
%						\fbas_1 = e + g + g^2, \\
%						\fbas_2 = e + \omega^2 g + \omega g^2, \\
%						\fbas_3 = e + \omega g + \omega^2 g^2.
%					\end{cases}$
%				
%				\item Then:
%%				\begin{align*}
%	$
%					\rho_e = \id, \quad
%					\rho_g = \begin{pmatrix}
%						1&0&0 \\ 0&\omega&0 \\ 0&0&\omega^2
%					\end{pmatrix}, \quad \text{and} \quad
%					\rho_{g^2} = \begin{pmatrix}
%						1&0&0 \\ 0&\omega^2&0 \\ 0&0&\omega
%					\end{pmatrix}.
%					$
%%				\end{align*}
%
%				\item Clearly, $\rho_g = (1) \oplus (\omega) \oplus (\omega^2) = \rho_g^0 \oplus \rho_g^1 \oplus \rho_g^2$.
				
%				\item It is the direct sum of the degree one repr. found earlier.\pause
%				
%				\item That is, \begin{align*}
%					\{\text{Regular repr.}\} = \bigoplus_i \{\text{Deg. 1 repr.}\}
%				\end{align*}
			\end{itemize}
			
		\end{example}
	\end{frame}
	
	\subsection{Subrepresentations and irreducible representations}
	\begin{frame}{Subrepresentations and irreducible representations}
			
		\begin{definition}[$G$-invariant subspace]
			A vector subspace $W$ of $V$ is called $G$-invariant if it is left fixed by the action of $G$.
		\end{definition}\pause

%		\begin{block}{Now:}
%			\begin{itemize}
%				\item Let $\rho^V : G \rightarrow \GL(V)$.
%				
%				\item Let $W$ be a $G$-invariant subspace of $V$.
%				
%				\item The restriction of $\rho^V$ to $W$, $\rho^{V|_W}$ is a isomorphism of $W$ onto itself.
%				
%				\item Hence, $\rho^{V|_W}: G \rightarrow \GL(W)$ is a representation of $G$ in $W$.
%			\end{itemize}
%		\end{block}
		
		\begin{definition}[Subrepresentation]
			A restriction of a $\rho: G \rightarrow \GL(V)$ to a $G$-invariant subspace $W \leq V$ is called a subrepresentation of $\rho$.
		\end{definition}\pause
		
		\begin{example}[Trivial subspaces]
			Any representation has itself and the zero space as trivial, non-proper subrepresentations.
		\end{example}
	\end{frame}
%	\begin{frame}{Subrepresentations and irreducible representations}
%		\begin{example}[Regular representation of $\Cyc_3$]
%			\begin{itemize}
%				\item In $e, g, g^2$-basis:
%				%				\begin{align*}
%					$
%					\rho_e = \id, \quad \rho_g = \begin{pmatrix}
%						0&0&1 \\ 1&0&0 \\ 0&1&0
%					\end{pmatrix}, \quad \rho_{g^2} = \begin{pmatrix}
%						0&1&0 \\ 0&0&1 \\ 1&0&0 
%					\end{pmatrix}.
%					$\pause
%					%				\end{align*}
%				
%
%				
%								\item It is the direct sum of the degree one repr. found earlier.
%								
%								\item That is, \begin{align*}
%										\{\text{Regular repr.}\} = \bigoplus_i \{\text{Deg. 1 repr.}\}
%									\end{align*}
%			\end{itemize}
%		\end{example}
%	\end{frame}
	\begin{frame}{Subrepresentations and irreducible representations}		
		\begin{definition}[Irreducible representation]
			If there are no $G$-invariant subspaces of a representation other than itself and the zero space, than it is called \alert{irreducible}.
		\end{definition}\pause
	
		\begin{example}[Degree one representations are irreducible]
			\begin{itemize}
				\item Any vector space of dimension one has no proper subspaces, hence they are irreducible.
				
				\item The trivial and alternating representations.
				
				\item The degree 1 representations of $\Cyc_n$.
			\end{itemize}
		\end{example}
	\end{frame}
	\begin{frame}{Subrepresentations and irreducible representations}
		\begin{example}[Trivial repr. inside permutation repr.]
			\begin{itemize}
				\item Let $V$ be the permutation representation. 
				
				\item Consider:
				\begin{align*}
					W = \Span\left\lbrace \sum_i \bas_i \right\rbrace.
				\end{align*}\pause
				
				\item Any $g \in G$ will reorder the sum, hence $W$ is $G$-invariant, and $\rho_g = 1$.
				
%				\item Hence, the trivial representation is a subrepresentation of the permutation representation (and the regular representation).
			\end{itemize}		
		\end{example}
	\end{frame}
%	\begin{frame}{Subrepresentations and irreducible representations}
%		\begin{example}[Alternating repr. inside regular repr. of $\Sym_n$]
%			\begin{itemize}
%				\item Let $V$ be the regular representation of $\Sym_n$.
%				
%				\item Consider $w = \sum_{\sigma \in \Sym_n} \sgn(\sigma)\bas_\sigma$.\pause
%				
%				\item The action of a $\tau \in \Sym_n$ on $w$ is $\sgn(\tau) w$.\pause
%				
%				\item The 1-dimensional spanned by  $w$ is a $\Sym_n$-invariant subspace.
%			\end{itemize}
%		\end{example}
%	\end{frame}
	
	\begin{frame}{Subrepresentations as a kernel}
		
		\begin{itemize}
			\item We want to ``divide'' or ``decompose'' a representations into two subrepresentations.
		\end{itemize}\pause
		
		\begin{definition}[$G$-linear map]
			A vector space map $\varphi : V \rightarrow W$ is called $G$-linear if it commutes with the group action of $G$.
			\begin{columns}
				\column{0.4\linewidth}
				\[\begin{tikzcd}[ampersand replacement=\&]
					V \&\& W \\
					\\
					V \&\& W
					\arrow["\varphi", from=1-1, to=1-3]
					\arrow["g"', from=1-1, to=3-1]
					\arrow["g", from=1-3, to=3-3]
					\arrow["\varphi", from=3-1, to=3-3]
				\end{tikzcd}\]
				\column{0.4\linewidth}
				\begin{align*}
					\varphi(g \cdot v) = g \cdot \varphi(v)
				\end{align*}
			\end{columns}
			% https://q.uiver.app/#q=WzAsNCxbMCwwLCJWIl0sWzAsMiwiViJdLFsyLDIsIlciXSxbMiwwLCJXIl0sWzAsMywiXFx2YXJwaGkiXSxbMSwyLCJcXHZhcnBoaSJdLFswLDEsImciLDJdLFszLDIsImciXV0=
		\end{definition}\pause
		
		\begin{proposition}
			Let $V$ and $W$ be representation spaces of some group $G$.
			
			Let $\varphi: V \rightarrow W$ be a $G$-linear map. \pause
			
			Then $\ker \varphi \leq V$ and $\text{im } \varphi \leq W$ are subrepresentations.
		\end{proposition}
	\end{frame}
	\begin{frame}{Subrepresentations as a kernel}
		\begin{proposition}
			Let $V$ be a representation space and let $W$ be a subrepresentation of $V$.
			
			Then there exists a subrepresentation $W'$ of $V$ complementary to $W$ such that \begin{align*}
				V = W \oplus W'.
			\end{align*}
		\end{proposition}\pause
%	\end{frame}
%	\begin{frame}{Subrepresentations as a kernel}
%		\begin{proof}
%			\begin{itemize}
%				\item Let $\pi: V \rightarrow W$ be a projection.\pause \textit{\alert{Not necessarily $G$-linear!}}\pause
%				\item But the \alert{average}\begin{align*}
%					\bar{\pi} = \frac{1}{|G|}\sum_{g \in G} \rho_g \cdot \pi \cdot \rho_g^{-1}
%				\end{align*} is a G-linear map, and also a projection $V \rightarrow W$.
%				
%				\item $\text{im } \varphi = W$. \pause
%				
%				\item Then $\ker \bar{\pi}$ is a subrepresentation of $V$ complementary to $W$, ie. \begin{align*}
%					V = W \oplus \ker \bar{\pi}.
%				\end{align*}
%			\end{itemize}
%		\end{proof}
%	\end{frame}
%	\begin{frame}{Subrepresentations as a kernel}
		\begin{example}[Standard representation of $\Sym_3$]
			\begin{itemize}
				\item $V$ = Permutation representation of $\Sym_3$.
				
				\item $W$ = Trivial representation of $\Sym_3$, 
				\begin{align*}
					W = \Span \left\lbrace \bas_1 + \bas_2 + \bas_3 \right\rbrace.
				\end{align*}\pause
				
				\item Then $W'$ is the complement of $W$ in $V$, 
				\begin{align*}
					W' = \Span \left\lbrace \bas_1 - \bas_2 , \bas_2 - \bas_3 \right\rbrace.
				\end{align*}
								
%				\item This is called the \alert{standard representation of $\Sym_3$}.
			\end{itemize}
		\end{example}
%		Example Decomp $\Sym_3$
		
	\end{frame}
	\begin{frame}{Complete decomposition}
		\begin{corollary}[Maschke's theorem]
			Any representation $V$ is a direct sum of a finite number of irreducible subrepresentations. \begin{align*}
				V = \bigoplus_{i=1}^k W_i
			\end{align*}
		\end{corollary}\pause
%		\begin{proof}
%			\begin{itemize}
%				
%				\item $V$ is finite-dimensional
%				
%				\item Proof follows by induction on complementary subrepresentations.
%			\end{itemize}
%		\end{proof}
				
		\begin{theorem}[Schur's lemma]
			\begin{itemize}
				\item Let $V$ and $W$ be irreducible representations of $G$.
				\item Let $\varphi: V \rightarrow W$ be a $G$-linear map.\pause
				\item Then we have: 
				\begin{itemize}
					\item[i)] Either $\varphi$ is an isomorphism, or $\varphi$ is the zero map.\pause
					\item[ii)] If it is an isomorphism, then $\varphi = \lambda \cdot \id$, for some $\lambda \in \CC$.
				\end{itemize}
				
			\end{itemize}
		\end{theorem}
	\end{frame}
	\begin{frame}{Complete decomposition}
		\begin{proof}
			\begin{itemize}
				\item[i)] \begin{itemize}
					\item $\ker\varphi$ is a subrepresentation of $V$.\pause
					\item But $V$ is irreducible
					\item Hence $\ker \varphi = \begin{cases}
						V, \quad \text{(and $\varphi$ is the zero map),} \\
						\{0\}, \quad \text{(and $\varphi$ is injective)}.
					\end{cases}$\pause
					\item Likewise, $\text{im } \varphi = \begin{cases}
						\{0\}, \quad \text{(and $\varphi$ is the zero map)}, \\
						W, \quad \text{(and $\varphi$ is surjective)}.
					\end{cases}$\pause
					\item In conclusion: $\varphi$ is either an isomorphism or the zero map.
				\end{itemize}\pause
				\item[ii)] \begin{itemize}
					\item Consider $V = W$ and let $\lambda \in \CC$ be an eigenvalue of $\varphi$.\pause
					
					\item The map $\varphi - \lambda \cdot \id$ is also a $G$-linear map.\pause
					
					\item But its kernel is non-empty (contains eigenvector of $\lambda$).\pause
					
					\item Hence $\varphi - \lambda \cdot \id = 0$, or $\varphi = \lambda \cdot \id$. \qedhere
				\end{itemize}
			\end{itemize}
		\end{proof}
	\end{frame}
	\begin{frame}{Complete decomposition}
		\begin{block}{Corollary of Schur's Lemma}
			If $V$ is an arbitrary representation of some group, then it is decomposed into the direct sum of a finite number of irreducibles $W_i$ by 
			\begin{align*}
				V &= m_1W_1 \oplus m_2W_2 \oplus \dots \oplus m_kW_k \\
				&= \bigoplus_i m_i W_i
			\end{align*}
			Where $m_i$ is the multiplicity of $W_i$ in $V$.
		\end{block}
	\end{frame}
	
	
	
	\section{Character Theory}
	
	\begin{frame}{Character Theory}
		\begin{itemize}
						
			\item Let $V$ be a representation of degree $n$ of a group $G$. 
			
			\item Let $\rho_g = (r_{ij})$ be the matrix representation of a $g \in G$.\pause
	
		\end{itemize}
		\begin{definition}[Character]
			The \alert{character} of $g$ in $V$ is the trace of the matrix representation:
			\begin{align*}
				\chi_V(g) := \text{Tr } \rho_g = \sum_{i=1}^{n} r_{ii}.
			\end{align*}
		\end{definition}
	\end{frame}
	
	\begin{frame}{Properties of characters}
		\begin{block}{Recall the trace function}
			\begin{itemize}
				\item It is the sum of eigenvalues.
				
				\item Similar matrices have the same trace.
				
				\item It is independent of basis
				
				\item It is constant under conjugation
			\end{itemize}
		\end{block}\pause
		\begin{proposition}{Properties of the character function}
			\begin{itemize}
				
				\item $\chi(e) = \dim(V)$.\pause
				
				\item $\chi(g^{-1}) = \overline{\chi(g)}$.\pause
				
				\item If $g,h \in G$ are conjugate, then $\chi(g) = \chi(h)$.
				
			\end{itemize}
		\end{proposition}
	\end{frame}
	\begin{frame}{Group characters}
		\begin{definition}[Group character]
			\begin{itemize}
				\item The tuple
			\begin{align*}
				\chi_V = \left(\chi(g)\right)_{g \in G}
			\end{align*}
			is called the \alert{(group) character} of $G$ in $V$.\pause
			
				\item Usually abbreviated to one representative from every conjugacy class of $G$.
			\end{itemize}
		\end{definition}\pause
		
		\begin{proposition}[Tensor operations]
			Let $\chi_V$ and $\chi_W$ be characters of some group in the spaces $V$ and $W$.\pause
			
			\begin{itemize}
				\item The character in $V \oplus W$ is $\chi_V + \chi_W$,\pause
				\item The character in $V \otimes W$ is $\chi_V \cdot \chi_W$.
			\end{itemize}
		\end{proposition}
	\end{frame}
	\begin{frame}{Character tables}
		
		
		\begin{definition}[Character table]
			\begin{table}[hbt!]
				
				\centering
				\parbox[t]{.45\linewidth}{
					\centering
					\begin{tabular}{c | c c c }
						$G$    & $\cdots $ & $[g]$        & $\cdots$ \\ 
						$|[g]|$ & $\cdots$ & $|[g]|$ & $\cdots$ \\ \hline
						Triv.   & $\cdots$  & 1            & $\cdots$ \\
						$\vdots$ &           & $\vdots$     &          \\
						$V$    & $\cdots$  & $\chi_V(g) $ & $\cdots$ \\
						$\vdots$ &           & $\vdots$     &
					\end{tabular}
					%					\caption{Layout of a character table of a group $G$.}
					%					\label{table:chartableexample}
				}\pause
				\hfill
				\parbox[t]{.45\linewidth}{
					\centering
						\begin{tabular}{c | c c c }
							$\Sym_3$  & $[(1)]$ & $[(1,2)]$ & $[(1,2,3)]$ \\ 
							$|[\sigma]|$ & $1$ & $3$ & $2$ \\ \hline
							Triv.. & $1$     & 1         & $1$       \\
							Alt.  & 1       & $-1$      & 1         \\
							Perm. & $3$     & $1$       & $0$       \\
							Stan. & 2       & 0         & -1        
							%	Reg.  & 4       & 0         & 0
						\end{tabular}
					%					\caption{Character table of $\Sym_3$.}
					%					\label{table:chartableSym3}
				}
			\end{table}
		\end{definition}
	
%		\begin{block}{Character calculus}
%			Addition and multiplication of characters are defined by component-wise addition or multiplication, ie.\pause
%			\begin{itemize}
%				\item $\chi + \psi = \left(\chi(g) + \psi(g)\right)_{g \in G},$ \pause
%				\item $\chi \cdot \psi = \left(\chi(g) \cdot \psi(g)\right)_{g \in G}$.
%			\end{itemize}
%		\end{block}\pause
		
		
	\end{frame}
	
	\begin{frame}{Orthogonality of characters}		
		\begin{definition}[Inner product of characters]\pause
			\begin{itemize}
				\item Let $\varphi$ and $\psi$ be characters of some group $G$.\pause
				\item Then their \alert{inner product} is \begin{align*}
					(\varphi|\psi) := \frac{1}{|G|}\sum_{g \in G}\overline{\varphi(g)}\psi(g),
				\end{align*}
				fulfilling the expected properties.\pause
			\end{itemize}
		\end{definition}
		
		\begin{theorem}[Irreducibility criterion]
			\begin{itemize}
%				\item A representation is irreducible if and only if its character $\varphi$ is such that \begin{align*}
%				(\varphi|\varphi) = 1.
%			\end{align*}\pause
			
				\item Characters of irreducible representations are orthonormal.
				
				\item Let $\chi, \psi$ be irreducible characters. Then \begin{align*}
					(\chi|\psi) = \begin{cases}
						1, \quad \chi = \psi, \\
						0, \quad \chi \neq \psi.
					\end{cases}
				\end{align*}
			\end{itemize}
		\end{theorem}
	\end{frame}
	\begin{frame}{Decomposition of representations}
		\begin{block}{Character of an arbitrary representation.}
			\begin{itemize}
				
				\item Let $\{W_i\}$ be every irreducible reprs. of a group.
				
				\item Let $\{\chi_i\}$ be their characters.
				
				\item Let $V$ be any representation of the group, then \begin{align*}
					V = \bigoplus_i m_i W_i.
				\end{align*}
				
				\item The character in $V$ is \begin{align*}
					\chi_V = \sum_i m_i \chi_i.
				\end{align*}
  				
				\item Then we have: \begin{align*}
					(\chi_V|\chi_V) = \sum_i m_i^2.
				\end{align*}
			\end{itemize}
		\end{block}
	\end{frame}
	
	\begin{frame}{Decomposition of representations}
		\begin{theorem}
			The character $\chi$ is irreducible if and only if $(\chi|\chi) = 1$.
		\end{theorem}
		
		\begin{example}[Standard representation is irreducible]
			For $\Sym_3$: 
			\begin{itemize}
				\item $\chi_\Stan = (2,0,-1)$.
				\item $(\chi_\Stan|\chi_\Stan) = \frac{1}{6}\left(2^2 + 3 \cdot 0 + 2 \cdot (-1)^2\right) = 1$.
			\end{itemize}
		\end{example}
		
	\end{frame}	
%	\begin{frame}{Complete decomposition}
%		\begin{block}{Remark}
%			The decomposition of a representation into direct sum of irreducibles is unique up to isomorphism.
%			
%			Two compositions of the same representations would be equal.
%			
%			Same characters - contain same irreducibles.
%		\end{block}
%		\begin{block}{Inner product of itself}
%			Let $\varphi = \bigoplus_i m_i W_i$. Then we have that:
%			\begin{align*}
%				(\varphi|\varphi) = m_i^2
%			\end{align*}
%		\end{block}
%	\end{frame}
	
	\begin{frame}{Decomposition of the regular representation}
		\begin{block}{Character of regular representation}
			\begin{itemize}
%			\item The matrices of the regular representation are permutation matrices.
			
%			\item The character is the sum along the diagonal.
			
			\item The diagonal elements are 1 if and only if it represents a fixed point for every $g \in G$.\pause
			
%			\item In $\rho_g$, the ``$h$th'' column in will have a 1 in ``$h$th'' row if and only if $gh=h$, which holds only for $g = e$.\pause
			
			\item Hence: 
			\begin{align*}
				\chi_\Reg(g) = \begin{cases}
					\dim V = |G|, \quad \text{if $g=e$},\\
					0, \quad \text{otherwise}.
				\end{cases}
			\end{align*}\pause
			
			\item And $(\chi_\Reg|\chi_\Reg) = |G|$.
			
			
%			\item Which leads us to: $m_i^2 = |G|$.
		\end{itemize}
		\end{block}
%		
	\end{frame}
	\begin{frame}{Decomposition of the regular representation}
		\begin{block}{Multiplicities in regular representation}
			\begin{itemize}
				\item Let $ V_\Reg = \bigoplus_i m_i W_i$.\pause
				
				\item The multiplicity of some $W_j$ in $V_\Reg$ is $m_j = (\chi_\Reg|\chi_j) \pause = \dim W_j$.\pause
				
				\item Hence $V_\Reg = \bigoplus_i \dim(W_i)W_i$.
			\end{itemize}
		\end{block}\pause
		\begin{block}{Square sum of multiplicities}
						\begin{itemize}
								\item But then \begin{align*}
										\sum_i (\dim W_i)^2 = |G|.
									\end{align*}
								\item The square sum of the degrees of every irreducible representation of a group is the order of the group.
							\end{itemize}
					\end{block}
		
	\end{frame}
	
	\begin{frame}[allowframebreaks]{Characters of cyclic groups}
		
		\begin{itemize}
			\item Earlier we found $n$ degree one representations of $\Cyc_n$.
			\item Hence those are all irreducible representations of $\Cyc_n$.\pause
		\end{itemize}
		\begin{example}[$\Cyc_3$]
			\begin{table}
				\centering
				\begin{tabular}{c | c c c}
					$\Cyc_3$ & $e$ & $g$        & $g^2$      \\ \hline
					$\chi_0$          & 1   & 1          & 1          \\
					$\chi_1$          & 1   & $\omega$   & $\omega^2$ \\
					$\chi_2$          & 1   & $\omega^2$ & $\omega$ \\ \hline\hline
					$\chi_\text{Reg}$ & 3 & 0 & 0
				\end{tabular}
			\end{table}
		\end{example}
		
%		\begin{example}[$\Cyc_4$]
%			\begin{table}
%				\centering
%				\begin{tabular}{c | c c c c}
%					$\Cyc_4$ & $e$ & $g$  & $g^2$ & $g^3$ \\ \hline
%					$\chi_0$           & 1   & 1    & 1     & 1     \\
%					$\chi_1$           & 1   & $i$  & $-1$  & $-i$  \\
%					$\chi_2$           & 1   & $-1$ & $1$  & $-1$   \\
%					$\chi_3$           & 1   & $-i$ & $-1$   & $i$ \\ \hline\hline
%					$\chi_\Reg$ & 4 & 0 & 0 & 0
%				\end{tabular}
%			\end{table}
%		\end{example}
	\end{frame}
	
	\begin{frame}{Character table of $\Sym_3$}
		\begin{example}[$\Sym_3$]
			\begin{table}[hbt!]
			\centering
			
			\begin{tabular}{c | c c c}
				$S_3$         & $[(1)]$   & $[(1,2)]$  & $[(1,2,3)]$ \\
				$|[\sigma]|$    & $ 1$ & $ 3$ & $ 2$ \\ \hline
				$\chi_\Triv$       & 1       & 1       & 1       \\
				$\chi_\Alt$       & 1       & -1      & 1       \\
				$\chi_\Stan$       & 2       & 0       & -1      \\ \hline\hline
				$\chi_\Perm$       & 3       & 1       & 0       \\
%				$\chi_\Stan^2$ & 4       & 0       & 1       \\
				$\chi_\Reg$       & 6       & 0       & 0       		\end{tabular}
			
%			\caption{Complete character table of $S_3$. The representations above the doublestruck line are irreducibles, and those below are composed.}
%			\label{table:completecharS3}
		\end{table}
		\end{example}
	\end{frame}
	
	\begin{frame}{Character table of $\Sym_4$}
		\begin{example}[$\Sym_4$]
			
			\begin{itemize}
				\item We know: Trivial, Alternating, Standard.
			
				
			\end{itemize}
			
			\begin{table}[hbt!]
				\centering
				\begin{tabular}{c | c c c c c}
					$\Sym_4$         & $[(1)]$   & $[(1,2)]$  & $[(1,2,3)]$ & $[(1,2,3,4)]$ & $[(1,2)(3,4)]$ \\
					$|[\sigma]|$    & $ 1$ & $ 6$ & $ 8$ & $ 6$  & $ 3$      \\ \hline
					$\chi_\Triv$       & 1       & 1       & 1       & 1        & 1            \\
					$\chi_\Alt$       & 1       & -1      & 1       & -1       & 1            \\
%					$\chi_W$       & 2       & 0       & -1      & 0        & 2            \\
					$\chi_\Stan$       & 3       & 1       & 0       & -1       & -1           \\ \hline\hline
%					$\chi'_\Stan$ & 3       & -1      & 0       & 1        & -1           \\ 
					$\chi_\Perm$       & 4       & 2       & 1       & 0        & 0            
					%$\chi_{\bigwedge^2S}$    & 6       & 2       & 0       & 0        & 2            & $\bigwedge^2S \cong T \oplus S \oplus V$                     \\
%					$\chi_\Stan^2$ & 9       & 1       & 0       & 1        & 1            \\
					%			$\chi_R$       & 24      & 0       & 0       & 0        & 0            & $R \cong T \oplus A \oplus 2V \oplus 3S \oplus 3S'$
				\end{tabular}
%				\caption{Complete character table of $\Sym_4$. The representations above the doublestruck line are irreducibles, and those below are composed.}
%				\label{table:completecharS4}
			\end{table}\pause
			\begin{itemize}
				\item Standard$'$ = Alternating $\otimes$ Standard.\pause
				
				\item Standard $\otimes$ Standard = Trivial $\oplus$ Standard $\oplus$ Standard$'$ $\oplus$ Unknown.
			\end{itemize}
		\end{example}
	\end{frame}
	\begin{frame}{Character table of $\Sym_4$}
		\begin{example}[$\Sym_4$]
		\begin{table}[hbt!]
			\centering
			\begin{tabular}{c | c c c c c}
				$\Sym_4$         & $[(1)]$   & $[(1,2)]$  & $[(1,2,3)]$ & $[(1,2,3,4)]$ & $[(1,2)(3,4)]$ \\
				$|[\sigma]|$    & $ 1$ & $ 6$ & $ 8$ & $ 6$  & $ 3$      \\ \hline
				$\chi_\Triv$       & 1       & 1       & 1       & 1        & 1            \\
				$\chi_\Alt$       & 1       & -1      & 1       & -1       & 1            \\
				$\chi_W$       & 2       & 0       & -1      & 0        & 2            \\
				$\chi_\Stan$       & 3       & 1       & 0       & -1       & -1           \\
				$\chi'_\Stan$ & 3       & -1      & 0       & 1        & -1           \\ \hline\hline
				$\chi_\Perm$       & 4       & 2       & 1       & 0        & 0            \\
				%$\chi_{\bigwedge^2S}$    & 6       & 2       & 0       & 0        & 2            & $\bigwedge^2S \cong T \oplus S \oplus V$                     \\
				$\chi_\Stan^2$ & 9       & 1       & 0       & 1        & 1            \\
				%			$\chi_R$       & 24      & 0       & 0       & 0        & 0            & $R \cong T \oplus A \oplus 2V \oplus 3S \oplus 3S'$
			\end{tabular}
			%				\caption{Complete character table of $\Sym_4$. The representations above the doublestruck line are irreducibles, and those below are composed.}
			%				\label{table:completecharS4}
		\end{table}\end{example}
	\end{frame}
	
	\begin{frame}
		\vfill
		\centering
		This ends the main presentation.
		\vfill
	\end{frame}
	
	
%	\begin{frame}[allowframebreaks]{Further discussion}
%		
%		\begin{block}{Expected number of irreducible representations}
%			\begin{itemize}
%				\item \item The irreducible characters create an orthonormal system, analogous to the basis of a vector space.
%			
%				\item Hence, \begin{align*}
%					\{\text{\# Irreducible representations}\} &= \{\text{\# Entries in the character}\} \\
%					&= \{\text{\# Conjugacy classes}\}.
%				\end{align*}
%			\end{itemize}
%		\end{block}
%		\begin{block}{Characters of $\Sym_5$}
%			\begin{itemize}
%				\item There are $120$ elements in $\Sym_5$ and 7 conjugacy classes.
%				
%				\item We can find the trivial, alternating and the standard representations easily.
%				
%				\item There also is the tensor poduct of alternating and the standard.
%				
%				\item Square sum of these 4 repr. = $1^2 + 1^2 + 4^2 + 4^2 = 34 < 120$
%			\end{itemize}
%		\end{block}
%		\begin{example}[Alternating repr. inside regular repr. of $\Sym_n$]
%\begin{itemize}
%	\item Let $V$ be the regular representation of $\Sym_n$.
%	
%	\item Consider $w = \sum_{\sigma \in \Sym_n} \sgn(\sigma)\bas_\sigma$.\pause
%	
%	\item The action of a $\tau \in \Sym_n$ on $w$ is $\sgn(\tau) w$.\pause
%	
%	\item The 1-dimensional spanned by  $w$ is a $\Sym_n$-invariant subspace.
%\end{itemize}
%\end{example}
%		
%		\begin{example}[Alternating repr. inside regular repr. of $\Sym_n$]
%			\begin{itemize}
%				\item<1-> Let $V$ be the regular representation of $\Sym_n$.
%				
%				\item<1-> Consider $w = \sum_{\sigma \in \Sym_n} \sgn(\sigma)\bas_\sigma$.
%				
%				\item<2-> The action of a $\tau \in \Sym_n$ on $w$ is \begin{align*}
%					\tau \cdot w &= \sum_{\sigma \in \Sym_n} \sgn(\sigma)\bas_{\tau\sigma} & \text{(Linear action of $\tau$)} \\ 
%					&= \sgn(\tau)\sum_{\sigma \in \Sym_n} \sgn(\tau)\sgn(\sigma)\bas_{\tau\sigma} & \text{($\sgn^2 = 1$)}\\
%					&= \sgn(\tau)\sum_{\sigma \in \Sym_n} \sgn(\tau\sigma)\bas_{\tau\sigma} & \text{(sgn homom.)} \\
%					&= \sgn(\tau)\sum_{\sigma' \in \Sym_n} \sgn(\sigma')\bas_{\sigma'} & \text{(Let $\sigma' = \tau\sigma$)} \\
%					&= \sgn(\tau) w.
%				\end{align*}				
%			\end{itemize}
%		\end{example}
%				
%		\begin{block}{Abelian group}
%			The map $\rho_g$ is $G$-linear: \begin{align*}
%				\rho_g \rho_h &= \rho_{gh} \\
%				&= \rho_{hg} \\
%				&= \rho_h \rho_g.
%			\end{align*}
%			By Schur's lemma: $\rho_g : V \rightarrow V$, and $\rho_g = \lambda \cdot \id$. 
%		\end{block}
%		
%		Hence:
%		\begin{theorem}
%			Any irreducible representation of an abelian group is of degree 1.
%		\end{theorem}
%		
%		
%		The number of degree one representations of any group is the index of the largest abelian subgroup of that group, ie. the index of the commutator subgroup in the group $[G:[G,G]]$.
%	\end{frame}
	
\section{Further questions}

\begin{frame}{Further questions}
	\begin{example}[Precise equation for $\tau \cdot w$]
		\begin{itemize}
			\item Let $V$ be the regular representation of $\Sym_n$.
			
			\item Consider $w = \sum_{\sigma \in \Sym_n} \sgn(\sigma)\bas_\sigma$.
			
			\item The action of a $\tau \in \Sym_n$ on $w$ is \begin{align*}
				\tau \cdot w &= \sum_{\sigma \in \Sym_n} \sgn(\sigma)\bas_{\tau\sigma} & \text{(Linear action of $\tau$)} \\ 
				&= \sgn(\tau)\sum_{\sigma \in \Sym_n} \sgn(\tau)\sgn(\sigma)\bas_{\tau\sigma} & \text{($\sgn^2 = 1$)}\\
				&= \sgn(\tau)\sum_{\sigma \in \Sym_n} \sgn(\tau\sigma)\bas_{\tau\sigma} & \text{(sgn homom.)} \\
				&= \sgn(\tau)\sum_{\sigma' \in \Sym_n} \sgn(\sigma')\bas_{\sigma'} & \text{(Let $\sigma' = \tau\sigma$)} \\
				&= \sgn(\tau) w. & \text{\alert{Identical to alt. repr.!}}
			\end{align*}	
			
%			\item That is, $\tau \cdot w = \sgn(\tau) \cdot w$.
			
			\item $\Span(w)$ is a $\Sym_n$-invariant subspace.
		\end{itemize}
		\end{example}
\end{frame}

\begin{frame}{Further questions}
	\begin{block}{What could be said of total number of irreducible representations of a group?}
		\begin{itemize}
			\item Firstly, square sum of degrees = size of group.
			
			\item The irreducible characters create an orthonormal system, analogous to the basis of a vector space:
			\begin{align*}
									\{\text{\# Irreducible representations}\} &= \{\text{\# Entries in the character}\} \\
									&= \{\text{\# Conjugacy classes}\}.
			\end{align*}
			
			\item Assuming no non-zero class functions orthogonal to the irreducible characters (Fulton Harris prove this).
		\end{itemize}
	\end{block}
\end{frame}

\begin{frame}{Further questions}
	\begin{corollary}
		Any representation of an abelian group is of degree 1.
	\end{corollary}
	
	\begin{block}{In general, how many degree one representations are there of any group $G$?}
		\begin{itemize}
			\item For example $\Sym_3$¸ there is an abelian subgroup $\text{A}_3$ of index 2, hence there are 2 degree one representations.
			
			\item The index of the largest abelian subgroup = number of degree one reprs.
			
			\item Recall commutator subgroup $[G,G]$.
			
			\item In general, the number of degree one representations is the size of the quotient group $G/[G,G]$.
			
			\item For any abelian group $G$, $[G,G]=e$, hence there are $|G|$ degree one representations.
			
			\item For $\Sym_n$, $[\Sym_n,\Sym_n] = \text{A}_n$, and $[\Sym_n:\text{A}_n] = 2$.
		\end{itemize}
	\end{block}
\end{frame}
\begin{frame}{Further questions}
	\begin{block}{Can we find the character table for other groups, for example $\text{A}_4$?}
		\begin{itemize}
			\item The restriction of the representations of $\Sym_4$ to $\text{A}_4$ gives us only the trivial, and standard reprs. Conjugacy classes split! 
			
			\item For $\Sym_5$, we need the notion of $\text{Sym}^2(V)$ and $\bigwedge^2(V)$ to find additional representations.
			
			\item For $V$, abelian! 
			
			\item $\text{D}_{2n} = \Cyc_2 \times \Cyc_n$.
		\end{itemize}
	\end{block}
	
\end{frame}	

\end{document}