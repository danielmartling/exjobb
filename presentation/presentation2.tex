\documentclass[handout, 10pt]{beamer}
\usepackage[utf8]{inputenc}
\usepackage{mystyle} % See mystyle.sty for packages and own commands
%------------------------------------------------------------
%Title page
\title[Representation Theory]{Introduction to Representation Theory of Finite Groups}
%\subtitle{(Your Sub Title)}
\titlegraphic{\includegraphics[height=2.0cm]{Logos/widelogo.png}}
\author[Daniel Martling]{
	Daniel Martling}% ,
%	2nd Author,
%	3rd Author }
%\institute[]{Department of Physics \\Stockholm University}
\date{2024-06-12}

\usepackage{tikz, tikz-cd, quiver}

\newcommand{\GL}{\text{GL}}
\newcommand{\Sym}{\mathcal{S}}
\newcommand{\Cyc}{\mathcal{C}}
\newcommand{\sgn}{\text{sgn}}
\newcommand{\bas}{\mathbf{e}}
\newcommand{\fbas}{\mathbf{f}}
\newcommand{\id}{\text{id}}
\newcommand{\Span}{\text{Span}}
\newcommand{\CC}{\mathbb{C}}
\newcommand{\wvec}{w}

%---------------------TITLE PAGE---------------------------------------
\begin{document}
	\begin{frame}
		\maketitle
	\end{frame}
	%------------------------------------------------------------
	
	\logo{\includegraphics[height=1cm]{Logos/circlelogo.png}~%
	}
	
	
	%-------------------------------------------------------------------
	\section{Introduction}
	
	%-------------------------------------------------------------------
	
	\begin{frame}
		\frametitle{Table of Contents}
		\tableofcontents
	\end{frame}
	
	\section{Representation Theory}
	\subsection{Initial definitions}
	
	\begin{frame}{Initial definitions}
		Let $G$ be a finite group and let $V$ be a finite-dimensional vector space over $\mathbb{C}$. \pause
		
		\begin{definition}
			A representation of $G$ in $V$ is a homomorphism
			\begin{align*}
				\rho : G \rightarrow \GL(V),
			\end{align*}
			that assigns a linear map $\rho_g: V \rightarrow V$ to every $g \in G$.
		\end{definition} \pause
		
		\begin{itemize}
			\item Homomorphism: 
			$\rho_g \rho_h = \rho_{gh} \quad\pause \& \quad \rho_e = \text{id}, \quad \text{and} \quad \rho_{g^{-1}} = \rho_g^{-1}$.
			
			\item $\dim V =$ the \textit{degree} of the representation. 
			
			\item A vector space with such a map is called a \alert{representation space of $G$}.
			
			\item $\rho$, $\{\rho_g\}_{g\in G}$ and $V$ are all referred to as the \alert{representation of $G$.} \pause
			
			\item A choice of basis gives us a \alert{matrix representation}. It is not canonical and dependent on this basis. 
		\end{itemize}		
	\end{frame}
	
	\subsection{Tensor algebra}
	
	\begin{frame}{Tensor operations}
		\framesubtitle{Quick introduction}
		
		Let $V$ and $W$ be vector spaces.
		
		Let $v = (v_1, \dots, v_m)^T \in V$ and $w = (w_1, \dots, w_n)^T \in W$. 
		
		Then the following are also vector spaces:
		
		\vfill
		
		\begin{block}{Direct sum $V \oplus W$}
			\begin{itemize}
				\item $v \oplus w = \begin{pmatrix}
					v \\ w
				\end{pmatrix} = (v_1, \dots, v_m, w_1, \dots, w_n)^T \in V \oplus W$
				\item $\dim V \oplus W = \dim V + \dim W$
			\end{itemize}
		\end{block}
		\vfill
		\begin{block}{Tensor product $V \otimes W$}
			\begin{itemize}
				\item $v \otimes w = (v_iw)_{i=1}^{m} = (v_1 w_1, \dots, v_m w_1, v_1 w_2, \dots, v_m w_2, \dots, v_1 w_n, \dots, v_m w_n)^T \in V \otimes W$
				\item $\dim V \otimes W = \dim V \cdot \dim W$
			\end{itemize}
		\end{block}		
	\end{frame}
	
	\begin{frame}{Tensor operations on representation spaces}
		Let $\rho^V: G \rightarrow \GL(V)$ and $\rho^W: G \rightarrow \GL(W)$ be representations. 
		
%		Remember: The $\rho^V_g$ and $\rho^W_g$ are matrices. 
		
		Then the following are also representations:
		\vfill
		\begin{block}{Direct sum of representations}
			\begin{itemize}
				\item $\rho^{V \oplus W} = \rho^V \oplus \rho^W$ is the block matrix $\begin{pmatrix}
					\rho^V & 0 \\ 0 & \rho^W
				\end{pmatrix}$.
				
				\item It is a repr. given by 
				\begin{align*}
%					(	\rho^V \oplus \rho^W) \cdot (v \oplus w) &= 
					\begin{pmatrix}
						\rho^V & 0 \\ 0 & \rho^W
					\end{pmatrix} \cdot \begin{pmatrix}
						v \\ w
					\end{pmatrix} 
					= \begin{pmatrix}
						\rho^V  v \\ \rho^W  w
					\end{pmatrix} 
					= (\rho^V v) \oplus (\rho^W w).
				\end{align*}
			\end{itemize}
		\end{block}
		
		\begin{block}{Tensor product of representations}
			\begin{itemize}
				\item $\rho^{V \otimes W} = \rho^V \otimes \rho^W$ is the block matrix $(\rho^V_{ij} \rho^W)$.
				
				\item It is a repr. given by 
				\begin{align*}
				(\rho^V_{ij} \rho^W) \cdot (v_i w)
					= (\rho^V_{ij}v_i \rho^Ww)
					= (\rho^V v) \otimes (\rho^W w).
				\end{align*}
			\end{itemize}
		\end{block}
	\end{frame}
	
	\subsection{Examples}
	\begin{frame}{Examples}
		\begin{block}{Trivial representation}
			\begin{itemize}
				\item For any group, the trivial representation is a map
				\begin{align*}
					\rho_g = 1, \quad \text{for all $g \in G$.}
				\end{align*}
				
				\item Trivially a homomorphism.
				
				\item Degree = 1.
			\end{itemize}
			
		\end{block}
	\end{frame}
	\begin{frame}{Examples cont'd.}
		\begin{block}{Alternating representation}
			\begin{itemize}
				\item For $\Sym_n$, the alternating representation is the ``sign'' map
				\begin{align*}
						\sgn &: \Sym_n \rightarrow \{\pm 1\} \\
					\rho &: \sigma \mapsto \sgn(\sigma), \quad \text{for all } \sigma \in \Sym_n.
					%				\sgn &: \Sym_n \rightarrow \{\pm 1\} \\
					%				\sgn &: \sigma \mapsto (-1)^s,
				\end{align*}
				
				\item It is a homomorphism: for any $\sigma, \tau \in \Sym_n$,
				\begin{align*}
					\sgn(\sigma)\sgn(\tau) &= (-1)^s(-1)^t \\
					&= (-1)^{s+t} \\
					&= \sgn(\sigma\tau).
				\end{align*}
				
				\item Degree = 1.
			\end{itemize}
		\end{block}
	\end{frame}
	
	\begin{frame}{Examples cont'd.}
		\begin{block}{Permutation representation}
			\begin{itemize}
				\item Let $G$ be a finite group and let $X$ be a finite set.
				
				\item $G$ acts on $X$ by permutation:
				\begin{align*}
					\sigma &: G \times X \rightarrow X, \\
					\sigma_g &: x \mapsto gx.
				\end{align*}
				
				\item Let the basis $(\bas_x)_{x \in X}$ span $V$.
				
				\item Then the permutation representation is a map
				\begin{align*}
					\rho &: G \rightarrow \GL(V), \\
					\rho_g &: \bas_x \mapsto \bas_{gx}.
				\end{align*}
				
				\item The matrices $\rho_g$ are \alert{permutation matrices}.
			\end{itemize}
		\end{block}
	\end{frame}
	
	\begin{frame}{Examples cont'd.}
		\begin{block}{Symmetric group $\Sym_n$}
			\begin{itemize}
				\item Let $G = \Sym_n$, the set of permutations of $X = \{1, 2, \dots, n\}$.
				
				\item $\rho : \Sym_n \rightarrow \GL(V)$ def. by $\rho_\sigma : \bas_i \rightarrow \bas_{\sigma(i)}$
				
				\item $\rho_\sigma = (r_{ij})_{n \times n}$, where $r_{ij} = \begin{cases}
					1, \quad \text{if} \quad j = \sigma(i), \\
					0, \quad \text{otherwise.}
				\end{cases}$
			\end{itemize}
		\end{block}
		\begin{example}[Matrix representations of $\Sym_3$]
%			\begin{table}[hbt!]
				\centering
				\begin{tabular}{r r r}
					$\rho_{(1)} = 
					\begin{pmatrix}
						1 & 0 & 0 \\
						0 & 1 & 0 \\
						0 & 0 & 1
					\end{pmatrix}$, & 
					$\rho_{(1,2,3)} = 
					\begin{pmatrix}
						0 & 0 & 1 \\
						1 & 0 & 0 \\
						0 & 1 & 0
					\end{pmatrix}$, & 
					$\rho_{(1,3,2)} = 
					\begin{pmatrix}
						0 & 1 & 0 \\
						0 & 0 & 1 \\
						1 & 0 & 0
					\end{pmatrix}$, \\ & & \\
					$\rho_{(1,2)} = 
					\begin{pmatrix}
						0 & 1 & 0 \\
						1 & 0 & 0 \\
						0 & 0 & 1
					\end{pmatrix}$, &
					$\rho_{(1,3)} = 
					\begin{pmatrix}
						0 & 0 & 1 \\
						0 & 1 & 0 \\
						1 & 0 & 0
					\end{pmatrix}$, &
					$\rho_{(2,3)} = 
					\begin{pmatrix}
						1 & 0 & 0 \\
						0 & 0 & 1 \\
						0 & 1 & 0
					\end{pmatrix}$.
				\end{tabular}
%				\caption{Matrix representations of $\Sym_3$}
%				\label{table:permS3}
%			\end{table}
		\end{example}
	\end{frame}
	
	\begin{frame}{Examples cont'd.}
		\begin{example}[Matrix representations of $\Sym_4$]
		
%		\begin{table}[hbt\Perm!]
			\centering
			\begin{tabular}{ r }
				$\rho_{(1)} = \left(\begin{matrix}
					1 & 0 & 0 & 0 \\
					0 & 1 & 0 & 0 \\
					0 & 0 & 1 & 0 \\
					0 & 0 & 0 & 1
				\end{matrix}\right)$,  \\ \\
			\end{tabular}
			\begin{tabular}{ r r }
				$\rho_{(1,2)} = \left(\begin{matrix}
					0 & 1 & 0 & 0 \\
					1 & 0 & 0 & 0 \\
					0 & 0 & 1 & 0 \\
					0 & 0 & 0 & 1
				\end{matrix}\right)$, &
				$\rho_{(1,2)(3,4)} = \left(\begin{matrix}
					0 & 1 & 0 & 0 \\
					1 & 0 & 0 & 0 \\
					0 & 0 & 0 & 1 \\
					0 & 0 & 1 & 0
				\end{matrix}\right)$, \\ & \\

				$\rho_{(1,2,3)} = \left(\begin{matrix}
					0 & 0 & 1 & 0 \\
					1 & 0 & 0 & 0 \\
					0 & 1 & 0 & 0 \\
					0 & 0 & 0 & 1
				\end{matrix}\right)$, &
				$\rho_{(1,2,3,4)} = \left(\begin{matrix}
					0 & 0 & 0 & 1 \\
					1 & 0 & 0 & 0 \\
					0 & 1 & 0 & 0 \\
					0 & 0 & 1 & 0
				\end{matrix}\right)$.
			\end{tabular}
%			\caption{Some matrix representations of $\Sym_4$}
%			\label{table:permS4}
%		\end{table}
\end{example}
	\end{frame}
	
	\begin{frame}{Examples cont'd.}
		\begin{block}{Regular representation}
			\begin{itemize}
				\item Let $X = G$. \textit{($G$ acts on itself.)}
				
				\item The regular representation is a map
				\begin{align*}
					\rho &: G \rightarrow \GL(V), \\
					\rho_g &: \bas_h \mapsto \bas_{gh}.
				\end{align*}
			\end{itemize}
		\end{block}
		\begin{block}{Cyclic group $\Cyc_n$}
			\begin{itemize}
				\item $\Cyc_n = \{ e, g, g^2, \dots, g^{n-1}\}$.
				
				\item $g^n = e$.
				
				\item $g \cdot g^a = g^{a+1}$.
			\end{itemize}
		\end{block}
	\end{frame}
	
	\begin{frame}{Examples cont'd.}
		\begin{example}[Matrix representations of $\Cyc_3$]
			\begin{itemize}
				\item In $e, g, g^2$-basis:
%				\begin{align*}
	$
					\rho_e = \id, \quad \rho_g = \begin{pmatrix}
						0&0&1 \\ 1&0&0 \\ 0&1&0
					\end{pmatrix}, \quad \rho_{g^2} = \begin{pmatrix}
						0&1&0 \\ 0&0&1 \\ 1&0&0 
					\end{pmatrix}.
					$
%				\end{align*}
				
				\item Switch basis: $\begin{cases}
						\fbas_1 = e + g + g^2, \\
						\fbas_2 = e + \omega^2 g + \omega g^2, \\
						\fbas_3 = e + \omega g + \omega^2 g^2.
					\end{cases}$
				
				\item Then:
%				\begin{align*}
	$
					\rho_e = \id, \quad
					\rho_g = \begin{pmatrix}
						1&0&0 \\ 0&\omega&0 \\ 0&0&\omega^2
					\end{pmatrix}, \quad \text{and} \quad
					\rho_{g^2} = \begin{pmatrix}
						1&0&0 \\ 0&\omega^2&0 \\ 0&0&\omega
					\end{pmatrix}.
					$
%				\end{align*}

				\item Clearly, $\rho_g = (1) \oplus (\omega) \oplus (\omega^2) = \rho_g^0 \oplus \rho_g^1 \oplus \rho_g^2$.
				
				\item That is, $\{\text{Regular repr.}\} = \bigoplus \{\text{Deg. 1 repr.}\}$
			\end{itemize}
			
		\end{example}
	\end{frame}
	
	\subsection{Subrepresentations and irreducible representations}
	
	\begin{frame}{Subrepresentations}
		\begin{definition}[$G$-linear map]
			A vector space map $\varphi : V \rightarrow W$ is called $G$-linear if it commutes with the action $G$.
			\begin{columns}
				\column{0.4\linewidth}
				\[\begin{tikzcd}[ampersand replacement=\&]
					V \&\& W \\
					\\
					V \&\& W
					\arrow["\varphi", from=1-1, to=1-3]
					\arrow["g"', from=1-1, to=3-1]
					\arrow["g", from=1-3, to=3-3]
					\arrow["\varphi", from=3-1, to=3-3]
				\end{tikzcd}\]
				\column{0.4\linewidth}
				\begin{align*}
					\varphi(g \cdot v) = g \cdot \varphi(v)
				\end{align*}
			\end{columns}
			% https://q.uiver.app/#q=WzAsNCxbMCwwLCJWIl0sWzAsMiwiViJdLFsyLDIsIlciXSxbMiwwLCJXIl0sWzAsMywiXFx2YXJwaGkiXSxbMSwyLCJcXHZhcnBoaSJdLFswLDEsImciLDJdLFszLDIsImciXV0=
		\end{definition}
			
		\begin{definition}[$G$-invariant subspace]
			A vector subspace $W$ of $V$ is called $G$-invariant if it is left fixed by the action of $G$.
		\end{definition}
	\end{frame}	
	
	\begin{frame}{Subrepresentations cont'd.}
		\begin{block}{Now:}
			\begin{itemize}
				\item Let $\rho^V : G \rightarrow \GL(V)$.
				
				\item Let $W$ be a $G$-invariant subspace of $V$.
				
				\item The restriction of $\rho^V$ to $W$, $\rho^{V|_W}$ is a isomorphism of $W$ onto itself.
				
				\item Hence, $\rho^{V|_W}: G \rightarrow \GL(W)$ is a representation of $G$ in $W$.
			\end{itemize}
		\end{block}
		
		\begin{definition}[Subrepresentation]
			A restriction of a $\rho: G \rightarrow \GL(V)$ to a $G$-invariant subspace $W \leq V$ is called a subrepresentation of $\rho$.
		\end{definition}
		
		\begin{example}[Trivial subspaces]
			Any representation has itself and the zero space as trivial, non-proper subrepresentations.
		\end{example}
	\end{frame}
	
	\begin{frame}{Subrepresentations cont'd.}
		\begin{example}[Trivial repr. inside permutation repr.]
			\begin{itemize}
				\item Let $V$ be the permutation representation. 
				
				\item Consider:
				\begin{align*}
					W = \Span\left\lbrace \sum_i \bas_i \right\rbrace.
				\end{align*}
				
				\item Any $g \in G$ will reorder the sum, hence $W$ is $G$-invariant, and $\rho_g = 1$.
				
				\item Hence, the trivial representation is a subrepresentation of the permutation representation (and the regular representation).
			\end{itemize}		
		\end{example}
	\end{frame}
	
	\begin{frame}{Subrepresentations cont'd}
		\begin{example}[Alternating repr. inside regular repr. of $\Sym_n$]
			\begin{itemize}
				\item Let $V$ be the regular representation of $\Sym_n$.
				
				\item Consider
				\begin{align*}
					W = \Span \left\lbrace \sum_{\sigma \in \Sym_n} \sgn(\sigma)\bas_\sigma \right\rbrace.
				\end{align*}
				
				\item An element $w \in W$ is $a \cdot \sum_{\sigma \in \Sym_n} \sgn(\sigma)\bas_\sigma $, for some $ a \in \CC$.
				
				\item The action of a $\tau \in \Sym_n$ on a $w \in W$ is

			\end{itemize}
		\end{example}
	\end{frame}
	
	\begin{frame}{Subrepresentations cont'd.}
		Kernel and image
		
		Complementary subrepresentations
		
		Standard representation of $\Sym_n$
		
		Example Standard of $\Sym_3$
		
		Example Decomp $\Sym_3$
		
	\end{frame}
	
	\begin{frame}{Irreducible representations}
		Def. irred.
		
		Examples
		
		Maschke's
		
		Schur's
		
		Abelian groups??
	\end{frame}
	
	\section{Character Theory}
	
	\begin{frame}{Trace of matrix}
		content...
	\end{frame}
	
	\begin{frame}{Def. Character}
		content...
	\end{frame}
	
	\begin{frame}{Properties of characters}
		content...
	\end{frame}
	
	\begin{frame}{Character tables}
		content...
	\end{frame}
	
	\begin{frame}{Examples}
		content...
	\end{frame}
	
	\begin{frame}{Orthogonality relations of characters}
		content...
	\end{frame}
	
	\begin{frame}{Complete decomposition}
		content...
	\end{frame}
	
	\begin{frame}{Complete decomposition of the regular representation}
		content...
	\end{frame}
	
	\begin{frame}{Character table of cyclic groups}
		content...
	\end{frame}
	
	\begin{frame}{Character table of $\Sym_3$}
		content...
	\end{frame}
	
	\begin{frame}{Character table of $\Sym_4$}
		content...
	\end{frame}
	
	\begin{frame}{Further discussion}
		Character table of $\Sym_5$
		
		Number of deg 1 repr of any group
	\end{frame}
	
%	
%	%-------------------------------------------------------------------
%	
%	%---------------------EXEMPELSLIDES---------------------------------------
%	\begin{frame}{Equations of motion} 
%		
%		\begin{columns}
%			\column{0.4\textwidth}
%			\begin{block}{Newton's second law}
%				$m\ddot{x} = F(\dot{x}, x, t)$
%			\end{block}
%			\begin{block}{Schrödinger's equation}
%				${\displaystyle i\hbar {\frac {d}{dt}}\vert \Psi (t)\rangle ={\hat {H}}\vert \Psi (t)\rangle }$
%			\end{block}
%			\begin{block}{Ampère's circuital law}
%				$ \nabla \times \mathbf{B}=\frac{1}{c}\left(4 \pi \mathbf{J}+\frac{\partial \mathbf{E}}{\partial t}\right)$
%			\end{block}
%		\end{columns}
%	\end{frame}
%	%---------------------SLIDE---------------------------------------
%	\begin{frame}{Important question}
%		Important question again
%	\end{frame}
%	%---------------------SLIDE---------------------------------------
%	\section{Important section}
%	\begin{frame}{Bullet points}
%		\begin{itemize}
%			\item Item A
%			\item Item B
%			\item Item C
%		\end{itemize}
%	\end{frame}
%	%---------------------SLIDE---------------------------------------
%	\begin{frame}{Using pause}
%		This is a sentence. \pause 
%		And this too. \pause
%		\alert{Bye}. 
%	\end{frame}
%	
%	%---------------------SLIDE--------------------------------------
%	\begin{frame}{A Theorem}
%		\begin{theorem}[Freshman's Dream] 
%			$(a+b)^p \equiv a^p + b^p \, (mod \,p)$ if p is a prime number.
%		\end{theorem} \pause
%		\begin{proof}
%			A valid proof.
%		\end{proof}
%		\begin{example}
%			Maybe an example?
%		\end{example}
%	\end{frame}
%	
%	%-----------------------------SLIDE --------------------------------------
%	
%	%Below must be compiled using LuLaTex (for emoji package):
%	
%	% \begin{frame}{Summary}
%		
%		% \begin{itemize}
%			%     \item[\emoji{check-mark-button}] Concept A
%			%      \item[\emoji{check-mark-button}] Concept B
%			%      \item[\emoji{check-mark-button}] Concept C
%			% \end{itemize}
%		% \end{frame}
%	
%	
%	%-----------------------------SLIDE --------------------------------------
%	
%	\begin{frame}{How do you write a thesis?}
%		
%		\begin{enumerate}
%			\item Eat
%			\item Sleep
%			\item Rave
%			\item Repeat
%		\end{enumerate}
%	\end{frame}
%	
%	
%	%-----------------------------SLIDE --------------------------------------
%	
%%	\begin{frame}{Frame title}{Frame subtitle}
%%		This is how you can cite \cite{Dirac}.
%%	\end{frame}
%	%-----------------------------SLIDE --------------------------------------
%	\begin{frame}{The end.}
%		\begin{columns}
%			\column{0.4\textwidth}
%			This is a column.
%			
%			\column{0.4\textwidth}
%			\includegraphics[width=0.9\textwidth]{Figures/meme.png}    
%		\end{columns}    
%		
%		
%	\end{frame}
%	
%	%---------------------------------------------------------
%%	\section{References}
%%	\begin{frame}[allowframebreaks]\frametitle{References}
%%		\bibliographystyle{apalike}
%%		\bibliography{bib}
%%	\end{frame}
%	
	
\end{document}