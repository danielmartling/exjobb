% your first section
\clearpage{\thispagestyle{empty}\cleardoublepage}
\section{Introduction}

Lorem ipsum dolor sit amet, consectetur adipiscing elit, sed do eiusmod tempor incididunt ut labore et dolore magna aliqua.
Ut enim ad minim veniam, quis nostrud exercitation ullamco laboris nisi ut aliquip ex ea commodo consequat. 
Duis aute irure dolor in reprehenderit in voluptate velit esse cillum dolore eu fugiat nulla pariatur. 
Excepteur sint occaecat cupidatat non proident, sunt in culpa qui officia deserunt mollit anim id est laborum.

You can also use swedish letters ä ö å, Ä, Ö, Å. Unicode characters in general should work.


%Another section
\clearpage{\thispagestyle{empty}\cleardoublepage}
\section{Theorems and definitions}

\begin{definition}[Test av definition]
	Test av definition
\end{definition}

\begin{problem}[Test av problem]
	Test av problem
\end{problem}

\begin{conjecture}[Test av conjecture]
	Test av conjecture
\end{conjecture}

\begin{theorem}[Test av theorem]
	Test av theorem
\end{theorem}

\begin{proposition}[Test av proposition]
	Test av proposition
\end{proposition}

\begin{corollary}[Test av corollary]
	Test av corollary
\end{corollary}

\begin{lemma}[Test av lemma]
	Test av lemma
\end{lemma}

\begin{question}[Test av question]
	Test av question
\end{question}

\begin{remark}[Test av remark]
	Test av remark
\end{remark}

\begin{example}[Test av example]
	Test av example
\end{example}

\begin{notation}[Test av notation]
	Test av notation
\end{notation}

\begin{note}[Notering]
	This note's for you
\end{note}


\begin{definition}
	This is a definition!
\end{definition}

\begin{theorem}\label{thm: label}
	This is a theorem!
\end{theorem}

To cross reference: By Theorem~\ref{thm: label} we can say...
\begin{lemma}
	This is a lemma.
\end{lemma}
\begin{proof}
	Put here the proof of the lemma!
\end{proof}


\begin{proof}[Proof of Theorem \ref{thm: label}]
	Another proof
\end{proof}


\begin{corollary}
	here is a corollary!
\end{corollary}

If you want to write a beautiful equation in line do like that $e^{i\pi}+1=0$.
You might also want to write it on its own line
\[
e^{i\pi}+1=0.
\]
If you think that you are going to need to cross reference your equation:
\begin{equation}\label{eq:beautiful}
	e^{i\pi}+1=0
\end{equation}
and this is how you cross reference it \eqref{eq:beautiful}
You might also want to have aligned equations on more lines.
\begin{align}
	e^{i\pi}+1&=0\\
	(\cos\theta)^2+(\sin\theta)^2 &=1
\end{align}

If you need to cross reference something in the bibliography, use the cite command: \cite[p. 45]{Biggs2002}. 
The actual references are stored in a different file, called bibliographyFile.bib 

You can reference most things you found useful,
for example, books \cite{Biggs2002,Knuth1998ArtOfProgramming,LascouxPolynomialsBook}.
Research articles and preprints, \cite{AlexanderssonSulzgruber2019,Sherman2015x}.
Or another thesis, \cite{Teff2013}.

You can also cite online sources. For example, the Encyclopedia of integer sequences, \cite{OEIS},
or a YouTube video \cite{Bazett2018yt}. 
It is even possible to refer to a discussion you had with someone, \cite{PrivateCommunicationWithJustin}.
This is usually rare, it is better to thank them in the acknowledgements section (if they were helpful).



%another section
\clearpage{\thispagestyle{empty}\cleardoublepage}
\section{Tables and graphs}
Here a few examples of tables and graphs.
%A subsection
\subsection{Tables}
\begin{center}
	\begin{tabular}{c c c c c c c c}
		\arrayrulecolor{Azzurro}
		\hline
		{\bfseries Code} & {\bfseries CdL} & {\bfseries xxx} & {\bfseries $T_{setup/lotto}$} & {\bfseries $T_{lav/st}$} & {\bfseries $T_{proc/pezzo}$} & {\bfseries Quantitity} & {\bfseries $T_{tot}$}\\
		\hline
		100 & 4 & 250 & 25 & 0,5 & 0,6 & 1 & 0,6\\
		111 & 2 & 250 & 20 & 2 & 2,08 & 1 & 2,08 \\
		111 & 3 & 250 & 15 & 1,5 & 1,56 & 1 & 1,56 \\
		112 & 2 & 250 & 20 & 2,5 & 2,58 & 1 & 2,58 \\
		112 & 3 & 250 & 15 & 2 & 2,06 & 1 & 2,06\\
		113 & 3 & 500 & 15 & 1 & 1,03 & 2 & 2,06\\
		120 & 1 & 50 & 30 & 2 & 2,6 & 0,1 & 0,26\\
		121 & 1 & 25 & 30 & 3 & 4,2 & 0,1 & 0,42 \\
		121 & 1 & 25 & 30 & 2,5 & 3,7 & 0,1 & 0,37 \\
		\hline
	\end{tabular}
\end{center}



%another section
\clearpage{\thispagestyle{empty}\cleardoublepage}

\section{Insert a figure}
\begin{figure}[H]
	\centering
	\includegraphics[width=1\textwidth]{grafo.png}
	\caption{\label{fig:1} put the caption here}
\end{figure}

\subsection{Footnote}

You can create a footnote like this.\footnote{I created a footnote.}



\clearpage{\thispagestyle{empty}\cleardoublepage}
\section{Conclusion}

Remember the internet is full of places that can help you write a beautiful document with latex.
Some example are \url{overleaf.com} and \verb!stackexchange! just ask Google! 
In addition with an SU e-mail address you should be able to 
create (for free) a premium account on \url{overleaf.com}, 
so you do not need to download a tex editor and you will have
the full latex distribution at your hand (if you are connected to the internet).


$$ \NN \ \ZZ \ \QQ \ \RR \ \CC \ \KK \ \FF$$
$$ \1 \ \0 $$

$$ \GL(V),  \SL_n(\KK), \SymP^k V, \AltP^k V, \Hom(V,W), \SO, \SU, \id, \Stab(g), \Cent(g), \Norm(g),$$ $$ \Type(\sigma), \Dih_{2n}, \Sym_n, \Alt_n, \Klein, \Cyc_n, \sgn(\sigma), \Span(\bas_i)_i $$

$$ \bas \ihat \jhat \ghat \hhat \vhat \what \uhat \vvec \uvec \wvec $$